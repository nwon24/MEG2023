\documentclass[a4paper]{article}

\newcommand{\theterm}{2}
\newcommand{\theweek}{5}
\newcommand{\thepdftitle}{MEG 2023 Term \theterm\ Week \theweek\ Handout Solutions}
\newcommand{\thedisplaytitle}{Term \theterm\ Week \theweek\ Handout Solutions}

\title{{\thepdftitle}}
\author{Nathan Wong\and Tom Yan}
\date{2023}

% \newcommand{\marginfn}[1]{\marginpar{\footnotemark}\footnotetext{#1}}
\newcommand{\marginnote}[1]{\marginpar{\footnotesize{#1}}}
\newcommand{\marginfnote}[1]{\footnotemark\marginpar{\footnotemark[\value{footnote}]\footnotesize{#1}}}
\usepackage{geometry}
% \geometry{a4paper,left=24.8mm,top=27.4mm,headsep=2\baselineskip,textwidth=107mm,marginparsep=8.2mm,marginparwidth=49.4mm,textheight=49\baselineskip,headheight=\baselineskip}
\geometry{a4paper,left=1in,top=1in,bottom=1in,headsep=2\baselineskip,textwidth=107mm,marginparsep=8.2mm,marginparwidth=49.4mm,textheight=49\baselineskip,headheight=\baselineskip}
\usepackage[bf,tiny]{titlesec}
% \usepackage{fancyhdr}
\usepackage{epigraph}
% \usepackage[indent=0pt,skip=10pt]{parskip}

\usepackage{amsmath}
\usepackage{amsthm}
\newtheorem{theorem}{Theorem}
\usepackage{amssymb}
\let\mathbbalt\mathbb

\usepackage{fontspec}
\usepackage{unicode-math}
\let\mathbb\mathbbalt

\newcommand{\naturals}{\mathbb{N}}
\newcommand{\reals}{\mathbb{R}}
\newcommand{\rationals}{\mathbb{Q}}
\newcommand{\integers}{\mathbb{Z}}

\usepackage[pdfusetitle]{hyperref}

\newcommand{\myquote}[2]{%
  \begin{quote}
    \emph{#1}
    \begin{flushright}---{#2}
    \end{flushright}
  \end{quote}}
\pagestyle{empty}
\begin{document}
\noindent Melbourne High School\\
\noindent Maths Extension Group 2023\\
\noindent \textbf{\thedisplaytitle}\\
\section*{The Fundamental Theorem of Arithmetic}
\begin{enumerate}
  \item
  \begin{enumerate}
    \item If the divisors of \(m\) are \[a_1,a_2,\ldots,a_r\] and
  the divisors of \(n\) are \[b_1,b_2,\ldots,b_s\] the divisors
  of \(mn\) are exactly all the products of all pairs of \(a\)'s
  and \(b\)'s. Therefore
  \begin{displaymath}
    \begin{split}
    \sigma(mn)&=a_1\sum_{k=1}^{s}b_k+a_2\sum_{k=1}^{s}b_k+\cdots+a_r\sum_{k=1}^{s}b_k\\ &=\sum_{k=1}^{r}a_k\cdot\sum_{k=1}^{s}b_k\\&=\sigma(m)\sigma(n).
    \end{split}
  \end{displaymath}
\item Since \[\sigma(p^k)=\frac{p^{k+1}-1}{p-1}\] the formula
  for \(\sigma(n)\) is just the product of these terms for all
  powers of primes that divide \(n\). Therefore if
  \[n=\prod_{k=1}^rp_k^{a_k}\] then \[\sigma(n)=\prod_{k=1}^{r}\frac{p_k^{a_k+1}-1}{p_k-1}.\]
\end{enumerate}
\item
  \begin{enumerate}
  \item Let \(p\) be a prime and \(a\) be a number
    less than \(p\). Suppose that the proposition is false and
    that there exist numbers \(x\equiv b,b',b'',\ldots\) all
    less than \(p\)
    such that \[ax\equiv0\pmod{p}.\]
    Let \(b\) be the smallest of these.

    Since \(b<p\), it must be that \(p\) lies between
    two successive multiples of \(b\); that is, there exists
    some \(m\) such that \[bm<p<b(m+1).\]
    From \(bm<p\) it follows that \(p-bm>0\), and from
    \(p<bm+b\) it follows that \(p-bm<b\). Put \(c=p-bm\).
    Then we have \(0<c<b\). But \[ac\equiv a(p-bm)\equiv ap-abm\equiv0\pmod{p}\] because clearly \(ap\equiv0\), and by hypothesis \(ab\equiv0\) too.
    Therefore \(ac\equiv0\), which contradicts the minimality of \(b\)
    because \(c<b\). This contradiction completes the proof.
  \item If \(a\not\equiv0\) and \(b\not\equiv0\), their least positive
    residues, say \(\alpha\) and \(\beta\),  are also not congruent to
    \(0\). If \[ab\equiv0\pmod{p}\] then \[\alpha\beta\equiv0\pmod{p},\]
    but this contradicts the previous result as \(0<\alpha,\beta<p\).
  \end{enumerate}
  \item
    \begin{enumerate}
    \item Let the two sets of prime factors be
      arranged in ascending order, so that \[p_1<p_2<\cdots<p_r\]
      and \[q_1<q_2<\cdots<q_s.\] None of the \(p\)'s can be a
      \(q\) because otherwise it could be cancelled out, and we
      know the resulting number cannot have two different factorisations
      because it is smaller than \(n\). Therefore either \(p_1<q_1\)
      or \(q_1<p_1\). Since \(p_1^2\le n\) and \(q_1^2\le n\), we have
      \(p_1q_1<n\), which implies \(n-p_1q_1>0\).
    \item Since both \(p_1\) and \(q_1\) divide \(n\), they also
      divide \(N=n-p_1q_1\). Rearranging for \(n\), we get
      \(n=N+p_1q_1\), and so \(p_1q_1\) divides \(n\).
    \item If \(p_1q_1\) divides \(n\), then \(q_1\) divides
      \(p_2\ldots p_r\). This is impossible because \(n/p_1\), being
      less than \(n\), has a unique factorisation consisting of
      exactly those primes \(p_2\) up to \(p_r\), and since no \(q\)
      is a \(p\), it cannot be that \(q_1\) occurs in the factorisation.
      This contradiction completes the proof.
    \end{enumerate}
\end{enumerate}
\end{document}