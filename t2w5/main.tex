\documentclass{article}
\usepackage{graphicx} % Required for inserting images

\title{Angle Chasing problems}
\author{Tom Yan}
\date{May 2023}

\begin{document}

\maketitle

\section{Introduction}
1. Quadrilateral $WXYZ$ is a quadrilateral with perpendicular diagonals. If $\angle WZX = 30^{\circ}$, $\angle XWY = 40^{\circ}$, and $\angle WYZ = 50^{\circ}$, find $\angle WZY$.\\\\
2. Two parallel lines are tangent to a circle with centre $O$. A third line, also tangent to
the circle, meets the two parallel lines at $A$ and $B$. Prove that $AO$ is perpendicular to $OB$\\\\
3. (BAMO 1999/2) Let $O = (0, 0)$, $A = (0, a)$, and $B = (0, b)$, where $0 < a<b$ are reals. Let $\tau$ be a circle with diameter $AB$ and let $P$ be any other point on $\tau$. Line
$PA$ meets the x-axis again at $Q$. Prove that $\angle BQP = \angle BOP$. \\\\
4. (AMO 2008)  Let $ABCD$ be a convex quadrilateral. Suppose there is a point $P$ on the
segment $AB$ with $\angle APD = \angle BPC = 45^{\circ}$. If $Q$ is the intersection of the line AB with the perpendicular bisector of CD, prove that $CQD = 90^{\circ}$.\\\\
5. (EGMO 2023/2) We are given an acute triangle $ABC$. Let $D$ be the point on its circumcircle such that $AD$ is a diameter. Suppose that points $K$ and $L$ lie on segments $AB$ and $AC$, respectively, and that $DK$ and $DL$ are tangent to circle $AKL$.
Show that line $KL$ passes through the orthocentre of triangle $ABC$.
\end{document}
