\documentclass{beamer}
\title{Pi Day}
\author{Maths Extension Group}
\institute{Melbourne High School}
\date{2023}

\AtBeginSection[]
{
\begin{frame}
\frametitle{Table of Contents}
\tableofcontents[currentsection]
\end{frame}
}

\begin{document}
\frame{\titlepage}
%\section{What is \(\pi\)?}
%\begin{frame}
%\frametitle{What is \(\pi\)?}
%\begin{itemize}
%\item Ratio of a circle's circumference to its diameter?\pause
%\item The circumference of a semicircle with unit radius?\pause
%\item Some number around \(3\)?\pause
%\item Just another Greek letter?\pause
%\item A tasty lunch?
%\end{itemize}
%\end{frame}
\section{Crazy Infinite Products and Sums for \(\pi\)}
\begin{frame}
\frametitle{Madhva-Leibniz}
\[
\pi=4-\frac{4}{3}+\frac{4}{5}-\frac{4}{7}+\frac{4}{9}-\frac{4}{11}+\cdots
\]
\end{frame}
\begin{frame}
\frametitle{Vi\'ete's Formula}
\[
\frac{2}{\pi}=\frac{\sqrt{2}}{2}\cdot\frac{\sqrt{2+\sqrt{2}}}{2}\cdot\frac{\sqrt{2+\sqrt{2+\sqrt{2}}}}{2}\cdots
\]
\end{frame}
\begin{frame}
\frametitle{Wallis (1655)}
\[
\frac{\pi}{2}=\frac{2\cdot2\cdot4\cdot4\cdot6\cdot6\cdots}{3\cdot3\cdot5\cdot5\cdot7\cdot7\cdots}
\]
\end{frame}
\begin{frame}
  \frametitle{Nilakantha (15th century)}
  \[
    \pi=3+\frac{4}{2\cdot3\cdot4}-\frac{4}{4\cdot5\cdot6}+\frac{4}{6\cdot7\cdot8}-\frac{4}{8\cdot9\cdot10}+\cdots
  \]
\end{frame}
\begin{frame}
  \frametitle{Basel Problem (Euler)}
  \[\frac{\pi^2}{6}=\frac{1}{1^2}+\frac{1}{2^2}+\frac{1}{3^2}+\frac{1}{4^2}+\cdots\]
\end{frame}
\begin{frame}
  \frametitle{A variant of the Basel problem}
  \[\frac{\pi^4}{90}=\frac{1}{1^4}+\frac{1}{2^4}+\frac{1}{3^4}+\frac{1}{4^4}+\cdots\]
\end{frame}
\section{Craziest Infinite Sums for \(\pi\)}
\begin{frame}
  \frametitle{??}
  \[\frac{\pi}{2}=\arctan{\frac{1}{1}}+\arctan{\frac{1}{2}}+\arctan{\frac{1}{5}}+\arctan{\frac{1}{13}}+\cdots\]
\end{frame}
\begin{frame}
\frametitle{Ramanujan-Sato}
\[\frac{1}{\pi}=\frac{2\sqrt2}{99^2}\sum_{k=0}^{\infty}\frac{(4k!)}{k!^4}\cdot\frac{26390k+1103}{396^{4k}}\]
???
\end{frame}
\begin{frame}
\frametitle{Chudnovsky formula (the craziest one)}
\[
\frac{1}{\pi}=\frac{\sqrt{10005}}{4270934400}\sum_{k=0}^{\infty}\frac{(6k)!}{(3k)!k!^3}\cdot\frac{13591409+545140134k}{(-640320)^{3k}}
\]
\end{frame}
\section{Irrationally good approximations}
\begin{frame}
  \frametitle{Factorial}
  \[n!\sim\sqrt{2\pi n}\left(\frac{n}{e}\right)^n\]
\end{frame}
\begin{frame}
  \frametitle{Binomial Coefficients}
  \[\binom{2n}{n}\sim\frac{4^n}{\sqrt{\pi n}}\]
\end{frame}
\begin{frame}
  \frametitle{Your Problem (Warning: very hard)}
  If two random real numbers \(x\) and \(y\) are chosen
  on the interval \([0,1]\), what is the probability
  that the closest integer to \(x/y\) is even?
\end{frame}
\end{document}
