%\pagewidth 210mm
%\pageheight 297mm
\documentclass[a4paper,10pt]{article}

\usepackage[pdfusetitle]{hyperref}
\usepackage[a4paper]{geometry}
\usepackage[bf,tiny]{titlesec}
\usepackage{fancyhdr}
\usepackage{epigraph}
%\usepackage[indent=0pt,skip=10pt]{parskip}

\usepackage{amsmath}
\usepackage{amsthm}
\newtheorem{theorem}{Theorem}
\usepackage{amssymb}
\let\mathbbalt\mathbb

\usepackage{fontspec}
\usepackage{unicode-math}
\let\mathbb\mathbbalt
%\setmainfont{TeX Gyre Termes}
%\setmathfont{TeX Gyre Termes Math}
\newcommand{\naturals}{\mathbb{N}}
\newcommand{\reals}{\mathbb{R}}
\newcommand{\rationals}{\mathbb{Q}}
\newcommand{\integers}{\mathbb{Z}}

\newcounter{problemno}
\setcounter{problemno}{1}
\newcommand{\problem}{\section*{\textbf{Problem \theproblemno}}\refstepcounter{problemno}}
\newcounter{interludeno}
\setcounter{interludeno}{1}
\newcommand{\interlude}[1]{\par\noindent\\\fbox{\begin{minipage}{\linewidth}\textbf{Interlude \theinterludeno.\enspace}#1\end{minipage}}\\\refstepcounter{interludeno}}

\newcounter{questionno}
\setcounter{questionno}{1}
\newcommand{\question}[1]{\par\noindent{\thequestionno.\enspace}#1\refstepcounter{questionno}\\}

\newcommand{\thetitle}{Some of Euler's Gems: Solutions}

\renewcommand\thesection{\S{\arabic{section}}.}
\renewcommand\thesubsection{\S{\arabic{section}.\arabic{subsection}}}

\usepackage{tikz}
\usepackage{pgfplots}

\title{MEG 2023 Euler's Gems: Solutions}
\author{Nathan Wong}
\date{2023}
\begin{document}
\noindent Melbourne High School\\
Maths Extension Group 2023\\
\textbf{\thetitle}\\
\begin{enumerate}
\item \begin{enumerate}
\item Suppose that \(n\) is composite and write \(n=mp\) for some prime \(p\) and integer \(m\). From the formula
  for the sum of a geometric series, the fraction \[\frac{2^{pm}-1}{2^m-1}\] is the sum of the first \(p\) terms of a geometric series
  with first term \(1\) and common ratio \(2^m\). That is, \[\frac{2^{pm}-1}{2^m-1}=1+2^m+2^{2m}+\cdots+2^{(p-1)m}.\] Hence \(2^n-1\) is divisible
  by \(2^m-1\) and therefore can only be prime if \(m=1\) and \(n=p\).
\item
  \begin{figure}[h]
\begin{center}
    \begin{tabular}{|c|c|}
      \hline\(n\)&\(\sigma(n)\)\\
      \hline
       \(1\)&\(1\)\\
       \(2\)&\(3\)\\
      \(3\)&\(4\)\\
      \(4\)&\(7\)\\
      \(5\)&\(6\)\\
      \(6\)&\(12\)\\
      \(7\)&\(8\)\\
      \(8\)&\(15\)\\
      \(9\)&\(13\)\\
      \(10\)&\(18\)\\
      \(11\)&\(12\)\\
      \(12\)&\(28\)\\
      \(13\)&\(14\)\\
      \(14\)&\(24\)\\
      \(15\)&\(24\)\\
      \(16\)&\(31\)\\
      \(17\)&\(18\)\\
      \(18\)&\(39\)\\
      \(19\)&\(20\)\\
      \(20\)&\(42\)\\
      \hline
    \end{tabular}
  \end{center}
  \end{figure}
\item Since \(m\) and \(n\) are coprime, the divisors of \(mn\)
  are exactly \[a_1b_1,a_1b_2,\ldots,a_1b_{d(n)},a_2b_1,\ldots,a_2b_{d(n)},\ldots,a_{d(m)}b_1,\ldots,a_{d(m)}b_{d(n)}.\]
  Hence \[
    \begin{split}
      \sigma(mn)&=a_1(b_1+b_2+\cdots+b_{d(n)})+a_2(b_1+b_2+\cdots+b_{d(n)})+\cdots+a_{d(m)}(b_1+b_2+\cdots+b_{d(n)})\\
                &=(a_1+a_2+\cdots+a_{d(m)})(b_1+b_2+\cdots+b_{d(n)})\\
      &=\sigma(m)\sigma(n)
    \end{split}
  \]
\item The divisors of \(p_k\) are exactly \[1,p,p^2,\ldots,p^k.\]
  Therefore \[\sigma(p^k)=1+p+p^2+\cdots+p^k=\frac{p^{k+1}-1}{p-1}.\]
\item We know that \(\sigma(N)=\sigma(2^{p-1}(2^p-1))=\sigma(2^{p-1})\sigma(2^p-1)\), so we can do each separately. From the
  preceding exercise we have \[\sigma(2^{p-1})=\frac{2^p-1}{2-1}=2^p-1.\]
  Since \(2^p-1\) is prime, its only divisors are \(2^p-1\) and \(1\), so \[\sigma(2^p-1)=2^p-1+1=2^p.\]
  Therefore \[\sigma(N)=\sigma(2^{p-1}(2^p-1))=(2^p-1)2^p=2(2^{p-1}(2^p-1))=2N.\]
  Therefore \(2^{p-1}(2^p-1)\) is perfect.
\item If \(N\) is even, it divides some power of \(2\). Let \(2^k\) be the highest power of \(2\) that divides \(N\); this
  means \(k\ge 1\). Hence \(N/2^k\) must be odd; otherwise \(2^k\) would not be the highest power of \(2\) that divides \(N\).
  Therefore \(N=2^km\) for \(k\ge 1\) and odd \(m\).
\item We have \[\sigma(N)=\sigma(2^k)\sigma(m)=(2^{k+1}-1)\sigma(m).\]
  But \(\sigma(N)=2N\) because \(N\) is a perfect number, so we have
  \[ 2N=(2^{k+1}-1)\sigma(m).\]
  Clearly \(2N=2^{k+1}m\). Substituting this in yields \[2^{k+1}m=(2^{k+1}-1)\sigma(m)\] as required.
\item Substituting in \(\sigma(m)=2^{k+1}c\), we get \[2^{k+1}m=(2^{k+1}-1)2^{k+1}c.\]
  Cancelling \(2^{k+1}\) from both sides we have \[m=(2^{k+1}-1)c.\]
\item If \(c>1\), then \(m\) is at least divisible by \(1\), \(m\), and \(c\). (We note that \(m\not=c\) because \(k\ge 1\).)
  Then \[\sigma(m)\ge 1+m+c=1+(2^{k+1}-1)c+c=1+2^{k+1}c.\]
\item But \(\sigma(m)=2^{k+1}c\), so the preceding inequality implies \[2^{k+1}c\ge1+2^{k+1}c\] which implies \(0\ge1\). Clearly
  this is a contradiction, so \(c=1\).
\item If \(c=1\) then \(\sigma(m)=2^{k+1}=m+1\). The number \(m\) is at least divisible by itself and \(1\), so the sum of its
  divisors is at least \(m+1\), but if \(\sigma(m)\) actually equals \(m+1\) then no other number can divide it except for \(1\) and \(m\);
  hence \(m\) must be prime.
\item From \(m=2^{k+1}-1\) we have \[N=2^k(2^{k+1}-1).\] We know that \(2^{k+1}-1\) is prime, so \(k+1\) must also be prime. If \(k+1\)
  is prime, then \(k\) is one less than a prime, say \(p\). So if we put \(k=p-1\), then we get
  \[N=2^{p-1}(2^p-1).\] This completes the proof.
  
\end{enumerate}
\item
  \begin{enumerate}
  \item Starting from the \(1/3\) term, we note that \(1/3+1/4>1/4+1/4=1/2\), since \(1/3>1/4\). Similarly,
    \(1/5+1/6+1/7+1/8>1/8+1/8+1/8+1/8=1/2.\) This can be repeated to infinity to show that
    \[1+\frac{1}{2}+\frac{1}{3}+\frac{1}{4}+\cdots>1+\frac{1}{2}+\frac{1}{2}+\frac{1}{2}+\cdots.\]
    Clearly the RHS goes to infinity, and therefore the LHS must diverge too.
  \item In the first graph, where the rectangles are underneath the graph, the first rectangle is excluded
    because \(\ln(x)\) is the area under the graph of \(1/x\) from \(1\) to \(x\). So we get \(H_n-1<\ln(n)\).
    From the second graph, \(H_n>\ln(n)\). Putting these two inequalities together yields \[\ln(n)<H_n<\ln(n)+1.\]
  \item We have \[H_{kn}\approx\ln(kn)+\gamma_1\] and \[H_{n}\approx\ln(n)+\gamma_2.\]
    As \(n\) approaches infinity, \(\gamma_1\) and \(\gamma_2\) will approach the same value, so we have
    \[H_{kn}-H_{n}=\ln(kn)-\ln(n)=\ln(k).\]
    \begin{enumerate}
    \item From the preceding formula, \(\ln2=H_{2n}-H_{n}\) as \(n\) goes to infinity. At first, it's not obvious
      how to subtract these two values because the first \(n\) terms will disappear. The trick is the same as
      that of recognising that the even numbers and all the numbers are both still countable infinities even
      though the even numbers seems to be an infinity that is half as small, in a sense. Countable infinity
      means that there is a one-to-one correspondence with the positive integers, which the even numbers have.
      We associate \(1\) with \(2\), \(2\) with \(4\), \(3\) with \(6\) and so on. The same principle applies
      here. Instead of subtracting in order, we change the order of the series slightly (we're allowed to this
      because the series is convergent) and subtract a term of \(H_n\) from every second term of \(H_{2n}\); in this
      way an infinite number of terms can be subtracted without losing an infinite  number of terms at the beginning.
      Hopefully this is clear:
      \[
        \begin{split}
          \ln2&=1+\left(\frac{1}{2}-1\right)+\frac{1}{3}+\left(\frac{1}{4}-\frac{1}{2}\right)+\frac{1}{5}+\left(\frac{1}{6}-\frac{1}{3}\right)+\cdots\\
                &=1-\frac{1}{2}+\frac{1}{3}-\frac{1}{4}+\frac{1}{5}-\frac{1}{6}+\cdots
        \end{split}
      \]
    \item Apply the same method as for \(\ln2\).
    \item Apply the same method as for \(\ln2\).
    \end{enumerate}
    The formula is simply a compact form of the algorithm we use to subtract the two infinite series. From the preceding examples
    we note that the algorithm consists of doing the subtraction every \(n\)th term (if we're calculating \(\ln n\)). This subtraction
    is equivalent to multiplying the fraction by \(1-n\). The combination of the floor and ceiling functions evaluate to either \(0\) or \(1\)
    depending on whether \(k\) is a multiple of \(n\) and therefore determine which fractions to multiply by \(1-n\).
  \item Let \[\ln(1+x)=a_0+a_1x+a_2x^2+a_3x^3+\cdots.\]
    Putting in \(x=0\) we get \(a_0=0\). Differentiating both sides yields \[\frac{1}{(1+x)}=a_1+2a_2x+3a_3x^2+\cdots.\]
    Again putting in \(x=0\), we find that \(a_1=1\). We continue this process to find the other coefficients. Differentiating again
    we have \[\frac{-1}{(1+x)^2}=2a_2+6a_3x+\cdots.\] Putting in \(x=0\) results in \(-1=2a_2\); hence \(a_2=-1/2\). Eventually
    we find that \(a_3=1/3\), \(a_4=-1/4\), and so on, leaving us with \[\ln(1+x)=x-\frac{x^2}{2}+\frac{x^3}{3}-\frac{x^4}{4}+\cdots.\]
  \item Put \(x=-1/t\) and substitute it in to get
    \[\ln\left(\frac{t-1}{t}\right)=-\frac{1}{t}-\frac{1}{2t^2}-\frac{1}{3t^3}-\cdots.\]
      Flip the fraction in the logarithm on the LHS to get rid of all the minus signs:
      \[\ln\left(\frac{t}{t-1}\right)=\frac{1}{t}+\frac{1}{2t^2}+\frac{1}{3t^3}+\cdots.\]
      Since the series only makes sense for \(|x|<1\), the restriction on \(t\) is that \(|-1/t|<1\). From this \(t<-1\) or \(t>1\).
    \item \[ \sum_{k=2}^{n}\ln\left(\frac{k}{k-1}\right)=\sum_{k=2}^n\left(\frac{1}{k}+\frac{1}{2k^2}+\frac{1}{3k^3}+\cdots\right)\]
          By adding the logarithms on the LHS, we get a telescoping product (or sum if we break the logarithms into the difference of two logarithms)
          that leaves us with \(\ln(n)-\ln(1)=\ln(n)\). By expanding the sum on the RHS,  we get
          \[\ln(n)=\left(\frac{1}{2}+\frac{1}{2\cdot4}+\frac{1}{3\cdot8}+\cdots+\right)+\left(\frac{1}{3}+\frac{1}{2\cdot9}+\frac{1}{3\cdot27}+\cdots\right)+\cdots+\left(\frac{1}{n}+\frac{1}{2n^2}+\frac{1}{3n^3}+\cdots\right).\]
          If we take the first term of each bracket, we get the sum
          \[\frac{1}{2}+\frac{1}{3}+\cdots+\frac{1}{n}\] which is just
          \(H_n-1\). Similarly, if we take the second term of each bracket,
          we get \[\frac{1}{2\cdot 4}+\frac{1}{2\cdot9}+\frac{1}{2\cdot16}+\cdots+\frac{1}{2n^2}=\frac{1}{2}(H^{(2)}_n-1).\]
          The third term of each bracket gives us \[\frac{1}{3}(H^{(3)}_n-1).\]
          
          Continuing this rearrangement of the terms, we find that
          \[\ln(n)=H_n-1+\frac{1}{2}(H^{(2)}_n-1)+\frac{1}{3}(H^{(3)}_n-1)+\frac{1}{4}(H^{(4)}_n-1)+\cdots.\]
        \item The limit of the sum should a number around \(0.577216.\)
        \end{enumerate}
      \item
        \begin{enumerate}
        \item Each partition of \(n\) contributes a unit to the coefficient
          of \(x^n\). Each infinite sum represents the number of times a particular number appears in the partition---the first bracket represents the number
          of \(1\)s, the second the number of \(2\)s, the third the number
          of \(3\)s, and so on. A partition of \(n\) is really just selecting
          a term from each bracket such that the indices add to \(n\)
          when the terms are multiplied. Multiplying out the brackets
          results in an enumeration of all the possible ways of choosing
          different terms from each bracket, and therefore each coefficient
          will be all the ways to partition that number.
        \item The first \(6\) values of \(p(n)\) seem to yield the prime
          numbers. However, this would-be glorious theorem is destroyed
          when we find that \(p(7)=15\).
        \item Using the formula \[1+x+x^2+x^3+\cdots=\frac{1}{1-x}\] the result
          follows immediately.
        \item \(p_d(7)=5\).
        \item \(p_o(7)=5\).
        \item From each bracket now we can only choose \(x^i\) or \(1\); this is
          equivalent to either including \(i\) in the partition or not.
          Whereas before we could choose \(x^{2i}\), \(x^{3i}\), and so on to
          include more than one of \(i\) in the partition, only being
          able to choose \(x^i\) means only one of \(i\) ends up in the final
          partition, and therefore the coefficients of the expanded product
          are the number of partitions that use only unique numbers.

          This generating function is equivalent to finding how many subsets
          of a set \(\{1,2,3,\ldots,n\}\) sum to a particular number, since
          a partition with distinct elements is the same as choosing elements
          of that set that sum to that particular number.
        \item This generating function can be written as
          \[(1+x+x^2+x^3+\cdots)(1+x^3+x^6+x^9+\cdots)(1+x^5+x^{10}+x^{15}+\cdots)\cdots.\]
          This is the same as the original generating function except
          that the brackets determining which even numbers ended up
          in the partition are missing. Therefore by expanding the brackets
          only partitions that use odd numbers will be counted in the coefficient of each term.
        \item This problem shows the power of generating functions. If we
          can manipulate our generating functions so that they are equal,
          we have completed the proof.

          Euler's proof involves transforming each factor in \(F_d(x)\) into
          a fraction using the difference of two squares, as follows.
          \[
            \begin{split}
              F_d(x)&=\frac{(1+x)(1-x)}{1-x}\cdot\frac{(1+x^2)(1-x^2)}{1-x^2}\cdot\frac{(1+x^3)(1-x^3)}{1-x^3}\cdots\\
                    &=\frac{(1-x^2)(1-x^4)(1-x^6)\cdots}{(1-x)(1-x^2)(1-x^3)(1-x^4)(1-x^5)\cdots}\\
                    &=\frac{1}{(1-x)(1-x^3)(1-x^5)\cdots}\\
              &=F_o(x)
            \end{split}
            \]
        \end{enumerate}
\end{enumerate}
\end{document}
