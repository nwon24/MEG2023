\documentclass[a4paper,12pt]{article}

\usepackage[margin=1in]{geometry}
\usepackage[bf,tiny]{titlesec}
\usepackage{fancyhdr}

\usepackage[indent=0pt,skip=10pt]{parskip}

\usepackage{amsmath}
\usepackage{amsthm}
\newtheorem{theorem}{Theorem}
\usepackage{amssymb}
\let\mathbbalt\mathbb

%\usepackage{tikz}
\usepackage{tkz-euclide}

\usepackage{fontspec}
\usepackage{unicode-math}
\let\mathbb\mathbbalt

\newcommand{\naturals}{\mathbb{N}}
\newcommand{\reals}{\mathbb{R}}
\newcommand{\rationals}{\mathbb{Q}}
\newcommand{\integers}{\mathbb{Z}}

\newcounter{problemno}
\setcounter{problemno}{1}
\newcommand{\problem}{\section*{Problem \theproblemno}\refstepcounter{problemno}}
\newcounter{questionno}
\setcounter{questionno}{1}
\newcommand{\question}{\par\noindent\thequestionno.\enspace\refstepcounter{questionno}}

\newcommand{\thetitle}{First Meeting Problems Solutions}

\pagestyle{empty}
\begin{document}
Melbourne High School\\
Maths Extension Group 2023\\
\textbf{\thetitle}\\

\question
%\[
%  \begin{split}
%    \frac{4t}{3}+2&=2(t-2)\\
%    4t+6&=6(t-2)\\
%    4t+6&=6t-12\\
%    2t&=18\\
%    t&=9
%  \end{split}
%\]
Solve to get the solution \(t=9\).
\question
The square has side length \(2a\); hence its area
is \(4a^2\). The circle has area \(\pi a^2\), so the probability
a randomly chosen point in the square is also in the circle
is \(\pi a^2/4a^2=\pi/4\).
\question
Solving \(x+5=5x\) yields \(x=5/4\).
\question
Compute the first few partial sums to spot a pattern, viz.
\[
	\begin{split}
	\frac{1}{1\cdot2}+\frac{1}{2\cdot3}=\frac{3+1}{6}=\frac{2}{3}\\
	\frac{1}{1\cdot2}+\frac{1}{2\cdot3}+\frac{1}{3\cdot4}=\frac{2}{3}+\frac{1}{12}=\frac{3}{4}\\
	\frac{1}{1\cdot2}+\frac{1}{2\cdot3}+\cdots+\frac{1}{4\cdot5}=\frac{3}{4}+\frac{1}{20}=\frac{4}{5}\\
	\end{split}
\]
If this pattern continues, the desired result is \(99/100\).
(Extension: for those who have learnt proof, try proving that
\[
	\frac{1}{1\cdot2}+\frac{1}{2\cdot3}+\frac{1}{3\cdot4}+\cdots+\frac{1}{n\cdot(n+1)}=\frac{n}{n+1}
\]
by induction.)
\question
The number must be divisible by both \(5\) and \(3\), and so the
last digit must be a \(0\). To be divisible by \(3\), the sum of
the digits must be divisible by \(3\); we need three \(2\)s. Hence
the smallest number that is gobbus is \(2220\).
\question
All triples that have a product of \(36\) are: \((1,1,36)\), \((2,1,18)\),
\((2,2,9)\), \((3,3,4)\), \((4,9,1)\), \((6,6,1)\). If knowing the sum
of the three numbers is not enough information, then it must be either
\((2,2,9)\) or \((6,6,1)\), as both sum to \(13\). Since there is a jar
with the most cookies in it, the answer is \((2,2,9)\).
\question
First rearrange to find that \(a+b=1/4\). Square both sides and we get
\(a^2+b^2+2ab=1/16\). Substitute in \(a^2+b^2=16\) and the equation becomes
\(2ab=-255/16\); hence \(ab=-255/32\). Note that \(1/a+1/b=(a+b)/ab\), so
we have
\[
	\begin{split}
		\frac{1}{a}+\frac{1}{b}&=\frac{1/4}{-255/32}\\
		&=\frac{-8}{255}.
	\end{split}
\]
\question
\(2,5,13,17,29,\ldots\), i.e., all the primes in the form \(4k+1\) (except for \(2\)). 
\question
Rearrange and factorise to get \((5x-y)(3x-2y)=100\).
Now we just need to enumerate the factors of \(100\) and solve
for \(x\) and \(y\).

It is wise to consider that since \(x\) and \(y\) are positive
integers, \(5x-y>3x-2y\), so we only need to consider half the pairs
of factors of \(100\). Beginning with \(5x-y=100\) and \(3x-2y=1\),
we see that this means \(7x=199\), which has no solution in positive integers.
Then we move on to \(5x-y=50\) and \(3x-2y=2\). Solving these two
simultaneous equations yields \(7x=98\), or \(x=14\), which implies \(y=20\).

Continuing in this fashion, we find one other solution, when \(5x-y=20\)
and \(3x-2y=5\). In this case \(x=5\) and \(y=5\). So all the solutions
in positive integers to our original equation are \((5,5)\) and \((14,20)\).
\question
Reduce modulo \(10\) to get the final digit of \(2^3=8\).

If not familiar with modular arithmetic, consider the last digit of the
first few powers of \(2\): \(2,4,8,16,32,64,128,256,512,\ldots\). Note
how the the last digits go \(2,4,8,6,2,4,8,6,2,\ldots\). That is, the last
digits repeat every fourth term in the sequence. Since \(2023\) is \(3\)
more than a multiple of \(4\), its last digit is the third number in the
repeated sequence \(2,4,8,6\). Hence the last digit of \(2^{2023}\) is \(8\).
\question
Without loss of generality, assume \(a\ge b\ge c\). This will be needed for
an application of the rearrangement inequality.

Square both sides of the equation \(a+b+c=1\) to get
\[
	a^2+b^2+c^2+2(ab+bc+ac)=1.
\]
Substituting this into \(a^2+b^2+c^2+1\) yields
\[
	\begin{split}
		a^2+b^2+c^2+1&=a^2+b^2+c^2+(a^2+b^2+c^2+2(ab+bc+ac))\\
		&=2(a^2+b^2+c^2)+2(ab+bc+ac).
	\end{split}
\]
By the rearrangement inequality, \(a^2+b^2+c^2\ge ab+bc+ac\), so we have
\[
	\begin{split}
		2(a^2+b^2+c^2)+2(ab+bc+ac)&\ge 2(ab+bc+ac)+2(ab+bc+ac)\\
		&=4(ab+bc+ac).
	\end{split}
\]
Hence \(a^2+b^2+c^2+1\ge4(ab+bc+ac)\).
\question
Consider a set \(S_n=\{1,2,3,\ldots,n\}\). Let \(a_1,a_2,a_3,\ldots,a_{2^n}\)
be the alternating sums of its \(2^n\) subsets, and let \(A_n\) equal the sum
of all alternating sums of all the subsets of \(S_n\). By these definitions we
have
\[
	\begin{split}
		S_n&=\sum_{i=0}^{n}a_i\\
		&=a_1+a_2+a_3+\cdots+a_{2^n}.
	\end{split}
\]
Now consider the set \(S_{n+1}=\{1,2,3,\ldots,n,n+1\}\). Note that \(2^n\)
of its subsets are the same as those of \(S_n\); the other \(2^n\) subsets
can be formed by taking each subset of \(S_n\) and adding \(n+1\) to it.
Now let \(b_1, b_2,b_3,\ldots,b_{2^n}\) be the alternating sums of the \(2^n\)
subsets that are not subsets of \(S_n\). In other words, these are the alternating
sums of all the subsets of \(S_n\) with \(n+1\) added to them. Since \(n+1\)
is the biggest element in each of these sets, each alternating sum begins with
\(n+1\), followed by the rest of the numbers in the set.

We see that \(b_i=(n+1)-a_i\) for \(1\le i\le 2^n\). To see why this is true
consider the set \(\{1,2,3,4\}\); its alternating sum is \(4-3+2-1=2\). If we
add \(5\) to the set, the alternating sum becomes \(5-4+3-2+1=5-(4-3+2-1)\).

So we have
\[
	\begin{split}
		b_1&=(n+1)-a_1\\
		b_2&=(n+1)-a_2\\
		b_3&=(n+1)-a_3\\\vdots\\
		b_{2^n}&=(n+1)-a_{2^n}.
	\end{split}
\]
Adding these together we get
\[
	\begin{split}
		b_1+b_2+b_3+\cdots+b_{2^n}&=(n+1)\cdot2^n-(a_1+a_2+a_3+\cdots+ a_{2^n})\\
		&=(n+1)\cdot2^n-S_n
	\end{split}
\]
The sum of the alternating sums of all subsets of \(S_{n+1}\) is the sum of the alternating
sums of all subsets of \(S_n\) plus all of \(b_1,b_2,b_3,\cdots,b_{2^n}\). Hence
\[
	\begin{split}
		A_{n+1}&=S_n+b_1+b_2+b_3+\cdots+b_{2^n}\\
		&=S_n+(n+1)\cdot2^n-S_n\\
		&=(n+1)\cdot2^n.
	\end{split}
\]
Thus we have found a closed-form expression for \(A_{n+1}\). We can just plug in \(n=9\)
to get \(A_{10}=10\cdot2^9.\)
\question
This can be done with just a scientific calculator.
First we calculate the \(5\) raised to powers of \(2\) modulo \(397\).
This is doable because we get the next power by squaring the previous residue
and reducing modulo \(397\). Since our residues are guaranteed to be
less than \(397\), our calculators do not blow up, as they would do
if we tried computing \(5^{261}\).
\[
	\begin{split}
		5^1&\equiv5\pmod{397}\\
		5^2&\equiv25\pmod{397}\\
		5^4&\equiv228\pmod{397}\\
		5^8&\equiv374\pmod{397}\\
		5^{16}&\equiv132\pmod{397}\\
		5^{32}&\equiv353\pmod{397}\\
		5^{64}&\equiv348\pmod{397}\\
		5^{128}&\equiv19\pmod{397}\\
		5^{256}&\equiv361\pmod{397}
	\end{split}
\]
Since \(261=256+4+1\), we get
\[
	\begin{split}
		5^{261}&\equiv5^{256}\cdot5^{4}\cdot5^{1}\\
		&\equiv361(228)(5)\pmod{397}\\
		&\equiv248\pmod{397}.
	\end{split}
\]
Here's an example of how to evaluate one of those powers of \(5\)
modulo \(397\) on a simple calculator. Let's say we wanted to
compute \(5^{16}\pmod{397}\). We see that \(5^8\equiv374\pmod{397}\),
so we evaluate \(374^2/397\) on our calculator. We get something like
\(352.332\ldots\). Round down and take away that multiple of \(397\)
from \(374^2\) to get \(5^{16}\pmod{397}\). In other words, now we
evaluate \(374^2-352\cdot397\) to get \(132\); this number is \(5^{16}\pmod{397}\).
Think about this a few times to understand why it works.
\question
\[
	\begin{tikzpicture}[scale=1.2]
	\tkzDefPoint(0,4){A}
	\tkzDefPoint(3,0){C}
	\tkzDefPoint(-3,0){B}
	\tkzDrawPolygon(A,B,C)
	\tkzLabelPoints[above](A)
	\tkzLabelPoints[left](B)
	\tkzLabelPoints[right](C)
	\tkzDrawBisector(A,C,B)
	\tkzGetPoint{Q}
	\tkzLabelPoints[above left](Q)
	\tkzDrawBisector(A,B,C)
	\tkzGetPoint{P}
	\tkzLabelPoints[above right](P)
	\tkzMarkAngle[size=0.5cm,arc=l](A,C,B)
	\tkzMarkAngle[size=0.5cm,arc=ll](C,B,A)
	\tkzDefPointBy[projection=onto B--P](A)
	\tkzGetPoint{H}
	\tkzDrawSegment(A,H)
	\tkzLabelPoints[below](H)
	\tkzDefPointBy[projection=onto Q--C](A)
	\tkzGetPoint{K}
	\tkzDrawSegment(A,K)
	\tkzLabelPoints[below](K)
	\tkzMarkRightAngle(A,H,B)
	\tkzMarkRightAngle(A,K,C)
	\tkzInterLL(A,K)(B,C)\tkzGetPoint{F}
	\tkzInterLL(A,H)(B,C)\tkzGetPoint{E}
		\tkzDrawSegment(K,F)\tkzDrawSegment(H,E)
		\tkzLabelPoints[below](F,E)
		\tkzDrawSegment(K,H)
\end{tikzpicture}
\]
Since \(\angle{BHE}\) is also a right angle, \(\angle{HBE}=\angle{HBA}\) by definition,
and \(HB\) is a common side, \(\triangle{AHB}\) is congruent to \(\triangle{BHE}\)
by ASA. This means \(AH=HE\), as they are corresponding sides of the two triangles.
Similarly, \(\triangle{AKC}\) is congruent to \(\triangle{KFC}\), again by ASA. This
implies \(AK=KF\).

Since \(AK=KF\) and \(AH=HE\), \(\triangle{AFE}\) is similar to \(\triangle{AKH}\)
by SAS; from this we conclude that \(\angle{AKH}=\angle{AFE}\) and
\(\angle{AHK}=\angle{AEF}\). Thus \(HK\) is parallel to \(FE\), and because \(FE\)
lies on \(BC\), \(HK\) is also parallel to \(BC\), as required.
\question
Firstly, since \(F_0=0\) we can get rid of the the \(F_0\) term.
Our power series becomes
\[
	G(x)=F_1x+F_2x^2+F_3x^3+\cdots.
\]
Replace each coefficient, except for the first,  with its recursive definition:
\[
	G(x)=F_1x+(F_0+F_1)x^2+(F_1+F_2)x^3+(F_2+F_3)x^4+\cdots.
\]
Now rearrange and group the first terms of each set of parentheses together.
Do likewise with the second term of each set of parentheses.
\[
	G(x)=F_1x+(F_0x^2+F_1x^3+F_2x^4+\cdots)+(F_1x^2+F_2x^3+F_3x^4+\cdots).
\]
Notice that by factoring out each set of parentheses we get \(G(x)\) again.
\[
	\begin{split}
		G(x)&=F_1x+(F_0+F_1x+F_2x^2+F_3x^3+\cdots)x^2+(F_1x+F_2x^2+F_3x^3+\cdots)x\\
		&=x+x^2G(x)+xG(x)
	\end{split}
\]
Rearranging yields
\[
	G(x)=\frac{x}{1-x-x^2}.
\]
\end{document}
