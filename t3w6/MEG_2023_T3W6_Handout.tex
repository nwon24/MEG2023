
\documentclass[a4paper]{article}

\newcommand{\theterm}{3}
\newcommand{\theweek}{6}
\newcommand{\thepdftitle}{MEG 2023 Term \theterm\ Week \theweek\ Handout}
\newcommand{\thedisplaytitle}{Term \theterm\ Week \theweek\ Handout}

\title{{\thepdftitle}}
\author{Nathan Wong\and Tom Yan}
\date{2023}

\newcommand{\leg}[2]{\left(\frac{#1}{#2}\right)}
\newcommand{\ileg}[2]{(#1|#2)}

%\newcommand{\marginfn}[1]{\marginpar{\footnotemark}\footnotetext{#1}}
\newcommand{\marginnote}[1]{\marginpar{\footnotesize{#1}}}
\newcommand{\marginfnote}[1]{\footnotemark\marginpar{\footnotemark[\value{footnote}]\footnotesize{#1}}}
\usepackage{geometry}
%\geometry{a4paper,left=24.8mm,top=27.4mm,headsep=2\baselineskip,textwidth=107mm,marginparsep=8.2mm,marginparwidth=49.4mm,textheight=49\baselineskip,headheight=\baselineskip}
\geometry{a4paper,left=1in,top=1in,bottom=1in,headsep=2\baselineskip,textwidth=107mm,marginparsep=8.2mm,marginparwidth=49.4mm,textheight=49\baselineskip,headheight=\baselineskip}
\usepackage[bf,tiny]{titlesec}
%\usepackage{fancyhdr}
\usepackage{epigraph}
%\usepackage[indent=0pt,skip=10pt]{parskip}

\usepackage{amsmath}
\usepackage{amsthm}
\newtheorem{theorem}{Theorem}
\usepackage{amssymb}
\let\mathbbalt\mathbb

\usepackage{fontspec}
\usepackage{unicode-math}
\let\mathbb\mathbbalt

\newcommand{\naturals}{\mathbb{N}}
\newcommand{\reals}{\mathbb{R}}
\newcommand{\rationals}{\mathbb{Q}}
\newcommand{\integers}{\mathbb{Z}}

\usepackage[pdfusetitle]{hyperref}

\newcommand{\myquote}[2]{%
  \begin{quote}
    \emph{#1}
    \begin{flushright}---{#2}
    \end{flushright}
  \end{quote}}
\pagestyle{empty}
\begin{document}
\noindent Melbourne High School\\\
\noindent Maths Extension Group 2023\\\
\noindent \textbf{\thedisplaytitle}\\\
\myquote{%
	\textrm{\texttt{out[8]=}}
\[\begin{split}
	\biggl\{&\left\{x\to -\frac{2i\pi c_1+\ln2-\ln3}{\ln2+\ln3} \text{ if } c_1\in\integers\right\},\\
	&\left\{x\to -1-\frac{2i\pi c_1}{-\ln2+\ln3} \text { if } c_1\in\integers\right\}\biggl\}
	\end{split}
\]}{Mathematica (2023)}
\section*{Problems in solving equations}
\begin{enumerate}
\item Find all $x$ such that $-4 < \frac{1}{x} < 3$.  
\item Solve the following system of equations $$xy = 12\sqrt{6}$$ $$yz=54\sqrt{2}$$ $$zx=48\sqrt{3}. $$
\item Find all real numbers $x$ for which $$\frac{8^x+27^x}{12^x+18^x}=\frac{7}{6}. $$  
\item Mr \marginnote{USSR 1990} Fat is going to pick three non-zero real numbers and Mr Taf is going to arrange the three numbers as coefficients of a quadratic equation $$\underline{\hspace{0.7cm}}x^2 + \underline{\hspace{0.7cm}}x + \underline{\hspace{0.7cm}} = 0$$ Mr Fat wins the game if and only if the resulting equation has two distinct rational solutions. Who has the winning strategy?
\item Find \marginnote{ARML 1997} a triple of rational numbers $(a,b,c)$ such that $$\sqrt[3]{\sqrt[3]{2}-1} = \sqrt[3]{a} + \sqrt[3]{b} + \sqrt[3]{c}.$$
\end{enumerate}
\pagebreak
\myquote{! Missing delimiter (. inserted).}{\texttt{lualatex} (2023)}
\section*{Gauss's Lemma}
We have seen that Euler's Criterion doesn't provide 
a good way to evaluate \(\ileg{a}{p}\) except when
\(a=-1\). 
When \(a\not=-1\), how do we attack the problem?

The answer is that Euler's Criterion can be used
to prove something called \emph{Gauss's Lemma}, which,
in theory although not in practice, would allow
us to evaluate \(\ileg{a}{p}\) for any \(a\).
In other words, using Gauss's Lemma we can find all
the primes for which \(2,3,5,7,11,\ldots\) are quadratic
residues. 

The lemma and its proof are hard to motivate despite
their simplicity.
In fact, the proof is very similar to Ivory's proof
of Fermat's Little Theorem.
But how anyone could have come up with the lemma in 
the first place is somewhat mystifying.\marginnote{The fact that Gauss
  was one of the greatest mathematicians of all time might have had something
  to do with it.}

The lemma is as follows. Given a number \(a\) and
an odd prime \(p\), let \(P=(p-1)/2\) and form the set
of numbers
\[A = a, 2a, \ldots, Pa.\]
The set of integers in the range \((-p/2,p/2)\) form a 
complete set of residues, so every number in the above
list can be reduced to a number in that range.
The rule for this reduction is simple; if \(ja\) reduced
modulo \(p\) is greater than \(p/2\), then reduce it
further into the range \((-p/2,0)\); otherwise, leave it 
because it's already in the range \((0,p/2)\).
Let \(B\) be the set of numbers in \(A\) that are reduced
in this manner. 
The numbers in \(B\) are all in the range \((-p/2,p/2)\).

In the set \(B\) clearly there are some numbers that are positive
and some that are negative.
If we count the number of negative numbers in the set and call
that number \(v\), then \[\leg{a}{p}=(-1)^v.\]
This is Gauss's Lemma.

Let's make things a little clearer with a numerical example.\marginnote{Numerical
  examples, although no substitute for proper proofs, are key in number
  theory because they allow us to formulate conjectures or verify the truth
  of a general case that would otherwise be impossible to guess. It is said
  that based on numerical evidence Gauss guessed the Prime Number Theorem, although
  the proof was not discovered until some years after his death.}
Suppose \(a=3\) and \(p=7\). We would like to know if \(3\)
is a quadratic residue modulo \(7\).
Following the above procedure, we set \(P=(7-1)/2=3\) and form
the numbers \(a,2a,\ldots,Pa\). Thus we have 
\[A=\{3, 6, 9\}.\]
Reducing this set into the range \((-p/2,p/2)=(-7/2,7/2)\)
to get \(B\), we have \[B=\{3, -1, 2\}.\]
The number of negative numbers is \(1\), so \[\leg{3}{7}=(-1)^1=-1\]
and thus \(3\) is not a quadratic residue modulo \(7\).\marginnote{Check
this by enumerating the quadratic residues modulo \(7\) as shown in
previous handouts.}

Here is another example before we begin the proof.
This time let \(p=11\) and \(a=5\). We would like to know whether
\(5\) is a quadratic residue modulo \(11\).

As before, set \(P=(11-1)/2=5\). Then form the set 
\[ A=\{5, 10, 15, 20, 25\}.\]
Reduce each element of that set into the range \((-11/2, 11/2)\)
and we get
\[ B=\{5, -1, 4, -2, 3\}.\]
The number of negative numbers is \(B\) is \(2\), and therefore
\[\leg{5}{11} = (-1)^2=1.\]
Verify this by noting that \(4^2\equiv5\pmod{11}.\)

Now we begin the proof. Take the sets \(A\) and \(B\) to be
the formed as described above.
Two things are immediately clear to us about the numbers in \(B\).
No two numbers of the set are equal; 
if \(ia\equiv ja\pmod{p}\) then \(i\equiv j\) because
\(a\not\equiv0\), and since \(i,j<P\) we have \(i=j\). 
On the other hand, it is also not possible for \(ia\equiv -ja\pmod{p}\)
(that is, if we take the absolute value of all the numbers in \(B\), we
won't get two duplicates); for if that is the case then \(ia+ja\equiv0\pmod{p}\)
and since \(a\not\equiv0\) we must have \(i+j\equiv0\).
This is clearly impossible because \(i,j\le P\) and so \(i+j\le 2P=p-1\).

Therefore the numbers \(1,2,\ldots,P\) appear exactly once in \(B\) with
either a positive or negative sign, but not both, and each number in \(B\)
is congruent to one and only one number in \(A\). Thus the products of
the two sets are congruent. If we let \(v\) be the number of negative signs in \(B\), we have
\[ a\times 2a\times\cdots\times Pa\equiv (-1)^v1\times2\times\cdots\times P\pmod{p}.\]
Condensing the products yields \[a^PP!\equiv (-1)^vP!\pmod{p}.\]
Clearly \(P!\not\equiv0\pmod{p}\), so \[a^P\equiv(-1)^v\pmod{p}.\]

This is where Euler's Criterion comes into play. Recall that it states 
\[a^P\equiv\leg{a}{p}\pmod{p}.\]
Consequently we have \[\leg{a}{p}\equiv(-1)^v\pmod{p}\] which is the result
we're trying to prove; the quadratic character of \(a\) is dependent only
on the number of negative numbers in the set \(B\), which is denoted here by \(v\).

As stated previously, Gauss's Lemma allows us to determine \(\ileg{a}{p}\) 
for any \(a\not\equiv0\), but it can do much more than that.
In fact we can go the other way and ask for which primes \(p\) 
is \(\ileg{a}{p}=1\) for given values of \(a\). For example, for which
primes \(p\) is \(\ileg{2}{p}=1\)? That is, for which primes is \(2\)
a quadratic residue? Yet another way of phrasing the question is to ask
for which primes \(p\) the equation 
\[x^2-2\equiv0\pmod{p}\] is solvable.\marginnote{Note how the theory
  of quadratic residues is actually connected with \emph{solving quadratic
    congruences}. Linear congruences were simple enough, but it seems
quadratic ones are a different thing altogether.}

The most natural way to proceed, perhaps the only way, is to blindly
follow the steps of Gauss's Lemma and see how far the general case
can lead us.
So let \(P=(p-1)/2\) and form the set
\[ A=\{2,4,\ldots,2P\}.\]
Note that \(2P=p-1\). This means that every member of \(A\)
is in the range \((0, p-1)\).
Now the question is: if we reduce each member of \(A\) into the range
\((-p/2,p/2)\) to make \(B\), how many numbers in \(B\)
are negative?

Since every member of \(A\) is already less than \(p\), the number of
negative numbers in \(B\) is the same as the number of numbers in \(A\)
that are greater than \(p/2\). Let \(u\) be the number in the range
\((1,P)\) satisfying \[2u<p/2\] and \[2(u+1)>p/2.\] Thus \(u\) is the
number of numbers in \(A\) that are \emph{less than} \(p/2\); therefore
the number of numbers in \(A\) that are greater than \(p/2\) is \[v=P-u=(p-1)/2-u.\]
Now we just need to determine the parity of \(v\) based on some condition
regarding \(p\), because we know that
\[\leg{2}{p}=(-1)^v.\]
But this information seems to depend on both \(p\) and \(u\),
so we cannot just plug in \(p=4k+1\) or \(p=4k+3\) like we did for the
case \(a=-1\). What do we do instead? \marginnote{There's nothing quite like a good cliffhanger.}

\section*{Problems}
\begin{enumerate}
\item Use Gauss's Lemma to find \(\ileg{5}{19}\).
\item\label{prob:2res} Determine whether \(2\) is a quadratic residue for as many primes
  as you can, starting \(3,5,7,11,\ldots\). Do you see any patterns that
  might give us a clue as to how to finish the above proof of the quadratic
  character of \(2\)?
\item (Continuation of the above proof of the quadratic character of \(2\))
  \begin{enumerate}
  \item If \[a<u<b\] find \(a\) and \(b\) in terms of \(p\).
  \item If we are optimistic, perhaps \(\ileg{2}{p}\) has something
    to do with whether \(p\equiv1\) or \(p\equiv-1\) modulo \(4\), as for
    the case \(a=-1\). Let \(p=4k+1\) and \(p=4k+3\) and show in both
    cases that \(u=k\).
  \item Is the preceding information enough to pin down the parity of \(v\)
    and hence solve the problem?
  \item If not, try other congruence classes for \(p\). If you get the right
    one, you will have solved the problem of evaluating \(\ileg{2}{p}\).\marginnote{This problem
      is much easier if you did question \ref{prob:2res}.}
  \end{enumerate}
\end{enumerate}
\end{document}
