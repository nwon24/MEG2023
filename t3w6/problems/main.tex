\documentclass{article}
\usepackage{graphicx} % Required for inserting images

\title{Solving Equations MEG}
\author{Tom Yan}
\date{August 2023}

\begin{document}

\maketitle

\section{Introduction}
1. Find all $x$ such that $-4 < \frac{1}{x} < 3$.  \\\\
2. Solve the following system of equations $$xy = 12\sqrt{6}$$ $$yz=54\sqrt{2}$$ $$zx=48\sqrt{3}. $$\\\\
3. Find all real numbers $x$ for which $$\frac{8^x+27^x}{12^x+18^x}=\frac{7}{6}. $$  \\\\
4. (USSR 1990) Mr Fat is going to pick three non-zero real numbers and Mr Taf is going to arrange the three numbers as coefficients of a quadratic equation $$\underline{\hspace{0.7cm}}x^2 + \underline{\hspace{0.7cm}}x + \underline{\hspace{0.7cm}} = 0$$ Mr Fat wins the game if and only if the resulting equation has two distinct rational solutions. Who has the winning strategy?\\\\
5. (ARML 1997) Find a triple of rational numbers $(a,b,c)$ such that $$\sqrt[3]{\sqrt[3]{2}-1} = \sqrt[3]{a} + \sqrt[3]{b} + \sqrt[3]{c}.$$
\newpage
1. If $x$ is positive we have $1<3x$ so $x>\frac{1}{3}$, if $x$ is negative $\frac{1}{3}<3$ will always be true. Meanwhile, we have $1<-4x$ so $x<-\frac{1}{4}$. Therefore, all $x$ such that $x>\frac{1}{3}$ or $x<-\frac{1}{4}$ works. \\\\
2. Solving for $x$ and substituting it into the third equation we get $z\frac{12\sqrt{6}}{y}= 45\sqrt{3}$, we multiply this by the second equation to get $z^2=4\times 54$ so $z= \pm6\sqrt{6}$. From there, its pretty easy to sub this value back into the equations to get two sets of solutions for $(x,y,z)$, $(\pm 4\sqrt{2}, \pm 3\sqrt{3}, \pm 6\sqrt{6})$. \\\\
3. We let $a=2^x$ and $b=3^x$. The equation becomes $\frac{a^3+b^3}{a^2b+b^2a}=\frac{7}{6}$. Which after some simplifying we get $6a^2-13ab+6b^2=0$, which is $(2a-3b)(3a-2b)=0$. Therefore $2^{x+1}=3^{x+1}$ or $2^{x-1}=3^{x-1}$, which implies $x=-1$ and $x=1$.  \\\\
4. Mr Fat has the winning strategy, because by choosing a set of distinct rational nonzero numbers $a,b,c$, such that $a+b+c=0$ will make him win. Let $a', b', c'$ be a random permutation of $a,b,c$ and let $f(x)=a'x^2+b'x+c'$. Then $f(1)=a'+b'+c'=a+b+c=0$, and so $1$ is a solution. Since the product of two numbers is $\frac{c'}{a'}$ by Vieta's, the other solution is clearly $\frac{c'}{a'}$, which is different from $1$. Thus Mr Fat can guarantee two distinct solutions.     \\\\
5. Official Solution: Let $x=\sqrt[3]{\sqrt[3]{2}-1}$ and $y = \sqrt[3]{2}$. So $y^3=2$ and $x=\sqrt[3]{y-1}$. Note that $$1=y^3-1=(y-1)(y^2+y+1)$$ and $$y^2+y+1=\frac{3y^2+3y+3}{3}=\frac{(y+1)^2}{3}$$ which implies that $$x^3=y-1=\frac{1}{y^2+y+1}=\frac{3}{(y+1)^3}$$ or $$x=\frac{\sqrt[3]{3}}{y+1}.$$ On the other hand $3=y^3+1=(y+1)(y^2-1+1)$ from which it follows that $$\frac{1}{y+1}=\frac{y^2-y+1}{3}.$$ Thus we have $$x=\sqrt[3]{\frac{1}{9}}(\sqrt[3]{4}-\sqrt[3]{2}+1).$$ Consequently $(a,b,c) = (\frac{4}{9}, -\frac{2}{9}, \frac{1}{9})$ is a desired triple. \\\\
\end{document}
