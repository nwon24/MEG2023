\documentclass[a4paper]{article}

\newcommand{\theterm}{3}
\newcommand{\theweek}{}
\newcommand{\thepdftitle}{MEG 2023 Term \theterm\ Holiday Handout Solutions}
\newcommand{\thedisplaytitle}{Term \theterm\ Holiday Handout Solutions}

\title{{\thepdftitle}}
\author{Nathan Wong\and Tom Yan}
\date{2023}

\newcommand{\leg}[2]{\left(\frac{#1}{#2}\right)}
\newcommand{\ileg}[2]{(#1|#2)}

%\newcommand{\marginfn}[1]{\marginpar{\footnotemark}\footnotetext{#1}}
\newcommand{\marginnote}[1]{\marginpar{\footnotesize{#1}}}
\newcommand{\marginfnote}[1]{\footnotemark\marginpar{\footnotemark[\value{footnote}]\footnotesize{#1}}}
\usepackage{geometry}
%\geometry{a4paper,left=24.8mm,top=27.4mm,headsep=2\baselineskip,textwidth=107mm,marginparsep=8.2mm,marginparwidth=49.4mm,textheight=49\baselineskip,headheight=\baselineskip}
\geometry{a4paper,left=1in,top=1in,bottom=1in,headsep=2\baselineskip,textwidth=107mm,marginparsep=8.2mm,marginparwidth=49.4mm,textheight=49\baselineskip,headheight=\baselineskip}
\usepackage[bf,tiny]{titlesec}
%\usepackage{fancyhdr}
\usepackage{epigraph}
%\usepackage[indent=0pt,skip=10pt]{parskip}

\usepackage{amsmath}
\usepackage{amsthm}
\newtheorem{theorem}{Theorem}
\usepackage{amssymb}
\let\mathbbalt\mathbb

\usepackage{fontspec}
\usepackage{unicode-math}
\let\mathbb\mathbbalt

\newcommand{\naturals}{\mathbb{N}}
\newcommand{\reals}{\mathbb{R}}
\newcommand{\rationals}{\mathbb{Q}}
\newcommand{\integers}{\mathbb{Z}}

\usepackage[pdfusetitle]{hyperref}

\usepackage{graphicx}

\newcommand{\myquote}[2]{%
  \begin{quote}
    \emph{#1}
    \begin{flushright}---{#2}
    \end{flushright}
  \end{quote}}
\pagestyle{empty}
\begin{document}
\noindent Melbourne High School\\\
\noindent Maths Extension Group 2023\\\
\noindent \textbf{\thedisplaytitle}\\\
\section*{Problems}
\begin{enumerate}
	\item We have $x=b^2+b+1$ and $x=2(b-2)^2+(b-2)+b=2b^2-7b+8$. Hence $(2b^2-7b+8)-(b^2+b+1)=b^2-8b+7=(b-7)(b-1)=0$. Clearly $b \neq 1$, so $b=7$ and thus $x=49+7+1=57.$
	\item Considering mod 3, WLOG let $x_1 = 3$, and thus we are left with $x_2x_3x_4 + x_3x_4x_5 = x_3x_4(x_2 + x_5).$ Clearly $3 \nmid x_3x_4$ so $3 \vert x_2+x_5$, so $(x_3,x_4)$ can be $(1,2), (1,5), (4,2), (4,5)$ and $(2,1), (5,1), (2,4), (5,4)$ thus giving 8 ways, and $x_3x_4$ can be permuted in $2$ ways after designating $x_3$ and $x_4$. So there are $5 \times 8 \times 2 = 80$ ways.
	\item Let $\angle PYZ = \theta$. Since $\angle PZA + \angle PYA = 180^{\circ}$, $PZAY$ is cyclic, so $\angle PYZ = \angle PAZ = \theta$ (subtended by same arc). Thus $\angle PAB = 180^{\circ} - \theta$, and so $PCB = \theta$. Since $\angle PYX = \angle PXC = 90^{\circ}$, $PYXC$ is cyclic. Thus $PYX = 180^{\circ}-\theta$. Since $\angle PYZ + \angle PYX = 180^{\circ}$, we are done. \\ \centerline{\includegraphics[width=3.4in, height=3.1in]{Simson line.png}}
	\item Let $t$ be real number that is not $\pm 1$, and $t=\frac{x-3}{x+1}$ thus $x=\frac{3+t}{1-t}$. Rewriting the given equation in terms of $t$ we have $$f(t)+f(\frac{t-3}{t+1})=\frac{3+t}{1-t}.$$ Similarly, let $t=\frac{3+x}{1-x}$, so then we have $x=\frac{t-3}{t+1}$ and $\frac{x-3}{x+1}=\frac{3+t}{1-t}$. Rewriting the given equation in terms of $t$ again, we have $$f(\frac{3+t}{1-t}) + f(t)=\frac{t-3}{t+1}.$$ Adding the two rewritten equations we get $$2f(t)+f(\frac{t-3}{t+1})+f(\frac{3+t}{1-t})=\frac{3+t}{1-t}+\frac{t-3}{t+1}$$ And since $f(\frac{t-3}{t+1})+f(\frac{3+t}{1-t})=t$, it implies $$2f(t)+t=\frac{8t}{1-t^2}.$$ Thus the function is $f(t)=\frac{4t}{1-t^2}-\frac{t}{2}$. It is easy to check that this satisfies the given equation. 
	\item A convex pentagon with side integer side lengths and an odd perimeter can have two right angles as a unit equilateral triangle on top of a unit square suffices. 

		Clearly a pentagon can't have 5 right angles. If a pentagon has 4 right angles, then the remaining angle is $540-4\times 90=180^{\circ}$ which is a contradiction. Hence we consider when a pentagon has 3 right angles.
There are two cases, when exactly two of the right angles are adjacent, and when all three of the right angles are adjacent. 

		Case 1: Three of the right angles are adjacent. 

\centerline{\includegraphics[width=2.5in, height=2.3in]{case 1.png}}

		Then the perimeter is $2(x+y)+(c-a-b)$. Clearly $2(x+y)$ is even, and $c-a-b$ is even since $a^2+b^2=c^2$ implies $a+b$ has the same parity as $c$. 

Case 2: Exactly two of the right angles are adjacent. \\ \centerline{\includegraphics[width=2.5in, height=2.3in]{case 2 .png}}
Let $x$ and $y$ be the sides of the rectangle formed with the dotted lines. Then from similar triangles we deduce the perimeter is $$2(x+y)-a-b-kb-ka+c+kc$$ $$= 2(x+y)+(c-a-b)(1+k)$$ Which is even as $2(x+y)$ and $c-a-b$ are even. 

		Alternatively, since $a^2+b^2=e^2=c^2+d^2$, $a+b$ has the same parity as $c+d$. Meaning $a+b+c+d$ is even. Since the perimeter is $a+b+c+d+2y$, the pentagon has an even perimeter. 


		{\centerline{\includegraphics[width=2.5in, height=2.3in]{case 2 alternative.png}}}

 Hence a pentagon with integer side lengths and odd perimeter can have at most two right angles.
\item The key observation is that \(11\times29=319\). This means that exactly
	\(319\) ordered pairs, and hence \(319\) numbers
		of the form \(11x+29y\), will be formed. 
		So all we need to prove that those numbers are incongruent
		to each modulo \(319\), as that would mean there is
		a bijection between those numbers
		and the numbers \(1,2,\ldots,319.\)

		This problem is a special case of the general theorem
		that if \(a\) runs through a complete set of residues
		modulo \(m\) and \(a'\) runs through a complete set of
		residues modulo \(m'\), and \(m\) and \(m'\) are coprime,
		then the \(mm'\) numbers
		\[am'+a'm\] form a complete set of residues modulo \(mm'\).
		To prove it, suppose that
		\begin{equation}\label{eq:congr}
			a_1m'+a_1'm\equiv a_2m'+a_2'm\pmod{mm'}.
		\end{equation}
		Clearly the congruence also holds relative to the modulus
		\(m\), so the terms with \(m\) in them disappear and
		we are left with
		\[a_1m'\equiv a_2m'\pmod{m}.\]
		Since \(m'\) and \(m\) are coprime we can divide both sides
		by \(m'\) and we have \[a_1\equiv a_2\pmod{m}.\]
		Both \(a_1\) and \(a_2\) are running through the residues
		of \(m\) and therefore both are less than \(m\); hence
		\(a_1=a_2\).

		The congruence in \eqref{eq:congr}
		also holds relative to the modulus
		\(m'\), so we have
		\[a_1'm\equiv a_2'm\pmod{m'}.\]
		Again we can deduce that
		\(a_1'\equiv a_2'\pmod{m'}\) 
		from which it follows that
		\(a_1'=a_2'\).
		Since \(a_1=a_2\) and \(a_1'=a_2'\)
		two numbers of the
		form \(a'm+am'\) that are congruent
		are actually the same number,
		and this proves the result.
		
		The problem is a case of the theorem where \(m=11\)
		and \(m'=29\); clearly they are coprime so the theorem proved
		above can be applied.
\end{enumerate}
\end{document}

