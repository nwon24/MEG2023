\documentclass[a4paper]{article}

\newcommand{\theterm}{}
\newcommand{\theweek}{}
\newcommand{\thepdftitle}{MEG 2023 Term 3 Holiday Handout}
\newcommand{\thedisplaytitle}{MEG 2023 Term 3 Holiday Handout}

\title{{\thepdftitle}}
\author{Nathan Wong\and Tom Yan}
\date{2023}

\newcommand{\leg}[2]{\left(\frac{#1}{#2}\right)}
\newcommand{\ileg}[2]{(#1|#2)}

\newcommand{\floor}[1]{\lfloor#1\rfloor}
\newcommand{\bfloor}[1]{\left\lfloor#1\right\rfloor}
\newcommand{\ceil}[1]{\lceil#1\rceil}
\newcommand{\bceil}[1]{\left\lceil#1\right\rceil}

%\newcommand{\marginfn}[1]{\marginpar{\footnotemark}\footnotetext{#1}}
\newcommand{\marginnote}[1]{\marginpar{\footnotesize{#1}}}
\newcommand{\marginfnote}[1]{\footnotemark\marginpar{\footnotemark[\value{footnote}]\footnotesize{#1}}}
\usepackage{geometry}
%\geometry{a4paper,left=24.8mm,top=27.4mm,headsep=2\baselineskip,textwidth=107mm,marginparsep=8.2mm,marginparwidth=49.4mm,textheight=49\baselineskip,headheight=\baselineskip}
\geometry{a4paper,left=1in,top=1in,bottom=1in,headsep=2\baselineskip,textwidth=107mm,marginparsep=8.2mm,marginparwidth=49.4mm,textheight=49\baselineskip,headheight=\baselineskip}
\usepackage[bf,tiny]{titlesec}
%\usepackage{fancyhdr}
\usepackage{epigraph}
%\usepackage[indent=0pt,skip=10pt]{parskip}

\usepackage{amsmath}
\usepackage{amsthm}
\newtheorem{theorem}{Theorem}
\usepackage{amssymb}
\let\mathbbalt\mathbb

\usepackage{fontspec}
\usepackage{unicode-math}
\let\mathbb\mathbbalt

\newcommand{\naturals}{\mathbb{N}}
\newcommand{\reals}{\mathbb{R}}
\newcommand{\rationals}{\mathbb{Q}}
\newcommand{\integers}{\mathbb{Z}}

\usepackage[pdfusetitle]{hyperref}

\newcommand{\myquote}[2]{%
  \begin{quote}
    \emph{#1}
    \begin{flushright}---{#2}
    \end{flushright}
  \end{quote}}
\pagestyle{empty}
\begin{document}
\noindent Melbourne High School\\\
\noindent Maths Extension Group 2023\\\
\noindent \textbf{\thedisplaytitle}\\\
\myquote{For too much rest itself becomes a pain.}{Homer, \emph{Odyssey} (c.~8th or 7th century BC)}
\section*{Problems}
\begin{enumerate}
	\item The \marginnote{AIMO 2017/1} number $x$ is $111$ when written in base $b$, but it is $212$ when written in base $b-2$. What is $x$ in base $10$? 
	\item Find \marginnote{AIME II 2021/3} the number of permutations $x_1, x_2, x_3, x_4, x_5$ of numbers $1, 2, 3, 4, 5$ such that the sum of five products\[x_1x_2x_3 + x_2x_3x_4 + x_3x_4x_5 + x_4x_5x_1 + x_5x_1x_2\]is divisible by $3$
	\item Let \marginnote{Simson Line} $ABC$ be a triangle and $P$ be any point on the circumcircle of $ABC$. Let $X, Y, Z$ be the feet of the perpendiculars from $P$ onto lines $BC, CA$ and $AB$. Prove that points $X, Y, Z$ are collinear.
	\item Find all functions $f: \mathbb{R} \backslash \{-1, 1\} \rightarrow \mathbb{R} $ for which $$f\left(\frac{x-3}{x+1}\right)+f\left(\frac{3+x}{1-x}\right)=x$$ for all $x,y \in \mathbb{R} \backslash \{-1, 1\}$. 
	\item Prove \marginnote{AMO 2022/1} that a convex pentagon with integer side lengths and an odd perimeter can have two right angles, but cannot have more than two right angles.  
	\item Having conquered \emph{The Trial}, Harold is now reading
		\emph{Gulliver's Travels}, a \(319\) page picture book.
		Eager as he is to dive into
		the travels of Gulliver, he also has a brilliant idea
		as to how he is going to read the book. 
		He tells you that he is going to form the set of ordered
		pairs 
		\[
			\begin{split}
				\{(1,1),(1,2),\ldots,(1,11),(2,1),(2,2),\ldots,(2,11),\\(3,1),(3,2),\ldots,(29,1),\ldots,(29,11)\}.
			\end{split}\]
		As he is a genius (and a good friend too), you don't doubt him.

		As if that was not distressing enough, he then says that he
		will randomly arrange those ordered pairs by giving each
		a number. So if the above set was the arrangement, the
		pair \((1,1)\) would be \(1\), \((1,2)\) would be \(2\),
		and so on. To find the \(n\)th page that he reads, he
		takes the \(n\)th ordered pair \((x_n,y_n)\) and forms
		the number \(11x_n+29y_n\); if this number is too big
		he subtracts off a multiple of \(319\) until he gets
		a page in the book. He then reads that page.

		Show that regardless of how Harold arranges the ordered
		pairs, he will be able to read the entire book without
		reading the same page twice.
		
\end{enumerate}
\pagebreak
\myquote{The king, although he be as learned a person as any in his dominions, had been educated in the study of philosophy, and particularly mathematics; yet when he observed my shape exactly, and saw me walk erect, before I began to speak, conceived I might be a piece of clock-work (which is in that country arrived to a very great perfection) contrived by some ingenious artist.}{J.~Swift, \emph{Gulliver's Travels} (1726)}
\section*{A Golden Conjecture: Part 2}
Easily summarised is the story so far.
We first discovered that
\[\leg{-1}{p}=\begin{cases}
	1&\text{if } p\equiv1\pmod{4};\\
	-1&\text{if } p\equiv-1\pmod{4}
\end{cases}
\]
and 
\[
	\leg{2}{p}=\begin{cases}
		1&\text{if } p\equiv\pm1\pmod{8};\\
		-1&\text{if } p\equiv\pm3\pmod{8}.
	\end{cases}
\]
Recall that the rule for \(\ileg{-1}{p}\) is more succinctly
written as 
\[
	\leg{-1}{p}=(-1)^{(p-1)/2}.
\]
As a side note, \(\ileg{2}{p}\) has a succinct rule too, namely
\marginnote{Verifying this by enumerating \(p\) shreds any incredulity.}
\[
	\leg{2}{p}=(-1)^{(p-1)(p+1)/8}.
\]
We then took a slight detour to gather more
numerical evidence; the following two conjectures resulted:
\begin{itemize}
	\item The value of \(\ileg{a}{p}\) depends only
		on the residue class of \(p\) modulo \(4a\),
		or, equivalently, the value of \(r\) when
		\(p\) is written as \(p=4ak+r.\)
		In other words, if \(p\equiv q\pmod{4a}\) then \[\leg{a}{p}=\leg{a}{q}.\]
	\item The value of \(\ileg{a}{p}\) is the same for
		\(p=4ak+r\) as it is for \(p=4ak-r\).
		\marginnote{Note this conjecture holds regardless
		of whether \(r\) is positive or negative or whether
		\(r\) is between \(-p/2\) and \(p/2\).
		For \(\ileg{2}{p}\) it was prudent to split the residue
		classes modulo \(8\) into \(\pm1\) and \(\pm3\); but
		we could have chosen \(\pm7\) and \(\pm5\) instead,
		because \(7\equiv-1\), \(-7\equiv1\), and similarly
		for \(\pm5\).}
		In other words, if \(p\equiv-q\pmod{4a}\) then \[\leg{a}{p}=\leg{a}{q}.\]
\end{itemize}
These two conjectures found firm ground in two special cases
that were proved using Gauss's Lemma. They are
\[
	\leg{3}{p}=
	\begin{cases}
		1&\text{if } p\equiv\pm1\pmod{12};\\
		-1&\text{if }p\equiv\pm5\pmod{12},
	\end{cases}
\]
and 
\[
	\leg{5}{p}=
	\begin{cases}
		1&\text{if } p\equiv\pm1\text{ or }p\equiv\pm11\pmod{20};\\
		-1&\text{if }p\equiv\pm3\text{ or }p\equiv\pm13\pmod{20}.
	\end{cases}
\]
Euler was the first to make the two conjectures that we have also
made; therefore from here they will be referred to as Euler's Conjectures.
He was, however, unable to supply a proof.\marginnote{It's fairly rare
to find a conjecture that eluded Euler, but this is one of them.}
The problem of Gauss's Lemma not being discovered yet might have had
something to do with it.

%Before the proof, it is a good idea to review the
%case \(a=3\). It involved finding how many multiples of \(3\)
%lie in the interval \[\left(\frac{p}{2},p\right),\]
%as those are the multiples of \(3\) that become negative when
%reduced into the range \((-p/2,p/2)\). We completed
%the proof by dividing the interval through
%by \(3\), putting \(p=12k+r\) for \(r=\pm1,\pm5\),
%and determining the parity of the number of numbers
%in the interval in each case.

We begin the proof of Euler's conjectures by applying Gauss's
Lemma in full generality. Form the set 
\[\left\{a,2a,\ldots,\frac{p-1}{2}a\right\};\] how many of those multiples
of \(a\) become negative when reduced to be in the range \((-p/2,p/2)\)?
Answer: all those that have least positive residues in the
range \((p/2,p)\). This means we need to determine the parity 
of all the multiples of \(a\) in all the intervals\marginnote{Since 
we are assuming that \(a\) is not a multiple of \(p\) (because if it
were the Legendre symbol, by definition, is equal to \(0\)), none of the endpoints
of any of the intervals can be multiplies of \(a\). Therefore we don't
have to worry about whether we need to count them or not.}
\begin{equation}\label{eq:intervs}
	\left(\frac{p}{2},p\right),\left(\frac{3p}{2},2p\right),\left(\frac{5p}{2},3p\right),\ldots.
\end{equation}
Let's take one of these intervals, say the interval
\begin{equation}\label{eq:interv}
	\left(\frac{(2t-1)p}{2},tp\right)
\end{equation}
for some positive integer \(t\). Now the question
becomes how many numbers \(x\) satisfy the inequality
\[\frac{(2t-1)p}{2}<ax<tp.\]
Let this number be \(\mu\).

Dividing through by \(a\) yields
\begin{equation}\label{eq:tpint}
	\frac{(2t-1)p}{2a}<x<\frac{tp}{a}.
\end{equation}
Before continuing, let's take a step back and
think about what we want to prove.
Our conjecture is that \(\ileg{a}{p}\)
depends only on the residue of \(p\)
modulo \(3a\). So let's write \(p=4ak+r\)
for a positive integer value of \(r\).
Substitute this into our interval above, and,
after a cumbersome expansion and simplification,
we get 
\[
	4tk+\frac{tr}{a}-2k-\frac{r}{2a}<x<4tk+\frac{tr}{a}.
\]
Since we are interested only in the number of
values of \(x\) (actually we are interested in
the \emph{parity} of that number) 
\marginnote{Recall that we are interested only
in the parity because of Gauss's Lemma, which
tells us that \(\ileg{a}{p}=(-1)^v\) where \(v\)
is the number of numbers in all the intervals in \eqref{eq:intervs}.}
that satisfy the inequality, and
not what those values actually are, we can manipulate
the inequality by adding the same value to both 
endpoints, and the number of values of \(x\) that
satisfy it will not change. For example,
two values of \(x\) satisfy \(3<x<6\) (\(4\) and \(5\)), but two values of \(x\) also satisfy \(5<x<8\)
or \(1<x<4\).

Therefore, we can get rid of all the terms involving \(t\)
in the inequality, which means we are looking for
the number of values of \(x\) that satisfy
\[ -2k-\frac{r}{2a}<x<0.\]
It is easier to work with positive numbers, so
add \(2k\) and \(r/a\) to both sides to obtain
the inequality 
\[ \frac{r}{2a}<x<2k+\frac{r}{a}.\]
The smallest integer greater than
\marginnote{Note that the \(\ceil{x}\) is the
\emph{ceiling function}; it is the smallest integer
greater than or equal to \(x\).}
\(r/2a\) is \(\ceil{r/2a}\) and 
the largest integer less than \(2k+r/a\)
is \(2k+\floor{r/a}\).
\marginnote{Is the use of floor and ceiling functions
here dependent on \(r\) being positive?}
Therefore
the number of numbers in the 
sequence
\[\ceil{r/2a},\ceil{r/2a}+1,\ldots,2k+\floor{r/a}\]
is \(\mu\), the magic number we are looking
for. Calculating \(\mu\) is very simple
because\marginnote{If this is baffling, remember the
example from last time. How many numbers \(x\)
satisfy \(5\le x\le 9\)? The answer is \(9-5+1=5\).}
\[\mu=2k+\floor{r/a}-\ceil{r/2a}+1.\]
Since we are only interested in the parity of
\(\mu\), we note that 
\[2k+\floor{r/a}-\ceil{r/2a}+1\equiv\floor{r/a}-\ceil{r/2a}+1\pmod{2}\]
and hence \(\mu\equiv\mu'\pmod{2}\) if \(\mu'\)
is the number of integers in the interval
\[\left(\frac{r}{2a},\frac{r}{a}\right).\]
Take a breath. What have we shown?
Answer: regardless of which interval in 
\eqref{eq:intervs} we choose, the parity of the number
of numbers in that interval is equal
to the parity of the number of numbers in the interval \((r/2a,r/a)\).

Now consider what happens if, instead of
writing \(p=4ak+r\), we write \(p=r\) instead.
That is, substitute \(p=r\) into \eqref{eq:tpint}.
Expand and we get 
\[\frac{tr}{a}-\frac{r}{2a}<x<\frac{tr}{a}.\]
Again we can manipulate the interval like we did
before; getting rid of the \(tr/a\) term and
adding \(r/a\) to both sides yields
\[\frac{r}{2a}<x<\frac{r}{a}.\]
Bingo! This is the interval we ended up getting
when we put \(p=4ak+r\). This means that
for each interval in \eqref{eq:intervs}, the 
parity of the number of numbers in that interval
is the same when \(p=4ak+r\) and when \(p=r\);
therefore the parity of the total number of numbers
in all the intervals is the same in both cases.
Thus, the glorious conclusion, after much work,
is that \(\ileg{a}{p}\) does not depend on
\(k\) if \(p=4ak+r\), but only on \(r\), the
residue class of \(p\) modulo \(4a\). We have
proved the first part of Euler's Conjecture.

The second part of Euler's Conjecture is, quite
thankfully, a little easier to prove. We need
to show that replacing \(r\) by \(-r\) does
not change the parity of the number of numbers
in the intervals in \eqref{eq:intervs}.

We have shown that the parity of the number of
numbers in each of the intervals in \eqref{eq:intervs}
is the same as the parity of the number of 
numbers in the interval \[\left(\frac{r}{2a},\frac{r}{a}\right).\]
If we replace \(r\) by \(-r\), we get the interval
\[\left(\frac{-r}{a},\frac{-r}{2a}\right);\]
note the reversal of the endpoints because \(-r\)
is negative.
Now just add \(3r/2a\) to both sides and we get
the interval \((r/2a,r/a)\), which shows that
replacing \(r\) by \(-r\) has no effect on the parity
of the numbers in the intervals in \eqref{eq:intervs}.
Hence the second part of the conjecture is also true.

What we have proved can be thus summarised: if
\(p\equiv r\pmod{4a}\) and \(q\equiv r\pmod{4a}\),
or if \(p\equiv r\pmod{4a}\) and \(q\equiv-r\pmod{4a}\), 
then \[\leg{a}{p}=\leg{a}{q}.\] 
More succinctly:
if \(p\equiv\pm q\pmod{4a}\) then \[\leg{a}{p}=\leg{a}{q}.\]
Where does this leave us? It leaves us on the precipice
of the most famous theorem in this area of number 
theory.
\end{document}
