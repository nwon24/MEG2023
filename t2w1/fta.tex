%\pagewidth 210mm
%\pageheight 297mm
\documentclass[a4paper,10pt]{article}

\usepackage[pdfusetitle]{hyperref}

\usepackage[a4paper,margin=1in]{geometry}
\usepackage[bf,tiny]{titlesec}
\usepackage{fancyhdr}

%\usepackage[indent=0pt,skip=10pt]{parskip}

\usepackage{amsmath}
\usepackage{amsthm}
\newtheorem{theorem}{Theorem}
\usepackage{amssymb}
\let\mathbbalt\mathbb

\usepackage{fontspec}
\usepackage{unicode-math}
\let\mathbb\mathbbalt
\setmainfont{TeX Gyre Termes}
\setmathfont{TeX Gyre Termes Math}
\newcommand{\naturals}{\mathbb{N}}
\newcommand{\reals}{\mathbb{R}}
\newcommand{\rationals}{\mathbb{Q}}
\newcommand{\integers}{\mathbb{Z}}

\newcounter{problemno}
\setcounter{problemno}{1}
\newcommand{\problem}{\section*{\textbf{Problem \theproblemno}}\refstepcounter{problemno}}
\newcounter{interludeno}
\setcounter{interludeno}{1}
\newcommand{\interlude}[1]{\par\noindent\\\fbox{\begin{minipage}{\linewidth}\textbf{Interlude \theinterludeno.\enspace}#1\end{minipage}}\\\refstepcounter{interludeno}}

\newcounter{questionno}
\setcounter{questionno}{1}
\newcommand{\question}[1]{\par\noindent{\thequestionno.\enspace}#1\refstepcounter{questionno}\\}

\newcommand{\thetitle}{Term 2 Week 1 Handout}

\renewcommand\thesection{\S{\arabic{section}}.}
\renewcommand\thesubsection{\S{\arabic{section}.\arabic{subsection}}}

\title{MEG 2023 Term 2 Week 1 Handout}
\author{Nathan Wong\and Tom Yan}
\date{2023}
\begin{document}
\noindent Melbourne High School\\
Maths Extension Group 2023\\
\textbf{\thetitle}\\
\twocolumn
\section{From Euclid to Gauss}
The Fundamental Theorem of Arithmetic states a seemingly obvious, but certainly not trivial, fact about
the positive integers. It asserts that every positive integer can be uniquely expressed as a product of
prime numbers; that is, there is only way to break a number down into primes. (Of course, we consider
rearranging the prime factors to be the same factorisation.) For example, the
number \(12\) is \(2\times2\times3\); it is impossible to write down a factorisation of \(12\) without
using \(2\) exactly twice and \(3\) exactly once.

The idea that unique factorisation is an obvious property of the integers was taken for granted for
thousands of years. Furthermore, mathematicians assumed that prime factorisation holds for other
number systems, which is simply not true. For example, consider numbers of the form
\(4k+1\); these are the numbers beginning \[1,5,9,13,17,21,\ldots.\] A ``prime'' in this system
of numbers is a number that is not able to factorised without going outside of the set of numbers.
The first non-prime in this system of numbers is \(25\). Just like with the positive integers, each
number is either a ``prime'' or able to factorised as a product of primes, but this factorisation
is not always unique. For example, \(693=9\times77=21\times33\), and \(77\), \(9\), \(21\), and \(13\)
are all primes in this number system.

If prime factorisation were a general theorem for virtually
all number systems, Fermat's Last Theorem would have been successful proven in the 19th century.
The really first rigorous proof of the fundamental theorem was given by Gauss in the early 19th century and
used modular arithmetic. Here we shall consider several common proofs of the theorem.

Before we examine \emph{unique} factorisation, we first need to verify that every number is actually
a product of primes. Fortunately this is easy and follows at once by the definition of prime and composite numbers.
The method is by strong induction; try it for yourself.

The real crux of the problem is showing that a prime factorisation is necessarily unique. The most
common proof uses Euclid's Lemma, the proof of which we left as an exercise on a previous handout.
For completeness, here is the proof of the lemma.

Recall that Euclid's Lemma asserts that if a prime \(p\) divides a product \(ab\), then \(p\) must
divide at least one of \(a\) and \(b\). That is, \(p\) either divides \(a\), \(b\), or both. To begin
the proof, we assume that \(p\) does not divide \(a\), because if it did, then we have nothing to prove.
Now the aim is to show that \(p\) divides \(b\).

If \(p\) does not divide \(a\), then \(\gcd(p,a)\) must be equal to \(1\). This follows immediately
from \(p\) having, by definition, only itself and \(1\) as factors. Since \(\gcd(p,a)=1\), there
exist positive integers \(x\) and \(y\) such that \[px-ay=1.\] (The proof of this was given in the same
previous handout on the Euclidean algorithm.)

Multiplying both sides by \(b\), we obtain \[pbx-aby=b.\] We have found an expression for \(b\). By hypothesis,
\(p\) divides \(ab\), so \(p\) divides \(aby\). Certainly \(p\) divides \(pbx\), so \(p\) divides both \(pbx\)
and \(aby\). Hence \(p\) also divides their difference \(pbx-aby\), which is \(b\). This completes the proof
of the lemma.

Using this lemma, we are able to prove prime factorisation by contradiction. Assume that there exists
some positive integer \(n\) that has two distinct factorisations: let \[n=p_1p_2\cdots p_r=q_1q_2\cdots q_s\]
where all of \(p_1,p_2,\ldots,p_r,q_1,q_2,\ldots,q_s\) are primes. Further suppose that this is the smallest
number with the property. We can assume this because it is a proerty of the natural numbers that every subset
has a least element; hence the set of all natural numbers with at least two different prime factorisations also
has a least element.

From our definition of \(n\), we see that \(p_1\) divides \(n=q_1q_2\cdots q_s\). By Euclid's lemma \(p_1\) divides
either \(q_1\) or \(q_2q_3\cdots q_s\). If \(p_1\) doesn't divide \(q_1\) then it divides \(q_2q_3\cdots q_s\). But by
reapplication of Euclid's lemma \(p_1\) divides either \(q_2\) or \(q_3q_4\cdots q_s\). Eventually we will find that
\(p_1\) divides at least one of the prime factors \(q_i\). Assume without loss of generality that this prime is \(q_1\).

Since \(p_1\) and \(q_1\) are both primes, if \(p_1\) divides \(q_1\), then \(p_1\) must equal \(q_1\). This means
we can cancel out \(p_1\) and \(q_1\) from both sides of the equation, which yields a smaller number that can
be expressed as a product of primes in two different ways. This contradicts our assumption that \(n\) is the smallest
number with that property, completing the proof.

\end{document}
