\documentclass[a4paper]{article}

\newcommand{\theterm}{2}
\newcommand{\theweek}{1}
\newcommand{\thepdftitle}{MEG 2023 Term \theterm\ Week \theweek\ Handout}
\newcommand{\thedisplaytitle}{Term \theterm\ Week \theweek\ Handout}

\title{{\thepdftitle}}
\author{Nathan Wong\and Tom Yan}
\date{2023}

%\newcommand{\marginfn}[1]{\marginpar{\footnotemark}\footnotetext{#1}}
\newcommand{\marginnote}[1]{\marginpar{\footnotesize{#1}}}
\newcommand{\marginfnote}[1]{\footnotemark\marginpar{\footnotemark[\value{footnote}]\footnotesize{#1}}}
\usepackage{geometry}
%\geometry{a4paper,left=24.8mm,top=27.4mm,headsep=2\baselineskip,textwidth=107mm,marginparsep=8.2mm,marginparwidth=49.4mm,textheight=49\baselineskip,headheight=\baselineskip}
\geometry{a4paper,left=1in,top=1in,bottom=1in,headsep=2\baselineskip,textwidth=107mm,marginparsep=8.2mm,marginparwidth=49.4mm,textheight=49\baselineskip,headheight=\baselineskip}
\usepackage[bf,tiny]{titlesec}
%\usepackage{fancyhdr}
\usepackage{epigraph}
%\usepackage[indent=0pt,skip=10pt]{parskip}

\usepackage{amsmath}
\usepackage{amsthm}
\newtheorem{theorem}{Theorem}
\usepackage{amssymb}
\let\mathbbalt\mathbb

\usepackage{fontspec}
\usepackage{unicode-math}
\let\mathbb\mathbbalt

\newcommand{\naturals}{\mathbb{N}}
\newcommand{\reals}{\mathbb{R}}
\newcommand{\rationals}{\mathbb{Q}}
\newcommand{\integers}{\mathbb{Z}}

\usepackage[pdfusetitle]{hyperref}

\newcommand{\myquote}[2]{%
  \begin{quote}
    \emph{#1}
    \begin{flushright}---{#2}
    \end{flushright}
  \end{quote}}
\pagestyle{empty}
\begin{document}
\noindent Melbourne High School\\
\noindent Maths Extension Group 2023\\
\noindent \textbf{\thedisplaytitle}\\

\myquote{If two numbers by multiplying one another make some number, and any prime number measure the product, it will also measure one of the original numbers.}{Euclid, \emph{Elements Book VII} (c.~300 BC)}
\section*{The Fundamental Theorem of Arithmetic}
Prime numbers are to the natural numbers what atoms are to matter. They are the fundamental, indivisible
building blocks of the universe we call number theory. As such, we might expect there is some grand
theorem that explains the importance of primes in higher arithmetic, and in fact there is; this
theorem
is the Fundamental Theorem of Arithmetic.

Let's begin by considering the number \(12\). Clearly \(12\) is composite, because it is divisible by \(2\).
So we write \(12=2\times 6\). Turning our attention to \(6\), we realise it too is composite because it is \(2\times 3\).
The numbers \(2\) and \(3\) being prime, we cannot factorise further, leaving us with \(12=2\times2\times3.\).
This is called the prime factorisation of \(12\). What the Fundamental Theorem of Arithmetic asserts is that
there is no other way to write \(12\) as a product of primes other than what we have written down. That is,
it is impossible to write down a prime factorisation of \(12\) without using \(2\) exactly twice and \(3\) exactly once
and using no other primes. Of course, \(12\) was just an example; we could've picked any positive integer. But no
matter which number we might have picked, the Fundamental Theorem of Arithmetic guarantees that any prime factorisation
we work out is unique.\marginnote{Just to be clear, by `unique' we mean unique up to reordering of the factors. So we may write
  \(12=2\times2\times3\) or \(12=2\times3\times2\), but we have still used \(2\) twice and \(3\) once. Also note how
  this theorem precludes \(1\) from being a prime, because inserting any number of \(1\)s doesn't change the value of
the product.}

Mathematicians considered unique factorisation an obvious property of the integers for thousands of years.
To illustrate how special this property of the natural numbers is, consider numbers of the form
\(4k+1\). These are the numbers beginning \[1,5,9,13,17,21,\ldots.\] A ``prime'' in this system
 is a number that cannot be factorised without going outside of the set of numbers.
The first non-prime in this system of numbers is \(25=5\times5\). Just like with the positive integers, each
number is either a prime or able to factorised as a product of primes, but this factorisation
is not always unique. For example, \(693=9\times77=21\times33\), and \(77\), \(9\), \(21\), and \(13\)
are all primes in this number system.

Before we examine \emph{unique} factorisation, we first need to verify that every number is actually
a product of primes.\marginnote{Note that there is something fishy about \(1\). This is why we define
the empty product to be equal to \(1\), so that \(1\) is still a product of primes---it is the product of zero primes.}
Fortunately this is easy and follows at once by the definition of prime and composite numbers.
The method is by strong induction; try it for yourself.

The real crux of the problem is showing that a prime factorisation is necessarily unique. The most
common proof uses Euclid's Lemma, the proof of which we left as an exercise on a previous handout.
Here is the proof for completeness.

Euclid's Lemma asserts that if a prime \(p\) divides a product \(ab\), then \(p\) must
divide at least one of \(a\) and \(b\). To begin
the proof, we assume that \(p\) does not divide \(a\), because if it does then we have nothing to prove.
Now the aim is to show that \(p\) divides \(b\).

If \(p\) does not divide \(a\), then \(\gcd(p,a)\) must be equal to \(1\). This follows immediately
from \(p\) having only itself and \(1\) as factors by definition. Since \(\gcd(p,a)=1\), there
exist positive integers \(x\) and \(y\) such that \[px-ay=1.\] (The proof of this was given in the same
previous handout on the Euclidean algorithm.)

Multiplying both sides by \(b\), we obtain \[pbx-aby=b.\] By hypothesis,
\(p\) divides \(ab\), so \(p\) divides \(aby\). Certainly \(p\) divides \(pbx\), so \(p\) divides both \(pbx\)
and \(aby\). Hence \(p\) also divides their difference \(pbx-aby\), which is \(b\). This completes the proof
of the lemma.

We are now able to prove prime factorisation by contradiction. Assume that there exists
some positive integer \(n\) that has two distinct factorisations: let \[n=p_1p_2\cdots p_r=q_1q_2\cdots q_s\]
where all of \(p_1,p_2,\ldots,p_r,q_1,q_2,\ldots,q_s\) are primes. Further suppose that this is the smallest
number with the property. We can assume this because every subset of the natural numbers
has a least element;\marginnote{The property of every subset having a least element is called the \emph{well-ordering property} of the natural numbers. It is certainly
not true for real numbers. Is there a least element of the set of real numbers in the interval \((0,1)\)?} hence the set of all natural numbers with at least two different prime factorisations also
has a least element.

From our definition of \(n\), we observe that \(p_1\) divides \(n=q_1q_2\cdots q_s\). By Euclid's lemma \(p_1\) divides
either \(q_1\) or \(q_2q_3\cdots q_s\). If \(p_1\) doesn't divide \(q_1\) then it divides \(q_2q_3\cdots q_s\). But by
reapplication of Euclid's lemma \(p_1\) divides either \(q_2\) or \(q_3q_4\cdots q_s\). Eventually we find that
\(p_1\) divides at least one of the prime factors \(q_i\) for some \(1\le i\le s\). Assume without loss of generality that this prime is \(q_1\).

Since \(p_1\) and \(q_1\) are both primes, if \(p_1\) divides \(q_1\), then \(p_1\) must equal \(q_1\). This means
we can cancel out \(p_1\) and \(q_1\) from both sides of the equation, which yields a smaller number that can
be expressed as a product of primes in two different ways. This contradicts our assumption that \(n\) is the smallest
number with that property, completing the proof.

This proof, although apparently short, is not the most direct method because it relies
on the development of the Euclidean algorithm and the properties of numbers and their greatest common divisors. Gauss
offered a shorter proof using modular arithmetic, and there is an even more direct
proof that does not require any preliminary lemmas. Both are left as exercises.

The property of prime factorisation seems neat, but why is it useful? There are many reasons why the theorem deserves
its grand name; the main one is that it allows us to develop theorems about all the natural numbers by first
focusing on the primes. If we can answer some question about the primes, it is likely we can answer the same question
about all positive integers. Consider the following example.

We define a function \(\sigma(n)\) to be the sum of the divisors of \(n\). For example, \(\sigma(6)=1+2+3+6=12\).
\marginnote{I chose \(6\) because it's an example of a \emph{perfect number}, which is a number that is equal to
  the sum of its divisors except itself. This means a number \(n\) is perfect if \(\sigma(n)=2n.\)}
Suppose our goal is to find a formula for \(\sigma(n)\). By computing the first twenty or so values, we observe that
the formula is unlikely to be a nice algebraic function, since \(\sigma(n)\) doesn't seem to be increasing, decreasing,
or exhibiting any sort of behaviour we might expect from a polynomial or exponential function. The way to get started
is by looking first at the case when \(n\) is prime.

What is \(\sigma(p)\) for any prime \(p\)? This is easy because the only divisors of \(p\) are itself and \(1\),
so \(\sigma(p)=p+1\). The next step up is to consider powers of \(p\). What is \(\sigma(p^k)\) for any nonnegative integer \(k\)?
Here is our first use of the fundamental theorem. Since the factorisation \(p^k\) is unique, no other prime can divide it.
Hence the only divisors are the powers of \(p\), namely \(1,p,p^2,\ldots,p^k\). Summing these together yields
\[\sigma(p^k)=1+p+p^2+\cdots+p^k.\]
This sum is just a geometric series with first term \(1\) and common ratio \(p\); therefore it has the following nice formula:
\[\sigma(p^k)=\frac{p^{k+1}-1}{p-1}.\]
We have succeeded in the first part of our plan. Having found a formula for \(\sigma(p^k)\) we can find a general
formula for \(\sigma(n)\). This is left as an exercise.

This idea of first examining a problem for primes and then generalising to all natural numbers is a common theme
in number theory. Another example is the classical problem about which numbers are the sum of two squares, which we shall examine in the future.
Fermat conjectured, and Euler proved, that every prime of the form \(4k+1\) is the sum of two squares. Using this result
and a multiplicative property of such numbers, Euler
was able to determine a general criterion for which numbers can be expressed as the sum of two squares. 

\subsection*{Problems}
\begin{enumerate}
\item In this problem we complete out investigation of perfect numbers.
  \begin{enumerate}
\item 
  The divisors of \(5\) are \(1\) and \(5\), and the divisors of \(4\) are \(1\), \(2\), and \(4\). Note that the divisors of
  their product, \(20\), are \(1\), \(2\), \(4\), \(5\), \(10\), and \(20\). By considering further examples if necessary, show
  that if \(m\) and \(n\) are coprime then \[\sigma(mn)=\sigma(m)\sigma(n).\]
\item By the Fundamental Theorem of Arithmetic, every positive integer can be written as
  \[n=p_1^{a_1}p_2^{a_2}p_3^{a_3}\cdots p_r^{a_r}\]
  where all of \(p_1,p_2,\ldots,p_r\) are distinct primes.
  Find a general formula for \(\sigma(n)\).
\end{enumerate}
\item
  \begin{enumerate}
  \item From our present view, knowing the Fundamental Theorem, it is obvious that if two numbers are both less than
    a prime \(p\), then their product is not divisible by \(p\), because \(p\) cannot possibly occur in the factorisation
    of either of the two numbers. Prove this without relying on unique factorisation.
  \item    \marginnote{This is equivalent to Euclid's lemma (it's the contrapositive). Therefore by completing this question you have
      also proved the Fundamental Theorem of Arithmetic in a different way.}
    Hence show that if \(a\not\equiv 0\pmod{p}\) and \(b\not\equiv 0\pmod{p}\) then \(ab\not\equiv 0\pmod{p}\).
  \end{enumerate}
\item
  \begin{enumerate}
  \item  Suppose that a number \(n\) has two different prime factorisations and it is the smallest
  number with the property. This means all of \(1,2,\ldots,n-1\) can be uniquely factorised. Let
  \[n=p_1p_2\cdots p_r=q_1q_2\cdots q_s\]
  where all of \(p_1,p_2,\ldots,p_r,q_1,q_2,\ldots,q_s\) are primes. Note that none of the \(p\)'s equal
  any of the \(q\)'s, because otherwise they could be cancelled to yield a smaller number with
  two different factorisation, which contradicts our definition of \(n\).
  
  Show that there exist numbers \(i\) and \(j\) such that \(n-p_iq_j>0\).
\item Let \(N=n-p_1q_1.\) Clearly, \(N\) is less than \(n\), and so can be factorised in only one
  way by hypothesis. Show that the product \(p_1q_1\) must divide \(n\).
\item Hence complete the proof of the Fundamental Theorem of Arithmetic.
\end{enumerate}
\end{enumerate}
\end{document}
