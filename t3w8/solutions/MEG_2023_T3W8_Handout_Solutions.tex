\documentclass[a4paper]{article}

\newcommand{\theterm}{3}
\newcommand{\theweek}{8}
\newcommand{\thepdftitle}{MEG 2023 Term \theterm\ Week \theweek\ Handout Solutions}
\newcommand{\thedisplaytitle}{Term \theterm\ Week \theweek\ Handout Solutions}

\title{{\thepdftitle}}
\author{Nathan Wong\and Tom Yan}
\date{2023}

\newcommand{\leg}[2]{\left(\frac{#1}{#2}\right)}
\newcommand{\ileg}[2]{(#1|#2)}

%\newcommand{\marginfn}[1]{\marginpar{\footnotemark}\footnotetext{#1}}
\newcommand{\marginnote}[1]{\marginpar{\footnotesize{#1}}}
\newcommand{\marginfnote}[1]{\footnotemark\marginpar{\footnotemark[\value{footnote}]\footnotesize{#1}}}
\usepackage{geometry}
%\geometry{a4paper,left=24.8mm,top=27.4mm,headsep=2\baselineskip,textwidth=107mm,marginparsep=8.2mm,marginparwidth=49.4mm,textheight=49\baselineskip,headheight=\baselineskip}
\geometry{a4paper,left=1in,top=1in,bottom=1in,headsep=2\baselineskip,textwidth=107mm,marginparsep=8.2mm,marginparwidth=49.4mm,textheight=49\baselineskip,headheight=\baselineskip}
\usepackage[bf,tiny]{titlesec}
%\usepackage{fancyhdr}
\usepackage{epigraph}
%\usepackage[indent=0pt,skip=10pt]{parskip}

\usepackage{amsmath}
\usepackage{amsthm}
\newtheorem{theorem}{Theorem}
\usepackage{amssymb}
\let\mathbbalt\mathbb

\usepackage{fontspec}
\usepackage{unicode-math}
\let\mathbb\mathbbalt

\newcommand{\naturals}{\mathbb{N}}
\newcommand{\reals}{\mathbb{R}}
\newcommand{\rationals}{\mathbb{Q}}
\newcommand{\integers}{\mathbb{Z}}

\usepackage[pdfusetitle]{hyperref}

\newcommand{\myquote}[2]{%
  \begin{quote}
    \emph{#1}
    \begin{flushright}---{#2}
    \end{flushright}
  \end{quote}}
\pagestyle{empty}
\begin{document}
\noindent Melbourne High School\\\
\noindent Maths Extension Group 2023\\\
\noindent \textbf{\thedisplaytitle}\\\
\section*{Problems: functions and functional equations}
\begin{enumerate}
\item For $f$ to be invertible, we need $g(f(x))=x$ for all $x$ in the domain of $f$. But $g(f(x))=\sqrt{x}=|x|$, which is not equal to $x$ when $x$ is negative. Also, $f$ is not injective because $f(a)=f(-a)$ and so the inverse does not exist.
\item Let $y=0$, then we have $f(x)=x+f(0)=x+2$. Hence $f(1998)=1998+2=2000$.
\item To prove injectivity, if $f(x)=f(y)$ then clearly $f(f(x))=f(f(y))$, and hence $x=y$ from the definition of an involution. 

 To prove surjectivity, take $z$ to be a real number in the co-domain of $f$. It suffices to show there exists $x$ such that $f(x)=z$. From $f(x)=z$ we have $f(f(x))=f(z)$ or $f(z)=x$. Thus, the $x$ we needed to show that exists is $f(z)$.
\item Subbing in $y=0$ yields $f(x)=x^2$. Thus $f(x)=x^2$ is the only possible solution. Trying this solution in the original equation, we have $f(x+y)+f(x-y)=2x^2+2y^2$, not $2x^2-2y^2$. Thus the only possible candidate fails the test, hence there are no solutions.
\item Subbing in $y=0$ we get $$-2f(0)=-2$$ $$f(0)=1$$ Subbing in $x=0$ we obtain $$f(y)+2f(-y)=3-y$$ and by replacing $y$ with $-y$ we obtain $$f(-y)+2f(y)=3+y$$ Solving the system of equations, treating $f(y)$ and $f(-y)$ as variables, we get $f(y)=y+1$. It is easy to verify $f(x)=x+1$ satisfies the conditions of the problem, thus it is the only solution. 
\end{enumerate}
\end{document}

