\documentclass[a4paper]{article}

\newcommand{\theterm}{3}
\newcommand{\theweek}{8}
\newcommand{\thepdftitle}{MEG 2023 Term \theterm\ Week \theweek\ Handout}
\newcommand{\thedisplaytitle}{Term \theterm\ Week \theweek\ Handout}

\title{{\thepdftitle}}
\author{Nathan Wong\and Tom Yan}
\date{2023}

\newcommand{\leg}[2]{\left(\frac{#1}{#2}\right)}
\newcommand{\ileg}[2]{(#1|#2)}
\newcommand{\floor}[1]{\lfloor#1\rfloor}
\newcommand{\bfloor}[1]{\left\lfloor#1\right\rfloor}

%\newcommand{\marginfn}[1]{\marginpar{\footnotemark}\footnotetext{#1}}
\newcommand{\marginnote}[1]{\marginpar{\footnotesize{#1}}}
\newcommand{\marginfnote}[1]{\footnotemark\marginpar{\footnotemark[\value{footnote}]\footnotesize{#1}}}
\usepackage{geometry}
%\geometry{a4paper,left=24.8mm,top=27.4mm,headsep=2\baselineskip,textwidth=107mm,marginparsep=8.2mm,marginparwidth=49.4mm,textheight=49\baselineskip,headheight=\baselineskip}
\geometry{a4paper,left=1in,top=1in,bottom=1in,headsep=2\baselineskip,textwidth=107mm,marginparsep=8.2mm,marginparwidth=49.4mm,textheight=49\baselineskip,headheight=\baselineskip}
\usepackage[bf,tiny]{titlesec}
%\usepackage{fancyhdr}
\usepackage{epigraph}
%\usepackage[indent=0pt,skip=10pt]{parskip}

\usepackage{diagbox}
\usepackage{amsmath}
\usepackage{amsthm}
\newtheorem{theorem}{Theorem}
\usepackage{amssymb}
\let\mathbbalt\mathbb

\usepackage{fontspec}
\usepackage{unicode-math}
\let\mathbb\mathbbalt

\newcommand{\naturals}{\mathbb{N}}
\newcommand{\reals}{\mathbb{R}}
\newcommand{\rationals}{\mathbb{Q}}
\newcommand{\integers}{\mathbb{Z}}

\usepackage[pdfusetitle]{hyperref}

\newcommand{\myquote}[2]{%
  \begin{quote}
    \emph{#1}
    \begin{flushright}---{#2}
    \end{flushright}
  \end{quote}}
\pagestyle{empty}
\begin{document}
\noindent Melbourne High School\\\
\noindent Maths Extension Group 2023\\\
\noindent \textbf{\thedisplaytitle}\\\
\myquote{Boiling water will\marginnote{Don't ask how this is related to functional equations.} soften a potato but
harden an egg.}{J.~Clear, \emph{Atomic Habits} (2018)}
\section*{Problems: functions and functional equations}
\begin{enumerate}
\item The function $f(x)=x^2$ is not invertible. However, if we let $g(x)=\sqrt{x}$ then $f(g(x))=(\sqrt{x})^2=x$. So, why isn't $f$ invertible?   
\item  Let \marginnote{AHSME 1998/17} $f(x)$ be a function such that $f(x+y)=x+f(y)$ for any two real numbers $x$ and $y$, and $f(0)=2$. What is the value of $f(1998)$?
\item An involution is a function whose inverse is itself ($f(f(x)=x$ for all $x$ in the domain of $f$). Show that an  involution must be bijective. 
\item Find all solutions to the equation $$f(x+y)+f(x-y)=2x^2-2y^2.$$ 
\item Find all functions $f: \mathbb{R} \rightarrow \mathbb{R} $ for which $$f(x+y)-2f(x-y)+f(x)-2f(y)=y-2$$ for all $x,y \in \mathbb{R}$. 
\end{enumerate}
\pagebreak
\myquote{ As a six-year-old lad I could
easier understand a mathematical proof than I could understand why I should take my cap off in the\marginnote{Soon Mr.~Eisenstein is to play an important role in our story.}
parlour, or why I should use a knife to cut a slice of meat rather than tearing it apart with a fork.}{G.~Eisenstein\footnote{Translated from the German by M.~Schmitz in \emph{Res.~Lett.~Inf.~Math.~Set.,} 2004, Vol.~6, pp 1--13.}, \emph{Curriculum Vitae} (c.~1843)}
\section*{Cliffhanger}
Let's begin by revisiting the problem of finding
\(\ileg{2}{p}\) for odd primes \(p\). 
We shall not leave ourselves
on a cliffhanger this time, but our method
of solving the problem will be different.
The method employed here will make our transition
to the most general case (\(\ileg{p}{q}\) for
odd primes \(p\) and \(q\)) easier and perhaps
more enlightening.

The basis of our proof, like before,
is Gauss's Lemma. 
Such a venerable lemma is easy to forget, so
here it is again: given an odd prime \(p\)
and some number \(a\) coprime with \(p\),
\[\leg{a}{p}=(-1)^v\]
where \(v\) is the number of numbers in the set
\begin{equation}\label{eq:gausslem1}
		\left\{a, 2a, 3a,\ldots, \frac{p-1}{2}a\right\}
\end{equation}
that have
least positive residues greater than \(p/2\).
An equivalent definition of \(v\) is the number
of negative numbers in the set formed by
taking each element of \eqref{eq:gausslem1} and
reducing it to be between \(-p/2\) and \(p/2\),
for if \(ka>p/2\) then reducing it in this manner
will make it negative.

Armed with this weapon, let's introduce another.
This one is called the \emph{floor function}; it
is denoted by \(\floor{x}\).
This symbol means the largest integer less than
or equal to \(x\).
So if \(x\) is a positive number, \(\floor{x}\)
is the integer part of \(x\).
\marginnote{What happens when \(x\) is negative?}
For example, \(\floor{\pi}=3\), \(\floor{e}=2\), \(\floor{28/5}=5\).
Using this notation we can write any number \(x\) as \[x=\floor x+y\]
where \(0\le y<1\).

Now we can turn back to the original problem.
For our specific case where \(a=2\), we have
to find how many numbers in the set 
\[ \{2,4,\ldots,p-1 \}\] are greater than \(p/2\).
Let \(u\) be the number of numbers in the set that are less
than \(p/2\) and \(v\) be the number of numbers
in the set that are greater than \(p/2\).
In other words, \(u\) satisfies\marginnote{Clearly only
strict inequality signs are needed because \(p/2\) is not an integer.}
\[2u<\frac{p}{2}\]
and 
\[2(u+1)>\frac{p}{2}.\]
Rearranging these two inequalities yields
\[u<\frac{p}{4}\] and \[u>\frac{p}{4}-1,\]
which is more succinctly written as
\[ \frac{p}{4}-1<u<\frac{p}{4}.\]
A little thought convinces us that between
\(p/4\) and \(p/4-1\) there is one and only
one integer, and that integer is \(\floor{p/4}\).
Thus \(u=\floor{p/4}\).

By definition,\marginnote{In the following section
some of the calculations have been omitted for brevity;
carry them out yourself if necessary to understand the logic.}
\(u+v=(p-1)/2\), so \(v=(p-1)/2-\floor{p/4}.\)
To get rid of the floor function so that we can
collect like terms, consider two cases:
when \(p\equiv1\pmod{4}\) and when \(p\equiv3\pmod{4}\).
If \(p=4k+1\), then \(\floor{p/4}=\floor{k+1/4}=k\); substituting
this into the above equation for \(v\) we get \(v=k\).
But \(k=(p-1)/4\), so we have \[v=\frac{p-1}{4}.\]
Similarly, if \(p=4k+3\), then \(\floor{p/4}=k\) and \(v=k+1\).
In this case \(k=(p-3)/4\) so \[v=\frac{p+1}{4}.\]

Here is the final hurdle. As shown above, when \(p=4k+1\)
we have \(v=(p-1)/4\). If \(p-1\) is divisible by \(4\),
then \(p-1+2=p+1\) is not divisible by \(4\), but has the same
quotient. That is,
\[\frac{p-1}{4}=\bfloor{\frac{p+1}{4}}\] when \(p=4k+1.\)
If \(p=4k+3\), we have \(v=(p+1)/4\) anyway, and clearly
\((p+1)/4=\floor{(p+1)/4}\). Therefore we have shown that
\[v=\bfloor{\frac{p+1}{4}}\]
regardless of whether \(p\) is \(1\) or \(3\) more than
a multiple of \(4\).

From Gauss's Lemma \[\leg{2}{p}=(-1)^v;\] substituting
in the value of \(v\) just found gives 
\begin{equation}\label{eq:quad2v}
	\leg{2}{p}=(-1)^{\bfloor{\frac{p+1}{4}}}.
\end{equation}

Here is where we need numerical evidence to help
\marginnote{If you successfully solved last week's
problems you will know what comes next.}
us continue; we would be lost without a rough idea
of what we're trying to prove because it
is not clear how to proceed. \marginnote{}
In this case the best way is to evaluate \(\ileg{2}{p}\)
for the first hundred primes to see if a pattern
is apparent.
If we do this, we find that \(\ileg{2}{p}=1\) for
\[p=7,17,23,31,41,47,71,73,79,89,97\] and \(\ileg{2}{p}=-1\) for
\[p=3,5,11,13,19,29,37,43,53,59,61,67,83.\]
What is the difference between the two lists?
If we look closely, we observe that all the numbers in
\marginnote{In this case the pattern is so overwhelming
as to make the conjecture obvious once it is noticed,
but in other cases, such as the one at the end of this
handout, many different guesses might have to made before
the correct one is landed upon and successfully proved.}
the first list are congruent to \(\pm1\) modulo \(8\)
while the ones in the second are congruent to \(\pm3\)
modulo \(8\).
Our conjecture, therefore, is that \(\ileg{2}{p}=1\)
if \(p\equiv\pm1\pmod{8}\) and \(\ileg{2}{p}=-1\)
if \(p\equiv\pm3\pmod{8}\).

To prove the conjecture, first put \(p=8k+1\) into Equation
\eqref{eq:quad2v}. We get 
\[\leg{2}{p}=(-1)^{\floor{2k+1/2}}=(-1)^{2k}=1.\]
Similarly if \(p=8k-1\) then \(\floor{(p+1)/4}=2k\),
which implies \(\ileg{2}{p}=(-1)^{2k}=1\) again.

Now for the other case. If \(p=8k+3\) then 
\[\leg{2}{p}=(-1)^{\floor{2k+1}}=(-1)^{2k+1}=-1.\]
Putting \(p=8k-3\) instead yields 
\[\leg{2}{p}=(-1)^{\floor{2k-1/2}}=(-1)^{2k-1}=-1.\]
This proves our conjecture! Indeed it is true
that the quadratic character of \(2\) modulo \(p\)
depends on the the residue class of \(p\) modulo \(8\).

Let's summarise what we have discovered so
far in the quest to evaluate the Legendre symbol.
We have solved two cases: \(a=-1\) and \(a=2\).
In the former case \(\ileg{-1}{p}=-1\) if \(p\equiv1\pmod{4}\)
and \(\ileg{-1}{p}=-1\) if \(p\equiv-1\pmod{4}\);
in the latter case \(\ileg{2}{p}=1\) if \(p\equiv\pm1\pmod{8}\)
and \(\ileg{2}{p}=-1\) if \(p\equiv\pm3\pmod{8}\).

Although it may not seem like it, we are in fact
two thirds \marginnote{Ok, maybe two thirds is an exaggeration,
because the part left for us to discover (and prove) is much more difficult
than what we have just done. But it's still good
to be optimistic.}
of the way towards evaluating any Legendre symbol.
Suppose we want to find \(\ileg{a}{p}\). 
Using the multiplicative property \(\ileg{ab}{p}=\ileg{a}{p}\ileg{b}{p}\),
if we write \(a=q_1q_2\cdots q_r\), where all the \(q\)'s
are primes, then 
\[\leg{a}{p}=\leg{q_1}{p}\leg{q_2}{p}\cdots\leg{q_r}{p}.\]
If \(a<0\) then stick at \(\ileg{-1}{p}\) at the front---we know
how to evaluate that.
So the only goal now is to evaluate \(\ileg{q}{p}\) when \(q\)
is an odd prime.

This question is much more difficult to answer
than the two we have already tackled.
To get started, here is a table values of \(p\), \(q\), and \(\ileg{p}{q}\).
\begin{center}
	\begin{tabular}{|c||c|c|c|c|c|c|c|c|c|c|c|c|}
		\hline
		\diagbox{\(p\)}{\(q\)}& \(3\)&\(5\)&\(7\)&\(11\)&\(13\)&\(17\)&\(19\)&\(23\)&\(29\)&\(31\)&\(37\)\\
		\hline\hline
		\(3\)& \(0\)&	\(-1\)&	\(1\)&	\(-1\)&	\(1\)&	\(-1\)&	\(1\)&	\(-1\)&	\(-1\)&	\(1\)&	\(1\)\\ \hline
		\(5\)&\(-1\)&	\(0\)&	\(-1\)&	\(1\)&	\(-1\)&	\(-1\)&	\(1\)&	\(-1\)&	\(1\)&	\(1\)&	\(-1\)\\ \hline
		\(7\)&\(-1\)&	\(-1\)&	\(0\)&	\(1\)&	\(-1\)&	\(-1\)&	\(-1\)&	\(1\)&	\(1\)&	\(-1\)&	\(1\)\\ \hline
		\(11\)&\(1\)&	\(1\)&	\(-1\)&	\(0\)&	\(-1\)&	\(-1\)&	\(-1\)&	\(1\)&	\(-1\)&	\(1\)&	\(1\)\\ \hline
		\(13\)&\(1\)&	\(-1\)&	\(-1\)&	\(-1\)&	\(0\)&	\(1\)&	\(-1\)&	\(1\)&	\(1\)&	\(-1\)&	\(-1\)\\ \hline
		\(17\)&\(-1\)&	\(-1\)&	\(-1\)&	\(-1\)&	\(1\)&	\(0\)&	\(1\)&	\(-1\)&	\(-1\)&	\(-1\)&	\(-1\)\\ \hline
		\(19\)&\(-1\)&	\(1\)&	\(1\)&	\(1\)&	\(-1\)&	\(1\)&	\(0\)&	\(1\)&	\(-1\)&	\(-1\)&	\(-1\)\\ \hline
		\(23\)&\(1\)&	\(-1\)&	\(-1\)&	\(-1\)&	\(1\)&	\(-1\)&	\(-1\)&	\(0\)&	\(1\)&	\(1\)&	\(-1\)\\ \hline
		\(29\)&\(-1\)&	\(1\)&	\(1\)&	\(-1\)&	\(1\)&	\(-1\)&	\(-1\)&	\(1\)&	\(0\)&	\(-1\)&	\(-1\)\\ \hline
		\(31\)&\(-1\)&	\(1\)&	\(1\)&	\(-1\)&	\(-1\)&	\(-1\)&	\(1\)&	\(-1\)&	\(-1\)&	\(0\)&	\(-1\)\\ \hline
		\(37\)&\(1\)&	\(-1\)&	\(1\)&	\(1\)&	\(-1\)&	\(-1\)&	\(-1\)&	\(-1\)&	\(-1\)&	\(-1\)&	\(0\)\\ \hline
	\end{tabular}
\end{center}
In the table there exists a pattern that is at the heart
of what may be the most beautiful theorem in number theory.
\marginnote{Admittedly I am biased.}
But for now, the story remains to be continued.
\section*{Problem}
Spot as many patterns as possible in the above table.
\end{document}
