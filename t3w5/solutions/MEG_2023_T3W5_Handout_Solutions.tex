\documentclass[a4paper]{article}

\newcommand{\theterm}{3}
\newcommand{\theweek}{5}
\newcommand{\thepdftitle}{MEG 2023 Term \theterm\ Week \theweek\ Handout Solutions}
\newcommand{\thedisplaytitle}{Term \theterm\ Week \theweek\ Handout Solutions}

\title{{\thepdftitle}}
\author{Nathan Wong\and Tom Yan}
\date{2023}

\newcommand{\leg}[2]{\left(\frac{#1}{#2}\right)}
\newcommand{\ileg}[2]{(#1|#2)}

%\newcommand{\marginfn}[1]{\marginpar{\footnotemark}\footnotetext{#1}}
\newcommand{\marginnote}[1]{\marginpar{\footnotesize{#1}}}
\newcommand{\marginfnote}[1]{\footnotemark\marginpar{\footnotemark[\value{footnote}]\footnotesize{#1}}}
\usepackage{geometry}
%\geometry{a4paper,left=24.8mm,top=27.4mm,headsep=2\baselineskip,textwidth=107mm,marginparsep=8.2mm,marginparwidth=49.4mm,textheight=49\baselineskip,headheight=\baselineskip}
\geometry{a4paper,left=1in,top=1in,bottom=1in,headsep=2\baselineskip,textwidth=107mm,marginparsep=8.2mm,marginparwidth=49.4mm,textheight=49\baselineskip,headheight=\baselineskip}
\usepackage[bf,tiny]{titlesec}
%\usepackage{fancyhdr}
\usepackage{epigraph}
%\usepackage[indent=0pt,skip=10pt]{parskip}

\usepackage{amsmath}
\usepackage{amsthm}
\newtheorem{theorem}{Theorem}
\usepackage{amssymb}
\let\mathbbalt\mathbb

\usepackage{fontspec}
\usepackage{unicode-math}
\let\mathbb\mathbbalt

\newcommand{\naturals}{\mathbb{N}}
\newcommand{\reals}{\mathbb{R}}
\newcommand{\rationals}{\mathbb{Q}}
\newcommand{\integers}{\mathbb{Z}}

\usepackage[pdfusetitle]{hyperref}

\newcommand{\myquote}[2]{%
  \begin{quote}
    \emph{#1}
    \begin{flushright}---{#2}
    \end{flushright}
  \end{quote}}
\pagestyle{empty}
\begin{document}
\noindent Melbourne High School\\\
\noindent Maths Extension Group 2023\\\
\noindent \textbf{\thedisplaytitle}\\\
\section*{Euler's Criterion}
\begin{enumerate}
\item We observe that 
\[\leg{p-a}{p}=\leg{-a}{p}=\leg{-1}{p}\leg{a}{p}\]
since \(p-a\equiv-a\pmod{p}\).
If \(p\equiv1\pmod{4}\) then \(\ileg{-1}{p}=1\), and so
\[\leg{p-a}{p}=\leg{a}{p}.\]
If \(p\equiv3\pmod{4}\) then \(\ileg{-1}{p}=-1\), and so
\[\leg{p-a}{p}=-\leg{a}{p}.\]
This completes the proof.
\item Let \(P=(p-1)/2\) and 
\(\alpha\) be the index of \(a\) for some primitive root
of \(p\) that we shall call \(g\).
From Fermat's Little Theorem we know that either
\[a^P\equiv1\pmod{p}\]
or \[a^P\equiv-1\pmod{p};\] the goal here is to prove
that the distinction between the two cases is exactly
the distinction between \(\ileg{a}{p}=1\) and \(\ileg{a}{p}=-1\).

 First consider the case where \(\ileg{a}{p}=1\).
Then \(a^P\equiv g^{\alpha P}\pmod{p}\).
Since \(\ileg{a}{p}=1\), it follows that \(\alpha\)
is even and therefore that \(\alpha P\) is a multiple
of \(p-1\). We know that \(g\) raised to any multiple
of \(p-1\) is unity, so 
\[a^P\equiv1\equiv\leg{a}{p}\pmod{p}.\]
If \(\ileg{a}{p}=-1\), then \(\alpha\) is odd, and so
\(\alpha P\) cannot be a multiple of \(p-1\). Since \(g\)
is a primitive root, the only powers of \(g\) that are congruent
to unity are the multiples of \(p-1\), and so
\(g^{\alpha P}\not\equiv1\); hence 
\[a^P\equiv g^{\alpha P}\equiv-1\equiv\leg{a}{p}\pmod{p}.\]
This completes the proof of Euler's Criterion.
\item Since \(p=4k+1\), we have \(p-1=4k\). By Wilson's Theorem
\[(p-1)!\equiv (4k)!\equiv-1\pmod{p}.\]
Notice that \(4k\equiv-1\), \(4k-1\equiv-2\), \(4k-2\equiv-3\), and so
on, all the way down to \(2k+1\equiv-2k\), so we actually have
\[ 1\times(-1)\times2\times(-2)\times\cdots\times 2k\times(-2k)\equiv-1\pmod{p}\]
which becomes
\[(2k)!^2\equiv-1\pmod{p}.\]
The solution to the congruence is therefore \(x\equiv\pm(2k)!\pmod{p}.\)
\end{enumerate}
\end{document}

