\section*{Section B}
\sol
Substituting \(a=st\) and
\(b=(s^2-t^2)/2\)
into the LHS of \(a^2+b^2=c^2\)
yields
\[
\begin{split}
s^2t^2+\frac{s^4-2s^2t^2+t^4}{4}&=
\frac{4s^2t^2+s^4-2s^2t^2+t^4}{4}\\
&=\frac{s^4+2s^2t^2+t^4}{4}\\
&=\left(\frac{s^2+t^2}{2}\right)^2
\end{split}
\]

The second part of the proof is
to show that the three numbers do
indeed share no common factors.
Note that this proof requires knowledge
of the Euclid's lemma, which asserts
that if a prime divides a product of
two numbers, it must divide at least one
of the two numbers. This will be covered
in future handouts.

Using Euclid's lemma, We
note that if any pair of \(a\), \(b\),
and \(c\) shared a common prime factor, 
the third would share that same common factor.
This is because if a prime divides \(n^2=n\cdot n\), it
divides either \(n\) or \(n\); hence it must divide \(n\).

Now suppose that \(p\) is a common prime
divisor of \(a=st\), \(b=(s^2-t^2)/2\), and \(c=(s^2+t^2)/2\).
From Euclid's lemma \(p\) divides either \(s\) or \(t\), but not both,
because \(s\) and \(t\) share no common factors; for now,
let's assume that it divides \(s\). Then we can write \(s=pu\)
for some integer \(u\). Substituting this into \(b\) and we get
\(b=(p^2u^2-t^2)/2\). Since, by assumption, \(p\) divides \(b\),
we have \(b=pv\) for some integer \(v\). Combining these two statements yields
\[
p^2u^2-t^2=2pv.
\]
Rearrange for \(t^2\) and we get \(t^2=p^2u^2-2pv\). The RHS is clearly
divisible by \(p\); hence \(t^2\) must be divisible by \(p\), which
implies \(t\) is divisible by \(p\). This is a contradiction, because
\(s\) and \(t\) share no common factors.

Now we turn to the case when \(p\) divides \(t\) instead of \(s\). As
before, we can write \(t=pu\) for some integer \(u\), and \(b=pv\)
for some integer \(v\). Again using these statements, we find that
\[
s^2-p^2u^2=pv.
\]
Rearrange for \(s^2\) and deduce that \(s\) is divisible by \(p\), once
again contradicting the assumption that \(s\) and \(t\) share no common factors.

Thus \(a\) and \(b\) cannot share any common factors; this completes
the proof that the triple \((st, (s^2-t^2)/2, (s^2+t^2)/2)\) for
coprime odd integers \(s\) and \(t\), where \(s>t\), always generates
a primitive Pythagorean triple.

\sol
Since \(s\) and \(t\) are odd,
let \(s=2m+1\) and \(t=2n+1\)
for integers \(m\) and \(n\).
We have chosen \(a\) to be odd
and \(b\) even, and so substituting
\(s\) and \(t\) into \(b=(s^2-t^2)/2\),
we get
\[
\begin{split}
b&=\frac{(2m+1)^2-(2n+1)^2}{2}\\
&=\frac{4m^2+4m+1-4n^2-4n-1}{2}\\
&=2m^2+2m-2n^2-2n\\
&=2m(m+1)-2n(n+1)
\end{split}
\]
Note that \(m(m+1)\) and \(n(n+1)\)
must be even; hence \(4\) divides both \(2m(m+1)\)
and \(2n(n+1)\).
This completes the proof that \(b\) is
divisible by \(4\).
\sol
We first note that a square can only
leave a remainder of \(0\) or \(1\)
when divided by \(3\). We can see this
just by calculating the squares of the
numbers \(0\), \(1\), and \(2\):
\[
\begin{split}
0^2&\equiv0\pmod{3}\\
1^2&\equiv1\pmod{3}\\
2^2&\equiv1\pmod{3}
\end{split}
\]

Assume that neither \(a\) or \(b\)
is divisible by \(3\). Then both
\(a^2\) and \(b^2\) must be congruent to
\(1\) mod \(3\). Then \(c^2\equiv a^2+b^2
\equiv2\pmod{3}\). We have just shown
that this is impossible; a square cannot
leave a remainder of \(2\) when divided \(3\).
Hence either \(a\) or \(b\) must be divisible
by \(3\).

If not familiar with modular arithmetic or
its notation, this can also be proven by
expanding expression such as \((3m+1)^2\) to
determine that the square of a number \(1\) more than
a multiple of \(3\) is also \(1\) more than
a multiple of \(3\).
\sol
We begin, as before, by nothing that
the only squares modulo \(5\) are \(0\), \(1\),
and \(4\).

Assume that none of \(a\), \(b\), or \(c\)
is divisible by \(5\).
Then \(a^2\) and \(b^2\) are both congruent
to either \(1\) or \(4\).
Enumerating all the possibilities we get
one of the following possibilities for \(c^2\):
\[
\begin{split}
c^2&\equiv1+1\equiv2\pmod{5}\\
c^2&\equiv1+4\equiv0\pmod{5}\\
\end{split}
\]
We know the first possibility is impossible,
so \(c^2\equiv0\pmod{5}\), which contradicts
our initial assumption.

Hence one of \(a\), \(b\), or \(c\)
must be divisible by \(5\).

We have shown that each \(3\), \(4\), and \(5\)
will always divide one of the numbers in
a Pythagorean triple. So every the product of Pythagorean triple must be divisible by \(3\cdot4\cdot5=60\). However, the highest common factor of the product of  all
Pythagorean triples must also be a factor of
the product of the Pythagorean triple \((3,4,5)\); 
hence the HCF we are looking for is \(60\).

\sol
The equation of the live is \(y=m(x+1)\).
Substituting this into the equation of the circle,
we get
\[
\begin{split}
x^2+m^2(x^2+2x+1)&=1\\
(1+m^2)x^2+m^22x+m^2-1&=0
\end{split}
\]
From here, the quadratic formula could be
used, but the better way is to recognise
that \(x=-1\) is a solution to the above
equation; therefore we can use polynomial
division to determine the other factor.
So dividing the quadratic by \((x+1)\), we
get
\[
\frac{(1+m^2)x^2+m^22x+m^2-1}{x+1}=
(1+m^2)x+(m^2-1)
\]
Solving this equation for \(x\) gives
\(x=(1-m^2)/(1+m^2)\). Substitute this
value of \(x\) into the equation \(y=m(x+1)\)
to find
\[
\begin{split}
y&=m\left(\frac{1-m^2+1+m^2}{1+m^2}\right)\\
&=m\left(\frac{2}{1+m^2}\right)\\
&=\frac{2m}{1+m^2}.
\end{split}
\]
Now we know that the other point of intersection
is given by
\[
(x,y)=\left(\frac{1-m^2}{1+m^2},\frac{2m}{1+m^2}\right)
\]
Since this point is on the circle given by the
equation \(x^2+y^2=1\), we have
\[
\left(\frac{1-m^2}{1+m^2}\right)^2+\left(\frac{2m}{1+m^2}\right)^2=1.
\]
Multiplying both sides by \((1+m^2)^2\),
\[
(1-m^2)^2+4m^2=(1+m^2)^2.
\]
Since \(m\) is a rational number, let \(m=v/u\),
where \(u\) and \(v\) are integers.
The above equation becomes
\[
\begin{split}
\left(1-\frac{v^2}{u^2}\right)^2+\frac{4v^2}{u^2}&=\left(1+\frac{v^2}{u^2}\right)^2\\
\left(\frac{u^2-v^2}{u^2}\right)^2+\frac{4v^2}{u^2}
&=\left(\frac{u^2+v^2}{u^2}\right)^2.
\end{split}
\]
Finally, multiply both sides by \(u^4\) and we
get
\[
(u^2-v^2)^2+4v^2u^2=(u^2+v^2)^2.
\]
This Pythagorean triple generating formula
\((a,b,c)=(u^2-v^2,2uv,u^2+v^2)\) doesn't
always generate primitive triples. As an extension,
come up with conditions on \(u\) and \(v\)
under which the triple generated is primitive.
\sol
Assume an arithmetic progression of three 
squares with a common difference that is also
a square exists. If we let \(a^2\), \(b^2\),
and \(c^2\) be the terms of the sequence and
\(d^2\) be the common difference, we have
\[
a^2+d^2=b^2
\]
and
\[ b^2+d^2=c^2.\]
This implies that \begin{equation}\label{eq:squares}
\begin{split}
b^4&=(a^2+d^2)(c^2-d^2)\\
&=a^2c^2-d^2(a^2-c^2)-d^4
\end{split}
\end{equation}
Since \(a^2=b^2-d^2\) and \(c^2=b^2+d^2\),
we have \(a^2-c^2=b^2-d^2-b^2-d^2=-2d^2\).
Substituting this into \eqref{eq:squares}
yields
\[
\begin{split}
b^4&=a^2c^2+2d^4-d^4\\
&=a^2c^2+d^4
\end{split}
\]
Let \(X=b\), \(Y=d\) and \(Z=ac\) and rearrange
to get the equation
\[
X^4-Y^4=Z^2.
\]
Since this equation has no nonzero integer solutions
(and you can try proving this if you are interested),
there cannot exist three squares in a arithmetic progression
that have a common difference of a square.
\sol
We could use the formula for primitive triples derived
earlier, but it is easier to use Euclid's formula as it
was originally written down:
\[
(a,b,c)=(2uv, u^2-v^2,u^2+v^2)
\]
where \(u\) and \(v\) are coprime integers of different parity.
The area of this triangle is \((1/2)ab=uv(u^2-v^2)\). Assume that
this value is a square. The factors
of the right-hand side are \(u\), \(v\), \(u+v\), and \(u-v\); since
\(u\) and \(v\) are coprime, each of those factors are coprime too.
Thus we conclude that each of the four factors must be squares themselves,
since their product is a square.

Let \(u+v=x^2\), \(u-v=y^2\), \(u=r^2\), and \(v=s^2\). Then
\[
\begin{split}
s^2+y^2=r^2
\end{split}
\]
and
\[
r^2+s^2=x^2.
\]
Hence \(y^2\), \(r^2\), and \(x^2\) form an arithmetic progression with
common difference \(s^2\). From the previous problem we know this is
impossible; hence a right-angled triangle with integer length sides
cannot have a square area.

You may notice that the question asked for \emph{rational} length sides.
Try to determine how the above proof is also (or is not) a proof that
a right-angled triangle with rational length sides cannot have a square area.
