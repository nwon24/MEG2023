\section*{Section B: The higher arithmetic}
Number theory, or the higher arithmetic, is the
branch of mathematics that studies the natural numbers.
It is a wondrous and beautiful subject, full of
amazing ideas that will make your study of it
worthwhile.\footnote{Your mileage may vary.}

Number theory investigates many different
types of natural numbers and the relationships
between them. Some are listed below.
  \begin{itemize}
  \item Prime numbers. They are the basic building
    blocks of all the natural numbers, and thus
    come up frequently in any arithmetic investigation.
    Many of number theory's famous solved and unsolved problems
    have to do with primes, including the Twin Prime Conjecture
    and the Prime Number Theorem.
  \item Shapely numbers. These are numbers like the familiar triangular and
    square numbers. Shapely numbers
    extend up to pentagonal numbers, hexagonal numbers, etc.
  \item Powers of numbers. The most famous theorem involving
    these numbers is \emph{Fermat's Last Theorem}, which asserts
    that a number raised to a power greater than \(2\) is not
    the sum of two like powers.
  \item Perfect numbers. These are the numbers that are
    the sum of their proper divisors (all of their divisors except for itself).
    For example, \(6\) is a perfect number; its proper divisors, which are \(1\), \(2\),
    and \(3\), have a sum of \(6\).
    
  \end{itemize}
  Although number theory is primarily the study of natural numbers, we often
  have to resort to other number systems, including the integers, rationals, reals,
  and sometimes even the complex numbers. For the most part, however, we will
  stick to the natural numbers, the integers, and sometimes the rationals.

  For all the interesting numbers of number theory, there are just as many
  interesting questions. Here are some about the primes; some have been solved
  while others remain unsolved.
  \begin{itemize}
	  \item Which primes are the sum of two squares?
	  \item The sequence \(3\), \(5\), \(7\) is a \emph{prime triplet}. Are
		  there infinitely many prime triplets?
	  \item Are there infinitely many primes of the form \(2^n-1\)?
	  \item Are there infinitely many primes of the form \(4n+1\)? \(4n+3\)?
	  \item Are there infinitely many primes of the form \(n^2+1\)?
	  \item For some given natural number \(x\), how many primes are there
		  less than or equal to \(x\)?
	  \item Is every even number greater than or equal to \(4\) the sum of two primes?
  \end{itemize}
  Number theory asks many questions of equations and polynomials too. Generally
  polynomials in number theory are considered to have integer or rational coefficients.
  For example, the equation \(x^2-Ny^2=1\) for a natural number \(N\) that is not a perfect
  square is known as \emph{Pell's Equation}. Amazingly, such an equation always
  has infinitely many solutions. What's more, those solutions are related to continued fractions!

  Even the simplest equations, linear equations, present difficulties because we often
  consider them in several variables. Consider the equation \(ax-by=n\) for natural numbers
  \(a\), \(b\), and \(n\). When does this equation have solutions in integers or natural numbers \(x\)
  and \(y\)?

  Then there are equations of a more elaborate kind, such as the equation \(y^2=x^3+17\). It's not at all
  obvious how to solve such an equation in the integers or rationals. In fact, it is still
  and unsolved problem in mathematics as to how to determine whether such equations (\(y^2\) equals
  a cubic in \(x\)) have finitely many rational solutions, infinitely many rational solutions, or no rational
  solutions at all.

  We shall consider all the above types of numbers and equations in our journey through number theory.
  \subsection*{Pythagorean triples}
  Let's begin our exploration into number theory by starting with something
  we already know: Pythagorean triples.

  A Pythagorean triple is a triple of positive integers \((a,b,c)\) where \(a^2+b^2=c^2\). Geometrically,
  \(c\) is the hypotenuse and \(a\) and \(b\) are the two shorter sides of a
  right-angled triangle. A natural question is whether there are infinitely
  many such triples. The answer is yes because if \((a,b,c)\) is a Pythagorean
  triple, then so is \((na,nb,nc)\) for any natural number \(n\). 
  The simple direct proof follows.
  \[
	  \begin{split}
		  (na)^2+(nb)^2&=n^2(a^2+b^2)\\
		  &=n^2c^2\\
		  &=(nc)^2
	  \end{split}
  \]

  Triples that are just multiples of other triples aren't very interesting, so
  we look mainly at those triples in which \(a\), \(b\), and \(c\) share no
  common factors. These Pythagorean triples are called \emph{primitive}. For
  example, \((3,4,5)\) is a primitive triple, while \((6,8,10)\) is not.

   \interlude{Write down as many primitive Pythagorean triples as you can and look
  for patterns.}

From our definition of primitive triples, we immediately note that one of \(a\)
and \(b\) has to be even and the other odd. If both \(a\) and \(b\) were even, then
\(c^2=a^2+b^2\) is even. This means \(c\) would also be even, contradicting the assumption that \(a\), \(b\), and \(c\)
share no common factors. What happens if both \(a\) and \(b\) are odd?
\interlude{Prove that both \(a\) and \(b\) cannot be odd.}

Since either \(a\) or \(b\) can be even and the other odd, from now on we can
assume, without loss of generality, that \(a\) is odd and \(b\) is even. Note
that this implies \(c\) must be odd too, because \(a^2+b^2\), being
the sum of an even and an odd number, is odd.

A way to generate primitive triples would be quite a useful thing. How
might we go about deriving a formula for \(a\), \(b\), and \(c\)?
First we note that \(a^2=c^2-b^2\), so \(a^2=(c+b)(c-b)\). The right-hand side,
\((c+b)(c-b)\), is a perfect square. Since \(c\) and \(b\) share no
common factors, neither do \(c+b\) and \(c-b\). Since they share
no common factors, both \(c+b\) and \(c-b\) have to be squares themselves.
\interlude{To see why this is true, write down a few primitive Pythagorean
  triples and check if both \(c+b\) and \(c-b\) are perfect squares.}

Since \(c\) is odd and \(b\) is even (by assumption), both \(c-b\) and \(c+b\) are odd.
Thus we can let \begin{equation}\label{eq:cminusb}c-b=t^2\end{equation} and \begin{equation}\label{eq:cplusb}c+b=s^2\end{equation} for odd integers \(t\) and \(s\).
Since \(c+b>c-b\), we have \(s>t\). We can make one further deduction about
\(t\) and \(s\); they must share no common factors, because \(c-b\) and \(c+b\)
share no common factors.

\interlude{Solve equations \eqref{eq:cminusb} and \eqref{eq:cplusb} to get \(b\) and \(c\) in
  terms of \(t\) and \(s\).}

If you done the above interlude correctly, you should have \(b=(s^2-t^2)/2\)
and \(c=(s^2+t^2)/2\).
Now we just need a formula for \(a\). From \(a^2=b^2-c^2\), we have
\[ \begin{split}a&=\sqrt{(b+c)(b-c)}\\&=\sqrt{s^2t^2}.\end{split}\] This implies \(a=st\).

At the end of this long voyage, we have discovered a formula for
generating primitive Pythagorean triples:
\[
  (a,b,c)=\left(st,\frac{s^2-t^2}{2},\frac{s^2+t^2}{2}\right)
\]
where \(s>t\ge1\) and \(s\) and \(t\) are odd integers that share no common factors.
This formula, with slight modifications, is attributed to Euclid.
\interlude{Generate as many primitive triples as you can using
  the above formula to make sure that it works.}

\section*{Problems}
\setcounter{questionno}{1}
\question{Show that the formula derived above is always a solution
to \(a^2+b^2=c^2\). If you can, prove that the formula always generates
a primitive triple by showing that \(st\), \((s^2-t^2)/2\), and \((s^2+t^2)/2\)
share no common factors.}
\question{What do you notice about the even number in primitive
Pythagorean triples? Prove that your conjecture is correct.}
\question{Show that one of \(a\) and \(b\) must be a multiple of \(3\).}
\question{Given a Pythagorean triple (not necessarily primitive) \((a,b,c)\), consider its \emph{product}
to be \(abc\). Prove that one of \(a\), \(b\), and \(c\) must be divisible by \(5\), and hence
determine the highest common factor of all products of all Pythagorean triples.}
\question{Consider a circle of radius \(1\) centred at the origin with
equation \(x^2+y^2=1\), and a line with a rational gradient \(m\)
that passes through the point \((-1,0)\). Find the coordinates
of the other point of intersection between the line and the
circle in terms of \(m\). Hence deduce a second formula to
generate Pythagorean triples (this one won't just generate primitive triples).}
\question{Given that the equation \(X^4-Y^4=Z^2\) has no
nonzero solution in integers \(X\), \(Y\), and \(Z\), show
that if three square numbers are in an arithmetic progression,
their difference cannot also be a square.}
\question{Hence, or otherwise, show that a right-angled triangle
with \emph{rational} side lengths cannot have a square area.}
