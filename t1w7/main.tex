\documentclass{article}
\usepackage{graphicx} % Required for inserting images
\usepackage[a4paper, total={6in, 8in}]{geometry}
\title{MEG Combi problems}
\author{Tom Yan}
\date{March 2023}

\begin{document}

\section*{Problems}
1. Find the number of ways can I rearrange the letters of : MATHSEXTENSIONGROUP \\\\\
2. $10$ points in the plane are given, with no 3 collinear. 4 distinct segments joining pairs of these points are chosen at random, all such segments being equally likely. Find the probability that some 3 of the segments form a triangle whose vertices are among the 10 given points.\\\\
3. Students sit at their desks in three rows of eight. Felix, the class pet, must be passed to each student exactly once, starting with Alex in one corner and finishing with Bryn in the opposite corner. Each student can pass only to the immediate neighbour left, right, in front or behind. How many different paths can Felix take from Alex to Bryn? (AMC Intermediate Q30) \\\\
4. Ali, Beth, Chen, Dom and Ella finish a race in alphabetical order: Ali in first place, then Beth, Chen and Ella. They decide to run another race and their placings all change. Two of the runners receive a placing higher than the week before, and the other three runners receive a placing lower than the week before. Given this information, in how many orders could the five runners have finished this second race? (AMC Senior Q30) \\\\
5. Each cell of an $m\times n$ board is filled with some nonnegative integer. Two numbers in the filling are said to be adjacent if their cells share a common side. (Note that two numbers in cells that share only a corner are not adjacent). The filling is called a garden if it satisfies the following two conditions: \\\\(i) The difference between any two adjacent numbers is either $0$ or $1$. \\
(ii) If a number is less than or equal to all of its adjacent numbers, then it is equal to $0$ . \\\\\ Determine the number of distinct gardens in terms of $m$ and $n$. 
\end{document}
