\documentclass[a4paper]{article}

\newcommand{\theterm}{2}
\newcommand{\theweek}{9}
\newcommand{\thepdftitle}{MEG 2023 Term \theterm\ Holiday Handout}
\newcommand{\thedisplaytitle}{Term \theterm\ Week \theweek\ Handout}

\title{{\thepdftitle}}
\author{Nathan Wong\and Tom Yan}
\date{2023}

% \newcommand{\marginfn}[1]{\marginpar{\footnotemark}\footnotetext{#1}}
\newcommand{\marginnote}[1]{\marginpar{\footnotesize{#1}}}
\newcommand{\marginfnote}[1]{\footnotemark\marginpar{\footnotemark[\value{footnote}]\footnotesize{#1}}}
\usepackage{geometry}
% \geometry{a4paper,left=24.8mm,top=27.4mm,headsep=2\baselineskip,textwidth=107mm,marginparsep=8.2mm,marginparwidth=49.4mm,textheight=49\baselineskip,headheight=\baselineskip}
\geometry{a4paper,left=1in,top=1in,bottom=1in,headsep=2\baselineskip,textwidth=107mm,marginparsep=8.2mm,marginparwidth=49.4mm,textheight=49\baselineskip,headheight=\baselineskip}
\usepackage[bf,tiny]{titlesec}
% \usepackage{fancyhdr}
\usepackage{epigraph}
% \usepackage[indent=0pt,skip=10pt]{parskip}

\usepackage{amsmath}
\usepackage{amsthm}
\newtheorem{theorem}{Theorem}
\usepackage{amssymb}
\let\mathbbalt\mathbb

\usepackage{fontspec}
\usepackage{unicode-math}
\let\mathbb\mathbbalt

\newcommand{\naturals}{\mathbb{N}}
\newcommand{\reals}{\mathbb{R}}
\newcommand{\rationals}{\mathbb{Q}}
\newcommand{\integers}{\mathbb{Z}}

\usepackage[pdfusetitle]{hyperref}

\newcommand{\myquote}[2]{%
  \begin{quote}
    \emph{#1}
    \begin{flushright}---{#2}
    \end{flushright}
  \end{quote}}
\pagestyle{empty}
\begin{document}
\noindent Melbourne High School\\
\noindent Maths Extension Group 2023\\
\noindent \textbf{Term 2 Holiday Handout}\\

\myquote{If all the year were playing holidays,\\
  To sport would be as tedious as to work.}{W. Shakespeare, \emph{King Henry IV, Part 1} (1596)}
\section*{Term 2 Holiday Problems}
\begin{enumerate}
	\item Our friend Harold is back and is now reading Franz Kafka's \emph{The Trial}, which has \(167\) pages.
		\marginnote{It turns out that \(167\) is the smallest prime that is not the sum of \(7\) or fewer cubes.} Taking the title literally, he decides to
		read it in a new way. First he picks a random number between \(1\) and \(166\) inclusive, say \(x\). To find the \(n\)th page that
		he reads, where \(n=1,2,\ldots\), he multiplies \(x\) by \(n^2\), and, if the resulting page number is greater than \(167\),  subtracts off a multiple of \(167\) to get a page that is actually in the book. 
		He continues reading until he hits a page that he has already read before. 

		For what values of \(x\) will Harold be able to read the entire book this way? \marginnote{Reading below on primitive roots won't hurt in solving this problem, although it is not necessary.} If there are no such values of \(x\), how many pages will he read before he stops? Note that \(167\) is a prime number.

\item  Find \marginnote{Tournament of Towns Junior 1981/1} all integer solutions to the equation $$y^k=x^2+x$$ where $k$ is a natural number greater than 1.  
\item  Write $\left(1+\frac{1}{3}\right)\left(1+\frac{1}{3^2}\right)\left(1+\frac{1}{3^{2^2}}\right)\ldots\left(1+\frac{1}{3^{2^{100}}}\right)$ in the form $a(1-b^c)$, where $a$, $b$ and $c$ are constants.  \
\item  Find the number of ways to colour each square of a $2007 \times 2007$ square grid black or white such that each row and each column has an even number of black squares.\
\item  Let $ABC$ be an acute triangle. Let $BE$ and $CF$ be altitudes of $\triangle{ABC}$, and denote by $M$ the midpoint of $BC$. Prove that $ME$, $MF$, and the line through $A$ parallel to $BC$ are all tangents to circle $AEF$. \

\item If \(g\) is\marginnote{For this problem and the next, see below for a description of what primitive roots are.}
		a primitive root modulo \(p\), show that \[g^{(p-1)/2}\equiv -1\pmod{p}.\]
              \item Use primitive roots to prove Wilson's theorem for odd primes \(p\): \[(p-1)!\equiv-1\pmod{p}.\]
                
\end{enumerate}
\pagebreak
\section*{Primitive Roots}
In exploring Fermat's Little Theorem, we have seen what the order
of some number \(a\) to a prime modulus \(p\) is; it is the smallest
number \(l>0\) such that \[a^l\equiv1\pmod{p}.\] Furthermore, we discovered
that Fermat's Little Theorem is equivalent to saying that the order of
a number not divisible by \(p\) is always a divisor of \(p-1\). Now we shall
look at particular numbers that have orders exactly equal to \(p-1\); these
are called \emph{primitive roots}.

More explicitly, a primitive root of a prime \(p\) is a number \(g\) with
the property that the smallest power of it that is equal to unity is \(p-1\).
This means that all of the numbers \[g,g^2,g^3,\ldots,g^{p-1}\] are all distinct
and form the complete set of positive residues \[1,2,\ldots,p-1.\]
Before we examine why this is useful, we shall prove that every prime number
has at least one primitive root.

Euler was the first to state this theorem, but his proof was incorrect; the first correct proof belongs to Legendre.\marginnote{Adrien-Marie Legendre (1752--1833) was a French mathematician who, besides making great contributions to number theory, has only one known portrait. Funnily enough, this portrait is a caricature.}  The initial result we establish is the following: if
two numbers \(a\) and \(b\) have order \(l\) and \(k\) respectively and \(l\)
and \(k\) are coprime then the
number \(ab\) has order \(lk\). The proof of this preliminary result falls
into two parts; first we prove that \(ab\) raised to the power of \(lk\) is indeed
\(1\), and then that \(lk\) is the smallest number with this property. The first
part is easy, for \[(ab)^{lk}\equiv (a^l)^k(b^k)^l\equiv 1^k1^l\equiv1\pmod{p}.\]
The second part is more difficult. Suppose that the order of \(ab\) is \(l_1k_1\) and
let \(l_2\) and \(k_2\) be the numbers satisfying \(l=l_1l_2\) and \(k=k_1k_2\).
Then by definition, we have \[(ab)^{l_1k_1}\equiv 1\equiv a^{l_1k_1}b^{l_1k_1}\pmod{p}.\]
Raising both sides of this congruence to \(l_2\), the power of \(a\) vanishes because 
\(l_1l_2=l\) and \(a^l\equiv1\); therefore it leaves \[b^{lk_1}\equiv1\pmod{p}.\]
The order of \(b\) is \(k\), so \(k\) divides \(lk_1\). Since \(k\) and \(l\) are relatively
prime, \(k\) must divide \(k_1\). But \(k_1\) also divides \(k\), so the only way this
is possible is if \(k=k_1\). Similarly, if we had raised both sides of the above congruence
to \(k_2\), we would find that \(l\) must divide \(l_1\). Again, this implies that \(l_1=l\),
and therefore the order of \(ab\) is \(l_1k_1=lk\), as required.

The preceding simple result is the core of our proof that every prime has a primitive root. Next:
factorise \(p-1\) as a product of primes in \emph{canonical form}, as follows. \[p-1=q_1^{a_1}q_2^{a_2}\cdots q_r^{a_r}\] where all of \(q_1,q_2,\ldots,q_r\) are distinct primes and \(a_1,a_2,\ldots,a_r\) are positive
integers. From this it follows that all of \(q_1^{a_1},q_2^{a_2},\ldots,q_r^{a_r}\) are coprime. Hence
if we can show that there exists some number that has order \(q_i^{a_i}\), for \(1\le i\le r\),  then by repeated application of the preceding result there must be a number
that has order \(p-1\), which is the primitive root we are looking for.

Luckily, we have a previous result from Lagrange's Theorem on our side. Let \(q^a\) be one of the prime
powers that is in the prime factorisation of \(p-1\). In the previous handout \marginnote{The result was actually left as an exercise on the last handout,
  so if you didn't do it you must take it for granted for now.} we showed
that the congruence \[x^d-1\equiv0\pmod{p}\] has exactly \(d\) solutions. From this we deduce that there
are exactly \(q^{a}\) solutions to the congruence 
\begin{equation}\label{eq:xqa1}
	x^{q^{a}}\equiv1\pmod{p}
\end{equation}
but only \(q^{a-1}\)
to the congruence 
\begin{equation}\label{eq:xqa2}
	x^{q^{a-1}}\equiv1\pmod{p}.
\end{equation}
Let \(b\) be some number that satisfies Equation \eqref{eq:xqa1}. Then the order of \(b\) divides \(q^a\).
If the order of \(b\) is not \(q^a\) exactly, it must be one of the proper divisors of \(q^a\), namely one
of the numbers \[q,q^2,\ldots,q^{a-1}.\] If the order of \(b\) is one of these numbers, then \(b\) also satisfies
\eqref{eq:xqa2}. Therefore \(b\) has order \(q^a\) if and only if it satisfies Equation \eqref{eq:xqa1} but
not Equation \eqref{eq:xqa2}. There are \(q^a-q^{a-1}\) of such numbers, and hence there exists
at least one  number whose order is \(q^a\). With this we conclude the proof that every prime number
has a primitive root.\marginnote{You may notice that this proof provides a way to actually a construct
a primitive root using an algorithm. Try it!}

Why are primitive roots useful? One reason is that they provide a way to simplify multiplication
modulo a prime number using a device known as the \emph{discrete logarithm}. Firstly, if \(g\) is
a primitive root of a prime \(p\), then we have already seen that the numbers \[g,g^2,g^3,\ldots,g^{p-1}\] reduced modulo \(p\)
are the numbers 
\[1,2,\ldots,p-1\] in some order. For example, if \(p=5\) and \(g=2\), then the numbers \(2,2^2,2^3,2^4\)
reduced modulo \(5\) are \[2,4,3,1,\] which form a complete set of positive residues.

In other words, for each number \(a\) between \(1\) and \(p-1\), there exists some number \(\alpha\) also
between \(1\) and \(p-1\) such that \[g^\alpha\equiv a\pmod{p}.\] This number \(\alpha\) is called the \emph{index}
of \(a\) to the modulus \(p\) for the given primitive root \(g\) and may be denoted \(I(a)\). This function \(I(a)\) is
called the discrete logarithm because of its analogous behaviour to the normal logarithm.\marginnote{Try proving some properties
  of the discrete logarithm, such as \(I(ab)\equiv I(a)+I(b).\)}

One application of indices is in performing calculations with large numbers modulo \(p\). Suppose that we wish
to evaluate \(ab\) modulo \(p\), and that \(\alpha\) and \(\beta\) are the indices of \(a\) and \(b\) respectively.
Then \[ab\equiv g^\alpha g^\beta\equiv g^{\alpha+\beta}\pmod{p}.\]  In other words, to get the product \(ab\), simply
add the indices of \(a\) and \(b\), and the number corresponding to the desired index is the product we are looking for. For course, \(\alpha+\beta\)
may be greater than \(p-1\), but since the powers of a primitive root repeat every \(p-1\) terms, reducing \(\alpha+\beta\)
modulo \(p-1\) doesn't change which number the index corresponds to. As a result primitive roots and indices have
changed a difficult problem of multiplying two large numbers into a simple matter of adding two indices and reducing modulo \(p-1\).

The crucial question is: how do we know which number corresponds to a given index? The answer is that we can make a table
of indices relatively easily. Given a primitive root \(g\), we make a table of its powers modulo \(p\), where each successive
power is obtained just by multiplying the previous one by \(g\) and reducing modulo \(p\). For example, given \(g=3\) and \(p=7\),
the table would look like Figure \ref{fig:indices}.
\begin{figure}[h]
	\centering
	\begin{tabular}{|c|c|c|c|c|c|c|}
		\hline
		\(n\)&\(1\)&\(2\)&\(3\)&\(4\)&\(5\)&\(6\)\\
		\hline
		\(3^n\pmod{p}\)&\(3\)&\(2\)&\(6\)&\(4\)&\(5\)&\(1\)\\
		\hline
        \end{tabular}
        \caption{An example where \(3\) is a primitive root of \(7\)}
        \label{fig:indices}
\end{figure}
If necessary, we can just reverse the table to be able to look up the index of a particular number instead of looking
up which number corresponds to a particular index. Of course the example above is simple, but it illustrates that when
given much larger primes, it is comparatively easier to make a table of values and use indices in computations than doing
large multiplications by brute-force. In fact, a table of indices is similar to how logarithm tables were used before there 
were calculators. Tables of indices have been compiled by number theorists in the past, including Jacobi's
\marginnote{Carl Jacobi (1804--1851) was a German mathematician who made many great contributions to many fields of mathematics,
including in the field of differential equations.} \emph{Canon Arithmeticus}, which
contained tables for primes up to \(1000\).

Primitive roots have other applications beyond being a simple `trick' to perform calculations. They play a role in the elementary
theory of higher-degree congruences; these higher-degree congruences include quadratic congruences, behind which lies one of
the most beautiful theorems in all of number theory.

\end{document}
