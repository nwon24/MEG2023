\documentclass{article}
\usepackage{graphicx} % Required for inserting images

\title{MEG Mastering Angle chasing}
\author{Tom Yan}
\date{June 2023}

\begin{document}

\maketitle

\section{Introduction}
1. In $\triangle ABC$, $AB = AC$ and $\angle A = 40^{\circ}$. The bisector from $\angle B $ intersects $AC$ at point $D$. What is $\angle BDC$? \\\\
2. (AIMO 2019/3) Let $ABCD$ be a square with side length $24$. Let $P$ be a point on side $AB$ with $AP=8$, and let $AC$ and $DP$ intersect at $Q$. Determine the area of triangle $CQD$\\\\
3. (Angle Bisector theorem) Let $ABC$ be a triangle and $D$ be a point on $\overline{BC}$ so that $\overline{AD}$ is the internal angle bisector of $\angle BAC$. Show that $$\frac{AB}{AC} = \frac{DB}{DC}.$$
4. (AIME 2016/6) In $\triangle ABC$ let $I$ be the center of the inscribed circle, and let the bisector of $\angle ACB$ intersect $AB$ at $L$. The line through $C$ and $L$ intersects the circumscribed circle of $\triangle ABC$ at the two points $C$ and $D$. If $LI=2$ and $LD=3$, then find $IC$.\\\\
5. Let $ABC$ be an acute triangle inscribed in circle $\Omega$. Let $X$ be the midpoint of the arc $\widehat {BC}$ not containing $A$ and define $Y$, $Z$ similarly. Show that the orthocenter of $XYZ$ is the incenter $I$ of $ABC$.\\\\
\end{document}
    