%\pagewidth 210mm
%\pageheight 297mm
\documentclass[a4paper,10pt]{article}

\usepackage[pdfusetitle]{hyperref}

\usepackage[a4paper]{geometry}
\usepackage[bf,tiny]{titlesec}
\usepackage{fancyhdr}

%\usepackage[indent=0pt,skip=10pt]{parskip}

\usepackage{amsmath}
\usepackage{amsthm}
\newtheorem{theorem}{Theorem}
\usepackage{amssymb}
\let\mathbbalt\mathbb

\usepackage{fontspec}
\usepackage{unicode-math}
\let\mathbb\mathbbalt

\newcommand{\naturals}{\mathbb{N}}
\newcommand{\reals}{\mathbb{R}}
\newcommand{\rationals}{\mathbb{Q}}
\newcommand{\integers}{\mathbb{Z}}

\newcounter{problemno}
\setcounter{problemno}{1}
\newcommand{\problem}{\section*{\textbf{Problem \theproblemno}}\refstepcounter{problemno}}
\newcounter{interludeno}
\setcounter{interludeno}{1}
\newcommand{\interlude}[1]{\par\noindent\\\fbox{\begin{minipage}{\linewidth}\textbf{Interlude \theinterludeno.\enspace}#1\end{minipage}}\\\refstepcounter{interludeno}}

\newcounter{questionno}
\setcounter{questionno}{1}
\newcommand{\question}[1]{\par\noindent{\thequestionno.\enspace}#1\refstepcounter{questionno}\\}

\newcommand{\thetitle}{Term 1 Week 9 Handout Solutions}

\renewcommand\thesection{\S{\arabic{section}}.}

\title{MEG 2023 Term 1 Week 9 Handout Solutions}
\author{Nathan Wong\and Tom Yan}
\date{2023}
\begin{document}
\noindent Melbourne High School\\
Maths Extension Group 2023\\
\textbf{\thetitle}\\
\section{More combinatorics problems}
\section{Euclid's Algorithm}
\begin{enumerate}
\item Let \(g=\gcd(a,b).\) First solve \[ax-by=g\] and then multiply the solution
  by \(n/g\), for if \((x',y')\) satisfies
  \(ax'-by'=g\) then \((x'n/g,y'n/g)\) satisfies
  \[\frac{ax'n}{g}-\frac{by'n}{g}=n.\]
\item \((x,y)=(5,16).\)
\item We get a solution, but the solution is not in positive
  integers: \((x,y)=(-10,-23).\) To get the solution in positive
  integers, we argue generally. Suppose \(x=x'\) and \(y=y'\) are
  the solutions we have found that are both not positive integers.
  To make them positive integers, we need to add a number to
  both without changing the equality sign. We note that
  \[7200(x'+M)-3132(y'+N)=36\] if and only if
  \(7200M=3132N\). Cancelling \(36\) from both sides
  we get \(200M=87N\). Therefore we can take \(M=87\)
  and \(N=200\) (this is the `smallest' solution because
  \(200\) and \(87\) are coprime).

  Our original solution was \(x'=-10\) and \(y'=-23\), so
  our positive integer solution is \((x,y)=(77, 177)\).
\item Let \(u=u'\) and \(v=v'\) be the integers satisfying
  the equation \[ux+vy=\gcd(x,y).\] Therefore we have
  \[a^{\gcd(x,y)}\equiv a^{u'x+v'y}\equiv(a^x)^u(a^y)^v\equiv1\cdot1\equiv1\pmod{n}.\]
\item We want to show that the equations \[N=am\] and \[N+k=bn\]
  for integers \(m\) and \(n\) are soluble. Putting the first
  into the second we get \[am+k=bn\] which implies \[am-bn=k.\]
  This equation is always soluble because \(a\) and \(b\)
  are coprime.
\item Suppose that \(p\) does not divide \(a\) (if it did we
  have nothing to prove), Our goal now is to show that it
  divides \(b\).

  If \(p\) does not divide \(a\), then it must be coprime with
  \(a\). Hence there exist positive integers \(x\) and \(y\) such that
  \[ px-ay=1.\]
  Multiplying both sides by \(b\) yields
  \[ pbx-aby=b.\]
  Clearly \(p\) divides \(pbx\), and by assumption \(p\) divides
  \(ab\), so it must divide \(aby\) too. But if \(p\) divides
  both \(pbx\) and \(aby\), it must divide their difference,
  which is \(b\).

\end{enumerate}
\end{document}