\documentclass[a4paper,twocolumn]{article}
\usepackage[a4paper,margin=1in]{geometry}
\usepackage[bf,tiny]{titlesec}
\usepackage{amsmath}

\usepackage{fontspec}
\usepackage{unicode-math}
\setmainfont{TeX Gyre Termes}
\setmathfont{TeX Gyre Termes Math}
\begin{document}
%\noindent Melbourne High School\\
%Maths Extension Group\\
%\textbf{Term 1 Week 9 Handout}

\section*{The Euclidean Algorithm}

Divisibility is a most useful concept in mathematics.
We will explore this idea and present a famous algorithm for finding the
greatest common divisor (GCD) of two numbers. 

Our intuitive understanding of divisibility is probably this: an
integer $b$ is said to divide another integer $a$ if and only if
$a$ is a multiple of $b$. 
Another common way to think about it is that dividing $a$ by $b$
leaves no remainder. This definition suffices for most purposes,
but sometimes it is useful to have a more formal notion of the concept:
$b$ divides $a$ if there is some integer $m$ for which $a=mb$. This
is useful in using divisibility information within an algebraic context.

Now that we have established the notion of
divisibility, we can turn to the concepts of divisors (which are also
called the `factors' of a number) and the greatest common divisor.
The \emph{divisors} of an integer $a$ are all the integers that divide $a$.
For example, the divisors of $24$ are $1,2,3,4,6,8,12,24$. Note that a divisor must
be less than or equal to the number it is a divisor of.

Given two integers, we can ask which divisors they have in common.
For example, $2$ is a common divisor of any pair of even numbers, and $3$ is
a common divisor of $9$ and $24$. The \emph{greatest common divisor} (GCD) of
two numbers is the largest of the two numbers' common divisors; in other words,
it is the largest number that divides both of them. This value is of such
important that it has its own notation; it is commonly denoted by $\gcd(a,b)$, where
$a$ and $b$ are nonzero integers. if $\gcd(a,b)=1$, then $a$ and $b$ are \emph{coprime}
or \emph{relatively prime}.

Now do we find the GCD of two numbers? Of course, we could just list the divisors
of each and find the greatest number that is in both lists. However this is slow
and not convenient for large numbers. Fortunately, there is an elegant algorithm of
Euclid's that finds the GCD of two numbers in a much more efficient way.

Suppose we have two positive integers $a$ and $b$ that we want to find the GCD of.
The core of the Euclidean algorithm is \emph{Euclidean division}, which means expressing $a$
in the form \(a=qb+c,\) where $q$ and $c$ are nonnegative integers and $c<b$. We call $q$
the \emph{quotient} and $c$ the \emph{remainder}. Writing
this equation is really the same as dividing $a$ by $b$ with remainder; for example, if $a=104$ and $b=24$
we would write \(104=4\times24+8.\)

Now let $d$ be any common divisor of $a$ and $b$; we know there must be at least one
because $1$ divides every positive integer. Note that $d$ divides $c$ too, because $c=a-qb$.
To make this conclusion we rely on a simple theorem that you can prove for yourself:
if two numbers $m$ and $n$ are both divisible by a third number $k$, then so is their
sum $m+n$ and their difference $m-n$. In our specific case, we assume that $d$ divides both
$a$ and $qb$, and so it must divide their difference $c=a-qb.$

By definition $d$ divides $b$, and we have just shown that $d$ also divides $c$; it follows
from this that $d$ is also a common factor of $b$ and $c$. We have assumed nothing about $d$
as a common divisor of $a$ and $b$, so we are justified in saying that the set of common
divisors of $a$ and $b$ is the same as the set of common divisors of $b$ and $c$. Hence the
GCD of $a$ and $b$ is equal to the GCD of $b$ and $c$.

This may not seem like progress, but it is. We have essentially reduced the original
problem to a smaller one: finding the GCD of $b$ and $c$. It is ``smaller'' because $c<b$,
whereas before $b<a$.

So now we just play the game again; we want to find the GCD of $b$ and $c$ and so we write
\(b=q'c+c'\)
where $q'$ and $c'$ are nonnegative integers and $c'<c$. Again we deduce that the GCD of $b$
and $c$ must be the same as that of $c$ and $c'$.

Obviously  we can't just keep going on like this forever, or else we wouldn't be solving
the original problem of finding $\gcd(a,b)$. Where does it stop?

Note that the sequence of remainders we obtain by successive use of Euclidean division is a strictly decreasing sequence.
The second time we did the division we had $c'<c$, and if we had continued we would have $c''<c'$, $c'''<c''$, and so on.
A decreasing sequence of nonnegative integers must reach $0$ at some point. That is, one of the terms in the sequence
$c,c',c'',\ldots$ must be $0$. Let the term directly preceding it be $t$ and the term preceding $t$ be $s$. Recall
that each term in the sequence comes from the Euclidean division of the preceding terms: $c$ is the remainder
when $a$ is divided by $b$, the next term $c'$ is the remainder when $b$ is divided by $c$, and so on.

How did we get this $0$ in our list of remainders? We must have used the division algorithm on $s$ and $t$,
the two previous terms. Therefore the relationship between $s$ and $t$ is
\(s=tu+0\)
for some positive integer $u$. From this we infer that $s$ is actually a multiple of $t$. But if $s$ is a multiple
of $t$, then the GCD of $s$ and $t$ must be $t$.

So we have found the GCD of $t$ and $s$, but we haven't solved our original problem, which was to
find the GCD of $a$ and $b$. Actually, we have! Recall that we showed the GCD of $a$ and $b$ is the same
as that of $b$ and $c$. In the same way, the GCD of $b$ and $c$ is the same as that of $c$ and $c'$. 
Continuing in this fashion. we find that the GCD of $a$ and $b$ is the same as that of $s$ and $t$; hence
$\gcd(a,b)=t$, the last nonzero remainder in our list of remainders supplied by repeated application of Euclidean division.

That, in a nutshell, is the Euclidean algorithm for finding the GCD of two numbers. To summarise, here is the
algorithm in two simple steps. Assume we want to find $\gcd(a,b)$.
\begin{enumerate}
\item If $a$ is a multiple of $b$, then $\gcd(a,b)=b$.
\item Otherwise, let $c$ be the remainder when $a$ is divided by $b$. Then $\gcd(a,b)=\gcd(b,c)$.
\end{enumerate}

Up until now, we have been using only algebraic symbols, which may have made the process of the algorithm
confusing. Let's now do a concrete example and find $\gcd(177,48)$. We begin by writing
\(177=3\times48+33.\)
Now the problem is reduced to finding $\gcd(48,33)$, for which we continue with
\(48=1\times33+15.\)
The remainder is not $0$ yet so the algorithm continues.
\(33=2\times15+3.\)
It is clear that the next remainder will be $0$. One more application
of the division algorithm yields
\(15=5\times3+0.\)
We have reached a $0$ remainder, so $\gcd(177,48)=3$, the last nonzero remainder. Note we could
also have deduced that since $\gcd(15,3)=3$, then $\gcd(177,48)=\gcd(15,3)=3.$

This was a small example, so you might have been able to do it the nai\'ve way (listing
the factors of both numbers) or even in your head. Now try finding \(\gcd(1160718174,316258250).\)
(Despite the size of the numbers, you can get the answer using only a scientific calculator. Your
immediate objection might be that the calculator has no button to give the remainder upon division;
how might you get the remainders manually?)

\section*{A simple linear equation}

In this section, let $g=\gcd(a,b).$

Surprisingly, Euclid's algorithm answers the question of when the equation
\begin{equation}ax-by=c\label{eq:lineareq}\end{equation} has natural number solutions in $x$ and $y$. It is clear that $ax-by$
is always a multiple of $g$, because both $ax$ and $by$ are multiples of $g$.
Thus \eqref{eq:lineareq} has natural number solutions only when $c$ is a multiple of $g$.

Now that we know when \eqref{eq:lineareq} has solutions, the next question is whether
there is always a solution with $c=g$; that is, whether a linear combination of $a$ and $b$
can ever equal their GCD. The answer is yes; the equation $ax-by=g$ always has solutions
in natural numbers $x$ and $y$. To see why, we need Euclid's algorithm.

If $c$ is the sum or difference of multiples of $a$ and $b$ (this is the same
as saying there are integer solutions to $ax-by=c$), we say that $c$
is \emph{linearly dependent} on $a$ and $b$. We now aim to prove two results:
\begin{enumerate}
\item If $c$ is linearly dependent on $a$ and $b$, so is $kc$ for any natural number $k$.
\item If both $c$ and $d$ are linearly dependent on $a$ and $b$, so are $c+d$ and $c-d$.
\item Both $a$ and $b$ are linearly dependent on themselves.
\end{enumerate}
The first is easy; if $c=ax-by$ then $kc=akx-bky$, so $kc$ is also linearly dependent
on $a$ and $b$, with solution $(kx,ky)$. The second is only slightly more difficult. 
Let \(c=ax_1-by_1\) and $$d=ax_2-by_2.$$ Then $$c+ d=(ax_1-by_1)+ (ax_2-by_2),$$
which implies \(c+d=a(x_1+x_2)-b(y_1+y_2).\) showing that $c+d$ is also linearly dependent
on $a$ and $b$. Similarly, \(c-d=(ax_1-by_1)-(ax_2-by_2), \) and so $$c-d=a(x_1-x_2)-b(y_1-y_2).$$

We verify the third statement simply by seeing that $a=a(b+1)-b(a)$ and $b=b(a+1)-a(b)$. 
Note that we cannot write $a=a\cdot1+b\cdot0$, because $x$ and $y$
are natural numbers.

We are now ready to use Euclid's algorithm to show that the equation $ax-by=g$ is soluble
in the natural numbers.
Recall that the first line of the algorithm is to write $a=qb+c$, which implies $c=a-qb.$
From above, we know that since $a$ and $qb$ are both linearly dependent on $a$ and $b$, so
is their difference, which is $c$. Now we know that both $b$ and $c$ are linearly dependent
on $a$ and $b$, and so the next remainder in the algorithm will also be linearly dependent on
$a$ and $b$. Continuing down the lines of the algorithm, we see that each remainder can
be expressed in the form $ax-by$, and since $g=\gcd(a,b)$ is the last nonzero remainder, it is
linearly dependent on $a$ and $b$. Thus we have accomplished what we set out to show; linear equations
of the form $ax-by=g$ are always soluble in the natural numbers.
\end{document}
