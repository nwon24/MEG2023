\documentclass[a4paper]{article}

\newcommand{\theterm}{4}
\newcommand{\theweek}{3}
\newcommand{\thepdftitle}{MEG 2023 Term \theterm\ Week \theweek\ Handout}
\newcommand{\thedisplaytitle}{Term \theterm\ Week \theweek\ Handout}

\title{{\thepdftitle}}
\author{Nathan Wong\and Tom Yan}
\date{2023}

\newcommand{\leg}[2]{\left(\frac{#1}{#2}\right)}
\newcommand{\ileg}[2]{(#1|#2)}
\newcommand{\floor}[1]{\lfloor#1\rfloor}
\newcommand{\bfloor}[1]{\left\lfloor#1\right\rfloor}
\newcommand{\ceil}[1]{\lceil#1\rceil}
\newcommand{\bceil}[1]{\left\lceil#1\right\rceil}

%\newcommand{\marginfn}[1]{\marginpar{\footnotemark}\footnotetext{#1}}
\newcommand{\marginnote}[1]{\marginpar{\footnotesize{#1}}}
\newcommand{\marginfnote}[1]{\footnotemark\marginpar{\footnotemark[\value{footnote}]\footnotesize{#1}}}
\usepackage{geometry}
%\geometry{a4paper,left=24.8mm,top=27.4mm,headsep=2\baselineskip,textwidth=107mm,marginparsep=8.2mm,marginparwidth=49.4mm,textheight=49\baselineskip,headheight=\baselineskip}
\geometry{a4paper,left=1in,top=1in,bottom=1in,headsep=2\baselineskip,textwidth=107mm,marginparsep=8.2mm,marginparwidth=49.4mm,textheight=49\baselineskip,headheight=\baselineskip}
\usepackage[bf,tiny]{titlesec}
%\usepackage{fancyhdr}
\usepackage{epigraph}
%\usepackage[indent=0pt,skip=10pt]{parskip}

\usepackage{amsmath}
\usepackage{amsthm}
\newtheorem{theorem}{Theorem}
\usepackage{amssymb}
\let\mathbbalt\mathbb

\usepackage{tikz}

\usepackage{fontspec}
\usepackage{unicode-math}
\let\mathbb\mathbbalt

\newcommand{\naturals}{\mathbb{N}}
\newcommand{\reals}{\mathbb{R}}
\newcommand{\rationals}{\mathbb{Q}}
\newcommand{\integers}{\mathbb{Z}}

\usepackage[pdfusetitle]{hyperref}

\newcommand{\myquote}[2]{%
  \begin{quote}
    \emph{#1}
    \begin{flushright}---{#2}
    \end{flushright}
  \end{quote}}
\pagestyle{empty}
\begin{document}
\noindent Melbourne High School\\\
\noindent Maths Extension Group 2023\\\
\noindent \textbf{\thedisplaytitle}\\\
\myquote{``Yes, universal history! It’s the study of the successive follies of mankind and nothing more. The only subjects I respect are mathematics and natural science,'' said Kolya.\marginnote{Don't listen to Kolya. History is a nice subject.}}{F.~Dostoevsky\footnote{Translated from the Russian
by C.~Garnett}, \emph{The Brothers Karamazov} (1880)}
\section*{Problems}
\begin{enumerate}
	\item Twenty \marginnote{ToT Spring 2016/1} children stand in a circle (both boys and girls are present). For each boy, his clockwise neighbour is in a blue T-shirt, and for each girl, her counterclockwise neighbour is in a red T-shirt. Is it possible to determine the precise number of boys in the circle? 
	\item Let \marginnote{ToT Fall 2019/2} $\omega$ be a circle with the centre $O$ and two different points $A$ and $C$ on $\omega$. For any point $P$ on $\omega$ distinct from $A$ and $C$ let $X$ and $Y$ be the midpoints of $AP$ and
	$P$ respectively. Furthermore, let $H$ be the point where the altitudes of triangle $OXY$ meet. Prove that the position of the point $H$ does not depend on the choice of $P$. 
	\item There \marginnote{ToT Fall 2019/3} is a row of 100 cells each containing a token. For 1 dollar it
is allowed to interchange two neighbouring tokens. Also it is allowed
to interchange with no charge any two tokens such that there are
exactly 3 tokens between them. What is the minimum price for
arranging all the tokens in the reverse order? 
\item In \marginnote{ToT Fall 2015/4} a right-angled triangle $ABC$ $(\angle C = 90^{\circ})$ points $K$, $L$ and $M$ are chosen on sides $AC$, $BC$ and $AB$ respectively so that $AK = BL = a$, $KM = LM = b$ and $∠KML = 90^{\circ}$. Prove that $a = b$. 
\end{enumerate}
\pagebreak
\myquote{We may have succeeded at the price of too much eccentricity,
or we may have failed; but we can hardly have failed completely,
the subject matter being so attractive that only extravagant incompetence
could make it dull.}{G.~H.~Hardy, E.~M.~Wright, \emph{An Introduction to the Theory of Numbers} (1938)}
\section*{Gauss, Eisenstein, and the Proof}
The proof of quadratic reciprocity from
Euler's conjecture is straightforward, but
not the most beautiful, requiring
complex manipulation of
inequalities and intervals.
Here we discuss a second proof that is decidedly
simpler and more enlightening.\marginnote{For this proof
we have Gauss to thank. The geometric twist is due to
Eisenstein.}

Let's solidify some notation first. The pronumerals
\(p\) and \(q\) refer to distinct odd primes.
Let \(P=(p-1)/2\) and \(Q=(q-1)/2\). Let \(v(a,p)\)
be the number of minus signs in the set
formed by reducing the numbers
\[a,2a,\ldots,\frac{p-1}{2}a\]
into the range \((-p/2,p/2)\).
From Gauss's Lemma 
\[\leg{a}{p}=(-1)^{v(a,p)}.\]
(Before, this value was denoted just
as \(v\).)
Next: let\marginnote{Those who have done some 
of the previous exercises have no doubt already
met this strange creature.}
\[S(q,p)=\sum_{k=1}^{P}\bfloor{\frac{kq}{p}}.\]
The value of \(S(q,p)\) has a geometric
meaning. In the example diagram below \(p=17\) and \(q=13\). 
\begin{figure}[h]
	\begin{tikzpicture}[scale=0.8]
	\coordinate (org) at (0,0);
	\coordinate (xmax) at (8,0);
	\coordinate (ymax) at (0,6);
	\draw [thin, black, -latex] (org) -- (xmax);
	\draw [thin, black, -latex] (org) -- (ymax);
	\foreach \x in {1,2,...,8}{
		\foreach \y in {1,2,...,6}{
			\node[draw, circle, inner sep = 0.5pt, fill] at (\x,\y) {};
		}
	}
	\draw [thin, black] (org) -- (8, {8*(13/17)});
\end{tikzpicture}
\end{figure}
	\marginnote{This diagram was a pain to create.}
Each dot is a lattice point, a point with integer
coordinates; in what follows disregard any lattice
points on the coordinate axes.

The diagonal line is the line \(y=13x/17.\)
It does not cross any lattice points, because
the first lattice
point on the line that is not the origin is 
\((17,13)\).

By definition, 
\(S(13,17)=\sum_{k=1}^{8}\floor{13k/17}.\)
The number \(13k/17\) is the value of the line
at the point \(x=k\); and hence \(\floor{13k/17}\)
is the \emph{number of lattice points} below
the line with \(x\)-coordinate equal to \(k\).
For example, if \(x=3\) the \(y\)-value is
\(39/17\), the floor of which is \(2\),
corresponding to the \(2\) lattice points
on the vertical line \(x=3\) that lie below
the line \(y=13x/17\).

Hence \(S(13,17)\) is the number of lattice points
under the line \(y=13x/17\) bounded horizontally
by the lines \(x=0\) and \(x=8\). If, however, the
plane is flipped so that the \(x\)-axis becomes
the \(y\)-axis and the \(y\)-axis becomes the \(x\)-axis, the equation
of the line becomes \(y=17x/13\) and the number
of lattice points below the line bounded
by the lines \(x=0\) and \(x=6\) is \(S(17,13)\).
This number \(S(17,13)\) is the also the number
of lattice points above the line \(y=13x/17\) bounded
by the lines \(y=0\) and \(y=6\) in
the original version of the plane. 
Plainly the total number of lattice points\marginnote{Remember
that we don't count the lattice points on the horizontal
and vertical axes.}
in the rectangle bounded by the \(x\)-axis,
the \(y\)-axis, the line \(x=8\), and the line
\(y=6\)
is \(6\times8=48\), so
\[S(17,13)+S(13,17)=48=\frac{(13-1)}{2}\frac{(17-1)}{2}.\]
Note that the same argument applies even if
\(p\) and \(q\) are unspecified. The line
\(y=qx/p\) splits the number of lattice points
in the rectangle bounded by the axes and the 
lines \(x=P\) and \(y=Q\)
into two triangles with no lattice point
lying on the line itself. The number below
the line is \(S(q,p)\), the number above the
line is \(S(p,q)\), and the total number is
\(PQ\); hence\marginnote{Isn't this beautiful?}
\[S(q,p)+S(p,q)=PQ=\frac{(p-1)}{2}\frac{(q-1)}{2}.\]
Now consider the quotient when \(kq\) is divided
\(p\) with a remainder that lies between \(-p/2\)
and \(p/2\). That is, let
\begin{equation}\label{eq:euclid}
	kq=q_kp+r_k
\end{equation}
where \(-p/2<r_k<p/2.\)
Rearranging for \(kq/p\) yields
\[\frac{kq}{p}=q_k+\frac{r_k}{p}.\]
From this equation it follows that
\[\bfloor{\frac{kq}{p}}=
\begin{cases}
	q_k,&\text{ if }r_k>0;\\
	q_k-1,&\text{ if }r_k<0.
\end{cases}
\]
Therefore \(S(q,p)\) is the sum
of the quotients \(q_k\) minus the number
of times \(\floor{kq/p}=q_k-1\). How many times does
this happen?\marginnote{We don't need to consider the
case \(r_k=0\) because \(p\) and \(q\) are distinct
odd primes. The first positive value of \(k\) for
which dividing \(qk\) by \(p\) leaves no remainder
is \(k=p\); here we consider only the values of \(k\)
from \(1\) to \((p-1)/2\).}

It happens whenever one of the remainders \(r_k\) is negative.
These remainders are obtained by reducing
multiples of \(q\) to be between \(-p/2\)
and \(p/2\); hence the number of
times \(\floor{kq/p}=q_k-1\) is exactly \(v(q,p)\).
This implies
\begin{equation}\label{eq:import}
	S(q,p)=\sum_{k=1}^{P}q_k-v(q,p).
\end{equation}
Next: what is the sum of those quotients?
From Equation \eqref{eq:euclid}
\[ q_kp=kq-r_k\]
and so, on summing both sides from \(k=1\) to \(k=P\),
\[\sum_{k=1}^P q_kp=\sum_{k=1}^P{kq-r_k}.\]
Split the right-hand sum into two sums and take the constants
\(p\) and \(q\) out of the summation signs to get
\[p\sum_{k=1}^P q_k=q\sum_{k=1}^P k-\sum_{k=1}^P r_k.\]
Consider this equation modulo \(2\). Since \(p\)
and \(q\) are odd primes, they disappear, leaving
\[\sum_{k=1}^P q_k\equiv\sum_{k=1}^P k-\sum_{k=1}^P r_k\pmod{2}.\]
The numbers \(r_k\) are the remainders on
reducing numbers of the form \(kq\) to be in the
range \((-p/2,p/2)\); from the proof of Gauss's Lemma,
amongst these remainders the numbers \(1,2,\ldots,P\)
appear exactly once, with either a positive or negative sign but not both.
\marginnote{A quick proof if you haven't seen Gauss's Lemma: if \(iq\equiv jq\)
(same sign) then \(i\equiv q\) and hence \(i=q\). If two remainders
appear with the opposite sign then \(iq\equiv -jq\) and hence \(i+j\equiv0\),
which is impossible because \(i,j\le (p-1)/2\) and hence \(i+j\le p-1\).}
The signs don't matter since \(-1\equiv1\pmod{2}\), and so
\[\sum_{k=1}^P r_k\equiv 1+2+\cdots+P\equiv\sum_{k=1}^P k\pmod{2}.\]
Therefore
\[\sum_{k=1}^P q_k\equiv \sum_{k=1}^P k-\sum_{k=1}^P k\equiv0\pmod{2}.\]
Reduce Equation \eqref{eq:import} modulo \(2\); since \(\sum_{k=1}^P q_k\equiv0\pmod{2}\) it is clear, then, that
\[S(q,p)\equiv -v(q,p)\equiv v(q,p)\pmod{2}.\]
Gauss's Lemma implies
\[\leg{q}{p}=(-1)^{v(q,p)}=(-1)^{S(q,p)}.\]
Similarly
\[\leg{p}{q}=(-1)^{v(p,q)}=(-1)^{S(p,q)}.\]
The proof is almost complete. From here it follows naturally
that
\[\leg{q}{p}\leg{p}{q}=(-1)^{S(q,p)}(-1)^{S(p,q)}=(-1)^{S(q,p)+S(p,q)}.\]
But \(S(q,p)+S(p,q)=(p-1)(q-1)/4\) and, consequently, 
\begin{equation}\label{eq:qrep}
	\leg{q}{p}\leg{p}{q}=(-1)^{\frac{p-1}{2}\frac{q-1}{2}}.
\end{equation}
This equation is, in fact, an
equivalent formulation of the law of quadratic reciprocity, and
the one that is most common in the literature.
The proof is simple. The exponent on the right is even whenever
\(p\equiv1\pmod{4}\) or \(q\equiv1\pmod{4}\), and \(\ileg{q}{p}\ileg{p}{q}=1\)
implies \(\ileg{q}{p}=\ileg{p}{q}\). Similarly, the exponent is odd
whenever \(p\equiv q\equiv3\pmod{4}\), and \(\ileg{q}{p}\ileg{p}{q}=-1\)
implies \(\ileg{q}{p}=-\ileg{p}{q}.\)

Thus ends a glorious journey into the world of quadratic
residues. 
The nuggets of gold discovered are best summarised by the following
three equations:\marginnote{Who would have thought there was so much
in the simple question of which numbers are squares modulo primes?}
\[
	\leg{-1}{p}=(-1)^{\frac{p-1}{2}},
	\quad
	\leg{2}{p}=(-1)^{\frac{p^2-1}{8}},
	\quad
	\leg{p}{q}\leg{q}{p}=(-1)^{\frac{p-1}{2}\frac{q-1}{2}}.
\]
\section*{Problem}
Generalise the Legendre symbol. Define a new symbol \(\ileg{a}{b}\),
where \(b\) is a positive odd integer with prime factorisation
\(b=p_1p_2\cdots p_r\) and \(a\) is any other integer,
to have the value
\[\leg{a}{b}=\leg{a}{p_1}\leg{a}{p_2}\cdots\leg{a}{p_r}.\]
Discover and prove properties of this symbol.\marginnote{If you're
thinking of naming this symbol after yourself, Jacobi has beaten you
to it.}
\end{document}

