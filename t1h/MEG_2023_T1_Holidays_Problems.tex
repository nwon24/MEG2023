\documentclass[a4paper]{article}
\usepackage{geometry}

\usepackage{fontspec}
\usepackage{unicode-math}

\title{MEG 2023 Term 1 Holiday Problems}
\author{Nathan Wong\and Tom Yan}
\date{2023}

\usepackage[pdfusetitle]{hyperref}

\usepackage{amsmath}
\usepackage{amsthm}

\begin{document}
\noindent Melbourne High School\\
Maths Extension Group 2023\\
\textbf{Term 1 Holiday Problems}\\

\begin{enumerate}
\item Suppose you have \(n\) distinct
  objects, each with an equal probability of being selected. After
  an object is selected, it is replaced. Each selection of an object
  is considered a ``turn.'' What is the expected value of turns that you
  need to complete before you have selected at least one of each object?

\item Your good friend Harold is about to start reading Virginia Woolf's \emph{Between the Acts}. The book
  has \(199\) pages, which you instantly recognise as being prime.

  Since Harold is much
  smarter than you, he doesn't read books in the correct page order. He begins by choosing a random number
  between \(1\) and \(199\)---call this number \(x\). To determine the \(n\)th
  page that he will read, he multiplies \(n\) by \(x\), and if the number is greater than \(199\), he
  subtracts off a multiple of \(199\) so that he has a page number that is actually in the book. He then reads
  that page before calculating the next page he will read.

  Show that not only will Harold read each page of the book, but he will also never
  read a page more than once. 
  
\item Find all natural numbers $n$ such that $2^{2^n}+5$ is prime. 
\item Let $N$ be a regular nonagon, having $O$ as the centre of its circumcircle, and let $PQ$ and $QR$ be adjacent edges of $N$. The midpoint of $PQ$ is $A$ and the midpoint of the radius perpendicular to $QR$ is $B$. Determine the angle between $AO$ and $AB$.
\item Let $0<a<1$. Solve \[x^{a^x}=a^{x^a}\] for positive numbers $x$.
\item Find the sum of all positive integers \(n\) for which \[n^2-19n+99\] is a perfect square.
\item Prove that the fraction \[\frac{21n+4}{14n+3}\]
  cannot be reduced further for all natural number values of \(n\).
\item Show that we can colour the elements of the set $S=\{1,2,\ldots, 2007\}$ with 4 colours such that any subset of $S$ with 10 elements, whose elements form an arithmetic sequence, is not all one colour.
\item Two points $K$ and $L$ are chosen inside triangle $ABC$ and a point $D$ is chosen on the side $AB$. Suppose that $B$, $K$, $L$, $C$ are concyclic, $\angle AKD = \angle BCK$ and $\angle ALD = \angle BCL$. Prove that $AK = AL$.

\item (This is a long problem.)
  \begin{enumerate}
  \item Evaluate \((a+b)^2\pmod{2}\), \((a+b)^3\pmod{3}\),
    and \((a+b)^5\pmod{5}\).
  \item The Binomial Theorem states that
    \[(a+b)^n=\sum_{k=0}^{n}\binom{n}{k}a^{n-k}b^k\]
    where
    \[\binom{n}{k}=\frac{n!}{k!(n-k)!}.\]
    Prove that
    \[(a+b)^p\equiv a^p+b^p\pmod{p}\] for any prime \(p\)

\item Suppose you have \(6\) different applied mathematics books that you would like place in \(3\) bins. How many
  ways are there to distribute the books such that the first bin has \(1\) book, the second \(2\) books, and the third \(3\) books?
\item Now suppose you have \(n\) different applied mathematics books and \(m\) bins. Someone supplies you a list of numbers whose sum is \(n\):
  \(k_1,k_2,k_3,\ldots,k_m\). How many ways are there to distribute the books such
  that the \(i\)th bin receives \(k_i\) books? That is, the first bin should receive \(k_1\) books, the second \(k_2\) books, and so on. Note
  that \(k_1+k_2+k_3+\cdots+k_m=n\).
\item If you have completed the last part correctly, you have just discovered \emph{multinomial coefficients}, which, as you might be able
  tell by the name, generalise the venerable binomial coefficients.

  Let \(n\) be a positive integer and let \(k_1,k_2,k_3,\cdots k_m\) be \(m\) positive integers that sum to \(n\). This multinomial coefficient
  is denoted \[\binom{n}{k_1,k_2,\ldots,k_m}.\] If your answer to the previous part involved binomial coefficients, make use of a
  telescoping product to show that \[\binom{n}{k_1,k_2,\ldots,k_m}=\frac{n!}{k_1!k_2!\cdots k_m!}.\]
\item Expand \((a+b+c)^3\), taking note of the coefficients and the powers of \(a\), \(b\), and \(c\) in each term.
\item By first doing some more examples if necessary, come up with a conjecture about the expansion of
  \[(x_1+x_2+x_3+\cdots+x_m)^n\]
  using multinomial coefficients.
\item Prove your conjecture by induction (you will need the binomial theorem).
\item If you have made it this far in the problem, well done.

  Prove that
  \[(x_1+x_2+x_3+\cdots+x_m)^p\equiv x_1^p+x_2^p+x_3^p+\cdots x_m^p\pmod{p}\]
    for any prime \(p\). (You can induction, but you could also look at the definition of a multinomial coefficient directly.)
  \item Hence, show that
    \[ a^p\equiv a\pmod{p}\]
    for any prime \(p\) and any positive integer \(a\).
\end{enumerate}
\item Find \((102^{73}+55)^{37}\) modulo \(111\).
\item We define the Fibonacci numbers by the recurrence relation
  \(F_n=F_{n-1}+F_{n-2}\), where \(F_0=0\) and \(F_1=1\). Earlier
  in the term we discovered that the generating function
  \[G(x)=F_0+F_1x+F_2x^2+F_3x^3+\cdots\]
  has the closed form expression
  \[\frac{x}{1-x-x^2}.\]
  In this problem, we will use this result to derive a closed-form
  expression for the \(n\)th Fibonacci number.

  \begin{enumerate}
\item  Begin by showing that the closed-form expression for \(G(x)\) can be
  written \[\frac{x}{(1-\alpha x)(1-\beta x)}\]
  where \[\alpha=\frac{1+\sqrt{5}}{2}\quad\text{and}\quad\beta=\frac{1-\sqrt{5}}{2}.\]
\item This fraction can be split into the sum of two fractions: \[\frac{A}{1-\alpha x}+\frac{B}{1-\beta x}.\]
  Show that \(A=1/\sqrt5\) and \(B=-1/\sqrt5\).
\item Show that \[1+x+x^2+x^3+\cdots=\frac{1}{1-x}.\]
  (This value makes sense for \(|x|<1\).)
\item Hence show that \[F_n=\frac{1}{\sqrt5}\left(\left(\frac{1+\sqrt5}{2}\right)^n-\left(\frac{1-\sqrt5}{2}\right)^n\right).\]
  This is known as \emph{Binet's Formula}.
\end{enumerate}
\end{enumerate}
\end{document}
