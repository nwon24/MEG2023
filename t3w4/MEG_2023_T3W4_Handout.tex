\documentclass[a4paper]{article}

\newcommand{\theterm}{3}
\newcommand{\theweek}{4}
\newcommand{\thepdftitle}{MEG 2023 Term \theterm\ Week \theweek\ Handout}
\newcommand{\thedisplaytitle}{Term \theterm\ Week \theweek\ Handout}

\title{{\thepdftitle}}
\author{Nathan Wong\and Tom Yan}
\date{2023}

%\newcommand{\marginfn}[1]{\marginpar{\footnotemark}\footnotetext{#1}}
\newcommand{\marginnote}[1]{\marginpar{\footnotesize{#1}}}
\newcommand{\marginfnote}[1]{\footnotemark\marginpar{\footnotemark[\value{footnote}]\footnotesize{#1}}}
\usepackage{geometry}
%\geometry{a4paper,left=24.8mm,top=27.4mm,headsep=2\baselineskip,textwidth=107mm,marginparsep=8.2mm,marginparwidth=49.4mm,textheight=49\baselineskip,headheight=\baselineskip}
\geometry{a4paper,left=1in,top=1in,bottom=1in,headsep=2\baselineskip,textwidth=107mm,marginparsep=8.2mm,marginparwidth=49.4mm,textheight=49\baselineskip,headheight=\baselineskip}
\usepackage[bf,tiny]{titlesec}
%\usepackage{fancyhdr}
\usepackage{epigraph}
%\usepackage[indent=0pt,skip=10pt]{parskip}

\usepackage{amsmath}
\usepackage{amsthm}
\newtheorem{theorem}{Theorem}
\usepackage{amssymb}
\let\mathbbalt\mathbb

\usepackage{fontspec}
\usepackage{unicode-math}
\let\mathbb\mathbbalt

\newcommand{\naturals}{\mathbb{N}}
\newcommand{\reals}{\mathbb{R}}
\newcommand{\rationals}{\mathbb{Q}}
\newcommand{\integers}{\mathbb{Z}}

\newcommand{\legendre}[2]{\left(\frac{#1}{#2}\right)}
\newcommand{\ilegendre}[2]{(#1|#2)}

\usepackage[pdfusetitle]{hyperref}

\newcommand{\myquote}[2]{%
  \begin{quote}
    \emph{#1}
    \begin{flushright}---{#2}
    \end{flushright}
  \end{quote}}
\pagestyle{empty}
\begin{document}
\noindent Melbourne High School\\
\noindent Maths Extension Group 2023\\
\noindent \textbf{\thedisplaytitle}\\
\myquote{If we arrive at an equation containing on each side the same term but with different coefficients, we must take equals from equals until we get one term equal to another term. But, if there are on one or on both sides negative terms, the deficiencies must be added on both bides until all the terms on both sides are positive. Then we must take equals from equals until one term is left on each side.}{Diophantus,\footnote{%
  As quoted by Thomas Little Heath in \emph{Diophantus of Alexandria} (1885)} \emph{Arithmetica} (c.~250 AD)}
\section*{Problems in factorisation and Diphantine equations}
\begin{enumerate}
\item Find integers $(x,y)$ such that $x^2=12+y^2$. 
\item Find all paris of integers $(b, c)$ such that $$2b+3c=bc.$$
\item Determine \marginnote{INMO} all non negative integral pairs $(x,y)$ for which $$(xy-7)^2=x^2+y^2.$$ 
\item The \marginnote{BMO Round 3 2015} integer $n$ is positive. There are exactly $2005$ ordered pairs $(x, y)$ of positive integers satisfying $$\frac{1}{x} + \frac{1}{y} = \frac{1}{n}.$$ Prove that $n$ is a perfect square. 
\item Determine \marginnote{IMO 2006/4}  all pairs $(x, y)$ of integers such that\[1+2^{x}+2^{2x+1}= y^{2}.\]
\end{enumerate}
\pagebreak
\myquote{The mathematician's patterns, like the painter's or the poet's,
must be \emph{beautiful}; the ideas, like the colours or the words,
must fit together in a harmonious way. Beuaty is the first test: there
is no permanent place in the world for ugly mathematics.}{G.~H.~Hardy, \emph{A Mathematician's Apology} (1940)}
\section*{Quadratic Residues}
Having\marginnote{Be aware that we are now entering a most beautiful area of number theory; prepare yourself.}
 conquered linear congruences, let's turn to
quadratic congruences. The central question in
this area of number theory is:
\begin{center}
  When is the congruence \(x^2\equiv a\pmod{m}\) solvable?
\end{center}
Note that we are primarily interested in \emph{when} the congruence
has a solution, not what the solutions are (if there are any).

As should be familiar by now, we start by considering when \(m\)
is prime. Therefore our main goal is to determine when
\[ x^2\equiv a\pmod{p}\] is solvable. Let's begin with the congruence
\[x^2\equiv 3\pmod{7},\] Is there a solution? Since the modulus
is small it is easy to run through all the possibilities, like so:
\[
  \begin{split}
    1^2&\equiv1\pmod{7}\\
    2^2&\equiv4\pmod{7}\\
    3^2&\equiv2\pmod{7}\\
    4^2&\equiv2\pmod{7}\\
    5^2&\equiv4\pmod{7}\\
    6^2&\equiv1\pmod{7}
  \end{split}
\]
It looks like our equation is not solvable, as when numbers
are squared modulo \(7\), the only possible residues are \(1\), \(2\),
and \(4\). 
We note that \(3^2\equiv4^2\), \(2^2\equiv5^2\), \(1^2\equiv6^2\), but
after some thought this is hardly surprising, as \(1\equiv=6\), \(2\equiv-5\), and \(3\equiv-4\)
modulo \(7\).

Let's try another example, this time with the congruence \[x^2\equiv 10\pmod{7}.\]
If we run through the possible values of \(x\) again, we find that the squares
are congruent to \(1,4,9,5,3\) only. Therefore our congruence is not solvable.

The numbers that are congruent to squares modulo a particular prime---\(1,2,4\) in the case of modulo \(7\) and
\(1,4,9,5,3\) in the case of modulo \(11\)---have a special name; they
are called \emph{quadratic residues}.
Numbers that are not quadratic residues are called \emph{quadratic nonresidues}.
In other words if \(a\) is a quadratic residue modulo \(p\),
the congruence \[x^2\equiv a\pmod{p}\] is solvable; if \(a\) is a quadratic nonresidue,
then the congruence is not solvable.
 We always exclude \(0\) has a quadratic
residue, not only because it is trivial, but also because it simplifies many of 
our arguments, as will be seen shortly. Additionally, in general we consider only odd primes, leaving \(2\) for special cases because modulo \(2\)
 every number is a quadratic residue.

All the quadratic residues of a given prime can be found by running \(x\)
through a set of positive residues, like we did above. As one further example,
let's find the quadratic residues modulo \(13\): \(1^2\equiv1\), \(2^2\equiv4\),
\(3^2\equiv9\), \(4^2\equiv3\), \(5^2\equiv12\), and \(6^2\equiv10\). As you 
might have realised already, we do not need to go further than \(6\), for the
residues repeat.\marginnote{That is, \(7^2\equiv10\), \(8^2\equiv12\), and so on, as you can verify.}
Therefore the quadratic residues modulo \(13\) are \(1,4,9,3,12,10\).

Let's turn to the general case.
The first question, which you may have already answered, is: How many quadratic
residues are there modulo an odd prime \(p\)? (Remember that we exclude \(2\) from
all of these investigations.) From the examples above, it seems that the numbers
\[1^2,2^2,\ldots,\left(\frac{p-1}{2}\right)^2\] are all incongruent and therefore
form quadratic residues, and all the numbers greater than \((p-1)/2\) and less than
\(p\) form the same set of quadratic residues but in reverse order. A consequence of this is that there are
exactly \((p-1)/2\) quadratic residues and \((p-1)/2\) quadratic nonresidues, which is consistent with the above examples.

How do we prove this hypothesis? An easy way is using primitive roots. \marginnote{Another way is to show that
if \(a\) and \(b\) are both less than \(p/2\), then \(a^2\not\equiv b^2\).}
If \(a\equiv x^2\) is a quadratic
residue and \(\alpha\) is the index  of \(x\), then \[a\equiv g^{2\alpha}\pmod{p}.\]
The index of \(a\) is \(2\alpha\) reduced modulo \(p-1\), and since \(p-1\) is even,
the index of \(a\) is even. Hence if \(a\) is a quadratic residue then its index is even.
The contrapositive: if the index of \(a\) is odd, then \(a\) is a quadratic nonresidue.
Combining these two statements, it follows that the quadratic residues modulo \(p\) are
precisely those numbers with even indices, and the quadratic nonresidues are precisely
those numbers with odd indices.
\marginnote{You may verify this by constructing a table of indices for \(11\) using \(2\)
as the primitive root; you will see that the indices of the quadratic residues \(1\),
\(4\), \(9\), \(5\), and \(3\) are the even numbers \(10\), \(2\), \(6\), \(4\), and \(8\)
respectively.} Indices range from \(1\) to \(p-1\), and as there
are \((p-1)/2\) even numbers and \((p-1)/2\) odd numbers in that range, there are exactly
\((p-1)/2\) quadratic residues and \((p-1)/2\) quadratic nonresidues.

The connection between primitive roots and quadratic residues allows us to
formulate a simple multiplicative property. Recall that multiplying numbers modulo \(p\)
is equivalent to adding their indices and reducing that index modulo \(p-1\) if necessary. If
two quadratic residues are multiplied, the sum of their indices, which are both even,
is also even, and so the product is another quadratic residue. If a quadratic residue
is multiplied by a quadratic nonresidue, an even index and an odd index are added together;
the resulting index is odd and so the product is a quadratic nonresidue. Finally, if
two quadratic nonresidues are multiplied, the resulting index is even because the sum
of two odd numbers is even. It follows that the product of two quadratic nonresidues
is a quadratic residue.

These multiplicative rules bear a strange resemblance to the rules governing the
multiplication of positive and negative numbers. Indeed, if we take a quadratic residue
to be represented by the number \(1\) and a quadratic nonresidue to be represented
by the number \(-1\), the multiplicative properties hold, for 
\[
\begin{split}
1\times1&=1,\\
1\times-1&=-1,\\
-1\times-1&=1.
\end{split}
\]
These multiplicative properties led Legendre to come up with a convenient symbol,
called the \emph{Legendre symbol},
to represent whether a certain number \(a\) is a quadratic residue modulo some
prime \(p\). It is defined as follows.
\[
\legendre{a}{p}=\begin{cases}
0&\text{if } a\equiv0\pmod{p}\\
1&\text{if } a \text{ is a quadratic residue modulo } p\\
-1&\text{if } a \text{ is a quadratic nonresidue modulo }p\\
\end{cases}
\]
For ease of display, when the Legendre symbol is to be used
inline, it will be denoted \ilegendre{a}{p}.
\section*{Problems}
\begin{enumerate}
\item Is the congruence \[x^2\equiv-1\pmod{17}\] solvable?
\item Prove the following two properties of the Legendre symbol:
\begin{enumerate}
\item \(\ilegendre{ab}{p}=\ilegendre{a}{p}\ilegendre{b}{p}\).
\item If \(a\equiv b\pmod{p}\) then \(\ilegendre{a}{p}=\ilegendre{b}{p}\).
\end{enumerate}
\item 
Wilson's Theorem states that \[(p-1)!\equiv-1\pmod{p}.\]
Use Wilson's Theorem and the results about quadratic residues developed above
to find for which primes \(-1\) is a quadratic residue and for which primes \(-1\)
is a quadratic nonresidue. \marginnote{This result is a very famous one and forms the first supplement to the \emph{Law of Quadratic Reciprocity}.} 
\end{enumerate}
\end{document}
