\documentclass[a4paper]{article}

\newcommand{\theterm}{3}
\newcommand{\theweek}{4}
\newcommand{\thepdftitle}{MEG 2023 Term \theterm\ Week \theweek\ Handout Solutions}
\newcommand{\thedisplaytitle}{Term \theterm\ Week \theweek\ Handout Solutions}

\title{{\thepdftitle}}
\author{Nathan Wong\and Tom Yan}
\date{2023}

\newcommand{\leg}[2]{\left(\frac{#1}{#2}\right)}
\newcommand{\ileg}[2]{(#1|#2)}

%\newcommand{\marginfn}[1]{\marginpar{\footnotemark}\footnotetext{#1}}
\newcommand{\marginnote}[1]{\marginpar{\footnotesize{#1}}}
\newcommand{\marginfnote}[1]{\footnotemark\marginpar{\footnotemark[\value{footnote}]\footnotesize{#1}}}
\usepackage{geometry}
%\geometry{a4paper,left=24.8mm,top=27.4mm,headsep=2\baselineskip,textwidth=107mm,marginparsep=8.2mm,marginparwidth=49.4mm,textheight=49\baselineskip,headheight=\baselineskip}
\geometry{a4paper,left=1in,top=1in,bottom=1in,headsep=2\baselineskip,textwidth=107mm,marginparsep=8.2mm,marginparwidth=49.4mm,textheight=49\baselineskip,headheight=\baselineskip}
\usepackage[bf,tiny]{titlesec}
%\usepackage{fancyhdr}
\usepackage{epigraph}
%\usepackage[indent=0pt,skip=10pt]{parskip}

\usepackage{amsmath}
\usepackage{amsthm}
\newtheorem{theorem}{Theorem}
\usepackage{amssymb}
\let\mathbbalt\mathbb

\usepackage{fontspec}
\usepackage{unicode-math}
\let\mathbb\mathbbalt

\newcommand{\naturals}{\mathbb{N}}
\newcommand{\reals}{\mathbb{R}}
\newcommand{\rationals}{\mathbb{Q}}
\newcommand{\integers}{\mathbb{Z}}

\usepackage[pdfusetitle]{hyperref}

\newcommand{\myquote}[2]{%
  \begin{quote}
    \emph{#1}
    \begin{flushright}---{#2}
    \end{flushright}
  \end{quote}}
\pagestyle{empty}
\begin{document}
\noindent Melbourne High School\\\
\noindent Maths Extension Group 2023\\\
\noindent \textbf{\thedisplaytitle}\\\
\section*{Quadratic Residues}
\begin{enumerate}
\item Enumerating the squares modulo \(17\), the squares are \(1,4,9,16,8,2,15,13.\)
Since \(16\equiv-1\) the congruence is solvable.
\item 
\begin{enumerate}
\item If \(a\) and \(b\) are both nonresidues, then \(\ileg{a}{p}=-1\) and \(\ileg{b}{p}=-1\).
But \(ab\) is a residue, so \(\ileg{ab}{p}=1=\ileg{a}{p}\ileg{b}{p}\). If both \(a\) and \(b\)
are residues, then \(\ileg{a}{p}=1\), \(\ileg{b}{p}=1\), and \(\ileg{ab}{p}=1\); and again
\(\ileg{ab}{p}=\ileg{a}{p}\ileg{b}{p}.\) For the last case, we may assume with loss of generality
that \(\ileg{a}{p}=-1\) and \(\ileg{b}{p}=1\); then \(\ileg{ab}{p}=-1=\ileg{a}{p}\ileg{b}{p}\) and
the proof is complete.
\item If \(\ileg{a}{p}=1\) and \(a\equiv x^2\pmod{p}\), then it follows that \(b\equiv x^2\) also
if \(a\equiv b\). Thus \(\ileg{a}{p}=\ileg{b}{p}\). If \(\ileg{a}{p}=-1\), suppose that there exists
some \(x\) such that \(b\equiv x^2\pmod{p}\). But since \(b\equiv a\) that would mean \(a\equiv x^2\),
which is a contradiction. Hence no such \(x\) exists, and so \(\ileg{a}{p}=-1=\ileg{b}{p}.\)
\end{enumerate}
\item From a property of the Legendre symbol we have
\[
\leg{-1}{p}=\leg{(p-1)!}{p}.
\]
The number \((p-1)!\) consists of the product of all the numbers from \(1\) to \(p-1\). We know
that amongst those numbers exactly \((p-1)/2\) of them are quadratic residues and exactly
\((p-1)/2\) of them are quadratic nonresidues. The product of the quadratic residues will
give a quadratic residue, so whether \((p-1)!\) is a residue or nonresidue depends only
on whether the product of the \((p-1)/2\) nonresidues gives a residue or nonresidue. Recall
that the product of two nonresidues gives a residue; hence if the number of nonresidues
is even, the resulting product will be a residue, and if the number of nonresidues is odd,
then the resulting product will be a nonresidue. When is \((p-1)/2\) even and when
is it odd? If it is to be even, then \(p-1\) must be a multiple of \(4\), and hence
\(p\equiv1\pmod{4}\). If it is to be odd, then \(p-1\) must an odd multiple of \(2\),
so that that \((p-1)/2\) is not further divisible by \(2\). Hence \(p-1=2(2k+1)\)
from which it follows that \(p=4k+3\), or \(p\equiv3\pmod{4}\). Hence
\[
\leg{-1}{p}=\begin{cases}
 1&\text{if }p\equiv1\pmod{4};\\
 -1&\text{if }p\equiv3\pmod{4}.
\end{cases}
\]
Observe that since \(\ileg{-1}{p}\) depends on whether the number of nonresidues \((p-1)/2\)
is even or odd, the result can be also be written more succinctly as 
\[
\leg{-1}{p}=(-1)^{(p-1)/2}.
\]
\end{enumerate}
\end{document}

