\documentclass{article}
\usepackage{graphicx} % Required for inserting images

\title{Expected Value MEG}
\author{Tom Yan}
\date{October 2023}

\begin{document}

\maketitle

\section{Introduction}
1. I have 12 addressed letters to mail, and 12 corresponding pre-addressed envelopes. For some wacky reason, I decide to put the letters into envelopes at random, one letter per envelope. What is the expected number of letters that get placed into their proper envelopes?  \\\\
2. I flip a coin 10 times. What is the expected number of pairs of consecutive tosses that comes up heads? (For example, the sequence THHTHHHTHH has 4 pairs of consecutive HH's).  \\\\
3. (HMMT 2006) At a nursery, 2006 babies sit in a circle. Suddenly, each baby randomly pokes either the baby to its left or to its right. What is the expected value of the number of unpoked babies?  \\\\
4. A $10$ digit binary number with four 1's is chosen at random. What is its expected value?\\\\
5. (IMO 1987/1) Let $p_n(k)$ be the number of permutations of the set $\{ 1, 2 \ldots , n \} , \; n \ge 1$, which have exactly $k$ fixed points. Prove that $$\sum_{k=0}^{n} k \cdot p_n (k) = n!.$$ (A \emph{fixed point} of a permutation in this case is an element $i$ such that $i$ is in the $i^{th}$ position of the permutation. For example, the permutation $(3,2,5,4,1)$ of $\{1,2,3,4,5\}$ has two fixed points: the 2 in the $2^{nd}$ spot and the 4 in the $4^{th}$ spot)

\newpage
\section{Solutions}
1. Each letter has a $\frac{1}{12}$ chance of being delivered to the right envelope. So since there are 12 letters, the expected number of letters that are placed to the proper envelope is $12 \times \frac{1}{12}=1$. \\\\
2. Define $X_i = 1$ if coin flip $i$ and $i+1$ are heads, and $X_i=0$ otherwise.  Then we seek $E(X_1 + X_2 +\ldots + X_9)$ Note that the chance that any pair of consecutive coin flips is $(\frac{1}{2})^2=\frac{1}{4}$, so $E(X_i)=\frac{1}{4}$ and $$E(X_1)+E(X_2)+\ldots+E(X_9)=9 \times \frac{1}{4}=\frac{9}{4}.$$
3. Number the babies $1,2, \ldots, 2006$. Define $X_i = 1$ if baby $i$ is poked and $X_i=0$ otherwise. Then we seek $E(X_1 + X_2 + \ldots X_{2006})$. Any baby has $\frac{1}{4}$ chance of being unpoked (if both its neighbours miss). Hence $E(X_i)=\frac{1}{4}$ for each $i$ and $$E(X_1 + X_2 + \ldots + X_{2006}) = E(X_1) + E(X_2) + \ldots E(X_{2006}) = 2006 \times \frac{1}{4} = \frac{1003}{2}. $$
4. Clearly the first digit has to be 1. Then each other digit has a $\frac{3}{9}=\frac{1}{3}$ chance of being 1. So the expected value is $2^9 + \frac{1}{3}(2^8+2^7+\ldots+1) = 2^9+\frac{1}{3}(2^9-1)=\frac{2047}{3}.$ \\\\
5. We expect there to be $\frac{1}{n} \times n = 1$ fixed point on average. But also, a permutation with $k$ fixed points occurs with chance of $\frac{p_n (k)}{n!}$ and such a permutation has $k$ fixed points, so the expected number of fixed points is $$\sum_{k=0}^{n} \frac{k \cdot p_n (k)}{n!}.$$ Which is also equal to 1. Thus $$\sum_{k=0}^{n} \frac{k \cdot p_n (k)}{n!}=1.$$ Multiplying both sides by $n!$ gives our desired result. 


\\\\
\end{document}
