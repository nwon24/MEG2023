\documentclass[a4paper]{article}

\newcommand{\theterm}{4}
\newcommand{\theweek}{2}
\newcommand{\thepdftitle}{MEG 2023 Term \theterm\ Week \theweek\ Handout}
\newcommand{\thedisplaytitle}{Term \theterm\ Week \theweek\ Handout}

\title{{\thepdftitle}}
\author{Nathan Wong\and Tom Yan}
\date{2023}

\newcommand{\leg}[2]{\left(\frac{#1}{#2}\right)}
\newcommand{\ileg}[2]{(#1|#2)}

%\newcommand{\marginfn}[1]{\marginpar{\footnotemark}\footnotetext{#1}}
\newcommand{\marginnote}[1]{\marginpar{\footnotesize{#1}}}
\newcommand{\marginfnote}[1]{\footnotemark\marginpar{\footnotemark[\value{footnote}]\footnotesize{#1}}}
\usepackage{geometry}
%\geometry{a4paper,left=24.8mm,top=27.4mm,headsep=2\baselineskip,textwidth=107mm,marginparsep=8.2mm,marginparwidth=49.4mm,textheight=49\baselineskip,headheight=\baselineskip}
\geometry{a4paper,left=1in,top=1in,bottom=1in,headsep=2\baselineskip,textwidth=107mm,marginparsep=8.2mm,marginparwidth=49.4mm,textheight=49\baselineskip,headheight=\baselineskip}
\usepackage[bf,tiny]{titlesec}
%\usepackage{fancyhdr}
\usepackage{epigraph}
%\usepackage[indent=0pt,skip=10pt]{parskip}

\usepackage{amsmath}
\usepackage{amsthm}
\newtheorem{theorem}{Theorem}
\usepackage{amssymb}
\let\mathbbalt\mathbb
\usepackage{diagbox}

\usepackage{fontspec}
\usepackage{unicode-math}
\let\mathbb\mathbbalt

\newcommand{\naturals}{\mathbb{N}}
\newcommand{\reals}{\mathbb{R}}
\newcommand{\rationals}{\mathbb{Q}}
\newcommand{\integers}{\mathbb{Z}}

\usepackage[pdfusetitle]{hyperref}

\newcommand{\myquote}[2]{%
  \begin{quote}
    \emph{#1}
    \begin{flushright}---{#2}
    \end{flushright}
  \end{quote}}
\pagestyle{empty}
\begin{document}
\noindent Melbourne High School\\\
\noindent Maths Extension Group 2023\\\
\noindent \textbf{\thedisplaytitle}\\\
\myquote{The excitement that a gambler feels when making a bet is equal
\marginnote{Not intended to glorify gambling.} to the amount he might win times the probability of winning it.}{B.~Pascal\footnote{As quoted by N.~Rose in \emph{Mathematical Maxims and Minims} (1988).}, ?? (??)}
\section*{Expected value problems}
\begin{enumerate}
\item I have 12 addressed letters to mail, and 12 corresponding pre-addressed envelopes. For some wacky reason, I decide to put the letters into envelopes at random, one letter per envelope. What is the expected number of letters that get placed into their proper envelopes?  
\item I flip a coin 10 times. What is the expected number of pairs of consecutive tosses that comes up heads? (For example, the sequence THHTHHHTHH has 4 pairs of consecutive HH's).  
\item At \marginnote{HMMT 2006} a nursery, 2006 babies sit in a circle. Suddenly, each baby randomly pokes either the baby to its left or to its right. What is the expected value of the number of unpoked babies?  
\item A $10$ digit binary number with four 1's is chosen at random. What is its expected value?
\item Let \marginnote{IMO 1987/1} $p_n(k)$ be the number of permutations of the set $\{ 1, 2 \ldots , n \} , \; n \ge 1$, which have exactly $k$ fixed points. Prove that $$\sum_{k=0}^{n} k \cdot p_n (k) = n!.$$ (A \emph{fixed point} of a permutation in this case is an element $i$ such that $i$ is in the $i^{th}$ position of the permutation. For example, the permutation $(3,2,5,4,1)$ of $\{1,2,3,4,5\}$ has two fixed points: the 2 in the $2^{nd}$ spot and the 4 in the $4^{th}$ spot)
\end{enumerate}
\pagebreak
\myquote{The fundamental theorem must certainly be regarded as one of
the most elegant of its type.}{C.~F.~Gauss\footnote{Translated from
the Latin by A.~A.~Clarke, 1965.}, \emph{Disquisitiones Arithmeticae} (1801)}
\section*{The Law of Quadratic Reciprocity}
Let's cast our minds back and recall the following table, a table
of values of odd primes \(p\) and \(q\), and the associated
Legendre symbol \(\ileg{q}{p}\).
\begin{center}
	\begin{tabular}{|c||c|c|c|c|c|c|c|c|c|c|c|c|}
		\hline
		\diagbox{\(p\)}{\(q\)}& \(3\)&\(5\)&\(7\)&\(11\)&\(13\)&\(17\)&\(19\)&\(23\)&\(29\)&\(31\)&\(37\)\\
		\hline\hline
		\(3\)& \(0\)&	\(-1\)&	\(1\)&	\(-1\)&	\(1\)&	\(-1\)&	\(1\)&	\(-1\)&	\(-1\)&	\(1\)&	\(1\)\\ \hline
		\(5\)&\(-1\)&	\(0\)&	\(-1\)&	\(1\)&	\(-1\)&	\(-1\)&	\(1\)&	\(-1\)&	\(1\)&	\(1\)&	\(-1\)\\ \hline
		\(7\)&\(-1\)&	\(-1\)&	\(0\)&	\(1\)&	\(-1\)&	\(-1\)&	\(-1\)&	\(1\)&	\(1\)&	\(-1\)&	\(1\)\\ \hline
		\(11\)&\(1\)&	\(1\)&	\(-1\)&	\(0\)&	\(-1\)&	\(-1\)&	\(-1\)&	\(1\)&	\(-1\)&	\(1\)&	\(1\)\\ \hline
		\(13\)&\(1\)&	\(-1\)&	\(-1\)&	\(-1\)&	\(0\)&	\(1\)&	\(-1\)&	\(1\)&	\(1\)&	\(-1\)&	\(-1\)\\ \hline
		\(17\)&\(-1\)&	\(-1\)&	\(-1\)&	\(-1\)&	\(1\)&	\(0\)&	\(1\)&	\(-1\)&	\(-1\)&	\(-1\)&	\(-1\)\\ \hline
		\(19\)&\(-1\)&	\(1\)&	\(1\)&	\(1\)&	\(-1\)&	\(1\)&	\(0\)&	\(1\)&	\(-1\)&	\(-1\)&	\(-1\)\\ \hline
		\(23\)&\(1\)&	\(-1\)&	\(-1\)&	\(-1\)&	\(1\)&	\(-1\)&	\(-1\)&	\(0\)&	\(1\)&	\(1\)&	\(-1\)\\ \hline
		\(29\)&\(-1\)&	\(1\)&	\(1\)&	\(-1\)&	\(1\)&	\(-1\)&	\(-1\)&	\(1\)&	\(0\)&	\(-1\)&	\(-1\)\\ \hline
		\(31\)&\(-1\)&	\(1\)&	\(1\)&	\(-1\)&	\(-1\)&	\(-1\)&	\(1\)&	\(-1\)&	\(-1\)&	\(0\)&	\(-1\)\\ \hline
		\(37\)&\(1\)&	\(-1\)&	\(1\)&	\(1\)&	\(-1\)&	\(-1\)&	\(-1\)&	\(-1\)&	\(-1\)&	\(-1\)&	\(0\)\\ \hline
	\end{tabular}
\end{center}
For the following discussion, ignore all the \(0\) entries in the table.
That is, when we say ``\(p\) runs through all the odd primes'' we are
eliding the further restriction that \(p\) does not equal the prime
we are considering, or \(q\) in the most general case.

One of the patterns is that some rows are the same as some columns.
\marginnote{By ``the same'' we mean the same sequence of \(1\)'s
and \(-1\)'s.}
For example, the column \(q=5\) is the same as the row \(p=5\),
as is the column \(q=13\) and row \(p=13\). 
On the other hand, the row \(q=3\) is not the same as the column
\(p=3\), and the same is true for \(q=7\) and \(p=7\).

The primes in the table for which their row and column are the
same are \(5,13,17,29,37\); the others are \(3,7,11,19,23,31.\)
What is the distinguishing feature of these two classes of primes?
Those in the first group are all of the form \(4k+1\) while
those in the second are all of the form \(4k+3\).\marginnote{Splitting
the primes into these two classes is common; for example, Fermat's
celebrated two square theorem states that all primes of the form
\(4k+1\) are the sum of two integer squares, while those of the form \(4k+3\)
are not the sum of two integer squares.}

Let's think about what this means. Let \(p\) be one of the primes
in the first group, those of the form \(4k+1\). For concreteness
suppose \(p=5\). The row \(p=5\)
consists of \(\ileg{q}{5}\), where \(q\) runs through the odd
primes \(3,5,7,11,\ldots\), while the column \(q=5\) consists of values of
\(\ileg{5}{p}\), but this time it is \(p\) that runs through
the odd primes. Since the row and column are the same, we conjecture
that 
\[\leg{5}{p}=\leg{p}{5}\]
for all odd primes \(p\). 
Similarly, since the row \(p=13\) and the column \(q=13\) are also
the same, it seems
that \[\leg{13}{p}=\leg{p}{13}.\]
There is nothing special about \(5\) or \(13\), other than that 
they are of the form \(4k+1\); therefore, we generalise our conjecture
by hypothesising that if \(q\equiv1\pmod{4}\) then
\[\leg{q}{p}=\leg{p}{q}.\]
One further step: we chose \(q\) to have the condition, but we
might as well have chosen \(p\); for if we just swap \(p\)
and \(q\) the condition becomes \(p\equiv1\pmod{4}\)
but the Legendre symbols don't change. 
Hence the revised guess:
if \(p\equiv1\pmod{4}\) or \(q\equiv1\pmod{4}\) then
\[\leg{q}{p}=\leg{p}{q}.\]
(Strictly speaking this tweak was not necessary, but it'll be useful
to have this form of the conjecture in our minds.)

The other case is not quite so easy. Let's take \(p=3\). Examining
the row \(p=3\) and the column \(q=3\), we notice that sometimes
\(\ileg{3}{p}=\ileg{p}{3}\) and sometimes \(\ileg{3}{p}=-\ileg{p}{3}\)---for
example, \(\ileg{5}{3}=\ileg{3}{5}=1\), but \(\ileg{7}{3}=1=-\ileg{3}{7}.\)
What is the distinction between the two cases? 

Looking at the table closely, we see that \(\ileg{3}{p}=\ileg{p}{3}\)
for \(p=5,13,17,29,37\) and \(\ileg{3}{p}=-\ileg{p}{3}\) for \(p=7,11,19,23,31\).
Like before, the first group of primes are \(1\) modulo \(4\), the second
\(3\) modulo \(4\).\marginnote{Again this distinction between the primes
based on their residue class modulo \(4\) pops up.}
Therefore it seems that if \(p\equiv3\pmod{4}\)
then \[\leg{3}{p}=-\leg{p}{3}.\]

Again, nothing was special about picking \(3\) here. We could have made the same conjecture
for any prime from the second class of primes, those
of the form \(4k+3\). Thus we conjecture that if
\emph{both} \(p\) and \(q\) are of
the form \(4k+3\) then \(\ileg{p}{q}=-\ileg{q}{p}.\) 
Looking for other examples in the table supports the conjecture, as the 
columns that are not the same as the rows are the columns for which
\(q\equiv3\pmod{4}\); and the differences between the row and column
occur only when \(p\equiv3\pmod{4}\) as well. Summarising, we have
made the following guess at the value of the general Legendre symbol
\(\ileg{p}{q}\) for odd primes \(p\) and \(q\):
\[
	\leg{p}{q}=\begin{cases}
		\leg{q}{p}&\text{if }p\equiv1\pmod{4}\text{ or }q\equiv1\pmod{4};\\
		-\leg{q}{p}&\text{if }p\equiv3\pmod{4}\text{ and }q\equiv3\pmod{4}.
	\end{cases}
\]
What we have just discovered is \emph{The Law of Quadratic
Reciprocity}.
Euler and Legendre both made but were unable to prove the conjecture.
Gauss supplied the first proof in 1801, and throughout his career he
added over half a dozen more; since then,
hundreds of proofs of the theorem have emerged, making quadratic
reciprocity one of the most proved
theorems in all of mathematics. \marginnote{Pythagoras's Theorem is
a contender, but quadratic reciprocity is hard to beat. Of course
the hundreds of proofs are not all completely different; indeed many are based
on the same ideas. However, some of the proofs have shown themselves
capable of generalisation to higher powers, showing that not all
mathematical proofs are equal. Those that connect significant and deep
mathematical ideas are, justifiably,
ranked as the greater achievements.}

Now for the proof (or at least one of the proofs).
We shall, in fact, show that the law
is a corollary of the conjecture of Euler's that we proved previously,
namely that if \(p\equiv q\pmod{4a}\) then \[\leg{a}{p}=\leg{a}{q}.\]
We break the proof into two cases: when \(p\equiv q\pmod{4}\), and
when \(p\equiv -q\pmod{4}.\) One of these two cases much be true
because \(p\) and \(q\), being odd primes, can only be congruent
to \(1\) or \(3\) modulo \(4\), and \(3\equiv-1\pmod{4}\).
\marginnote{We can show that Euler's conjecture can be deduced
from quadratic reciprocity (if we supply a different proof of the
latter, that is). Hence the two theorems are equivalent; both imply
the other.}

Suppose \(p\equiv q\pmod{4}.\) Then \(p-q=4a\) for some integer \(a\).
Note that this implies \(p\equiv q\pmod{4a}.\) Since \(p=4a+q\) and \(q\equiv0\pmod{q}\) we have
\[\leg{p}{q}=\leg{4a+q}{q}=\leg{4a}{q}.\]
By the multiplicative property of the Legendre symbol \(\ileg{4a}{q}=\ileg{4}{q}\ileg{a}{q}\); but clearly \(\ileg{4}{q}=1\) because \(4=2^2\). 
Therefore
\[\leg{p}{q}=\leg{a}{q}.\]
Let's apply the same process to \(\ileg{q}{p}\).
Since \(q=p-4a\), we have 
\[\leg{q}{p}=\leg{p-4a}{p}=\leg{-4a}{p}.\]
Again we may ignore the \(4\); hence
\[\leg{q}{p}=\leg{-a}{p}=\leg{-1}{p}\leg{a}{p}.\]
Comparing \(\ileg{q}{p}\) and \(\ileg{p}{q}\) and applying the result
\(\ileg{a}{p}=\ileg{a}{q}\) we deduce that
\[\leg{q}{p}=\leg{-1}{p}\leg{p}{q}.\]
We know that \(\ileg{-1}{p}=1\) if \(p\equiv1\pmod{4}\)
and \(\ileg{-1}{p}=-1\) if \(p\equiv3\pmod{4}\); consequently
\(\ileg{q}{p}=\ileg{p}{q}\) if \(p\equiv q\equiv1\pmod{4}\)
and \(\ileg{q}{p}=-\ileg{p}{q}\) if \(p\equiv q\equiv3\pmod{4}\).
This completes the first case.

The second case: let \(p\equiv -q\pmod{4}\). Note that in
this case we wish to prove \(\ileg{p}{q}=\ileg{q}{p}\), since
one of \(p\) and \(q\) must be congruent to \(1\) modulo \(4\).

If \(p\equiv-q\pmod{4}\) then
\(p+q=4a\) for some integer \(a\), from which it follows
that \(p\equiv-q\pmod{4a}.\) Euler's conjecture assures us
that once again \(\ileg{a}{p}=\ileg{a}{q}\).

Substitute \(p=4a-q\) into \(\ileg{p}{q}\) to obtain
\[\leg{p}{q}=\leg{4a-q}{q}=\leg{a}{q}.\]
Similarly, substitute \(q=4a-p\) into \(\ileg{q}{p}\) to obtain
\[\leg{q}{p}=\leg{4a-p}{p}=\leg{a}{p}.\]
Since \(\ileg{a}{p}=\ileg{a}{q}\)
\[\leg{p}{q}=\leg{q}{p}\]
and the proof is complete.

%There is, in fact, a beautiful way to symbolically represent
%quadratic reciprocity, thusly:
%\[
	%\leg{p}{q}\leg{q}{p}=(-1)^{\frac{p-1}{2}\frac{q-1}{2}}.
%\]
%The left-hand side is \(1\) whenever \(\ileg{p}{q}=\ileg{q}{p}\)
%and the right-hand side is \(1\) whenever either \((p-1)/2\)
%or \((q-1)/2\) is even. The left-hand side is \(-1\) whenever
%\(\ileg{p}{q}=-\ileg{q}{p}\) and the right-hand side is \(-1\)
%when both \((p-1)/2\) and \((q-1)/2\) are odd, and this
%is the case only when \(p\equiv q\equiv3\pmod{4}\).
%Most proofs of the law prove this equation directly; we shall
%discuss one of these proofs soon.

The law of reciprocity, along with what we know about
\(\ileg{-1}{p}\) and \(\ileg{2}{p}\), allows us to evaluate
any Legendre symbol much more easily than by brute force.
\marginnote{Implementing the evaluation of a Legendre symbol
as a computer program is an interesting exercise.}
For example, consider \(\ileg{34}{97}\). Since \(34=17\times2\),
\[\leg{34}{97}=\leg{2}{97}\leg{17}{97}.\]
The modulus \(97\) is \(1\) modulo \(8\), so \(\ileg{2}{97}=1\).
We can flip \(\ileg{17}{97}\) because \(17\equiv1\pmod{4}\); consequently
\[\leg{17}{97}=\leg{97}{17}=\leg{12}{17}.\]
Apply the same procedure for \(\ileg{12}{17}\); as \(12=3\times4\)
we compute \(\ileg{12}{17}=\ileg{3}{17}\ileg{4}{17}=\ileg{3}{17}.\)
Then \(\ileg{3}{17}=\ileg{17}{3}=\ileg{2}{3}=-1.\) Hence
\[\leg{34}{97}=\leg{2}{97}\leg{17}{97}=1\times(-1)=-1.\]
\end{document}

