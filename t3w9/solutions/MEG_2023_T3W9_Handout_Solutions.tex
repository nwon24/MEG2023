\documentclass[a4paper]{article}

\newcommand{\theterm}{3}
\newcommand{\theweek}{9}
\newcommand{\thepdftitle}{MEG 2023 Term \theterm\ Week \theweek\ Handout Solutions}
\newcommand{\thedisplaytitle}{Term \theterm\ Week \theweek\ Handout Solutions}

\title{{\thepdftitle}}
\author{Nathan Wong\and Tom Yan}
\date{2023}

\newcommand{\leg}[2]{\left(\frac{#1}{#2}\right)}
\newcommand{\ileg}[2]{(#1|#2)}

\newcommand{\floor}[1]{\lfloor#1\rfloor}
\newcommand{\bfloor}[1]{\left\lfloor#1\right\rfloor}

%\newcommand{\marginfn}[1]{\marginpar{\footnotemark}\footnotetext{#1}}
\newcommand{\marginnote}[1]{\marginpar{\footnotesize{#1}}}
\newcommand{\marginfnote}[1]{\footnotemark\marginpar{\footnotemark[\value{footnote}]\footnotesize{#1}}}
\usepackage{geometry}
%\geometry{a4paper,left=24.8mm,top=27.4mm,headsep=2\baselineskip,textwidth=107mm,marginparsep=8.2mm,marginparwidth=49.4mm,textheight=49\baselineskip,headheight=\baselineskip}
\geometry{a4paper,left=1in,top=1in,bottom=1in,headsep=2\baselineskip,textwidth=107mm,marginparsep=8.2mm,marginparwidth=49.4mm,textheight=49\baselineskip,headheight=\baselineskip}
\usepackage[bf,tiny]{titlesec}
%\usepackage{fancyhdr}
\usepackage{epigraph}
%\usepackage[indent=0pt,skip=10pt]{parskip}

\usepackage{amsmath}
\usepackage{amsthm}
\newtheorem{theorem}{Theorem}
\usepackage{amssymb}
\let\mathbbalt\mathbb

\usepackage{fontspec}
\usepackage{unicode-math}
\let\mathbb\mathbbalt

\newcommand{\naturals}{\mathbb{N}}
\newcommand{\reals}{\mathbb{R}}
\newcommand{\rationals}{\mathbb{Q}}
\newcommand{\integers}{\mathbb{Z}}

\usepackage[pdfusetitle]{hyperref}

\newcommand{\myquote}[2]{%
  \begin{quote}
    \emph{#1}
    \begin{flushright}---{#2}
    \end{flushright}
  \end{quote}}
\pagestyle{empty}
\begin{document}
\noindent Melbourne High School\\\
\noindent Maths Extension Group 2023\\\
\noindent \textbf{\thedisplaytitle}\\\
\section*{A Golden Conjecture}
\begin{enumerate}
\item Some of the rows are the same as some of
the columns. Which rows and which columns are
these? What is the significance of those particular
values of \(p\) and \(q\)?
	For example, the row \(p=3\) is not quite
		the same as the column \(q=3\), but the row
		\(p=5\) and the column \(q=5\) are the same.
		Thus, if the pattern continued for the rest of
		the table, it seems that
		\[\leg{5}{p}=\leg{p}{5}\]
		for all odd primes \(p\).
		What other such observations can be made
		from the table?
\item (Warning: what follows is tedious.)  
	We aim to find the parity of the number
	of multiples of \(5\) in the two intervals
		\[\left(\frac{p}{2},p\right)\]
		and 
		\[\left(\frac{3p}{2},2p\right).\]
	Divide both intervals by \(5\) and the question
		becomes how many integers there are in
		the intervals
		\[A=\left(\frac{p}{10},\frac{p}{5}\right)\]
		and 
		\[B=\left(\frac{3p}{10},\frac{2p}{5}\right).\]
		Let \(v_A\) be the number of integers in \(A\)
		and \(v_B\) be the number of integers in \(B\);
		naturally \(v=v_A+v_B\).
	First let \(p=20k+r\). 
		The intervals \(A\) and \(B\)
		become \[\left(2k+\frac{r}{10},4k+\frac{r}{5}\right)\]
		and 
		\[\left(6k+\frac{3r}{10},8k+\frac{2r}{5}\right).\]
		The work begins here. If we let \(r=1\)
		then \(v_A=4k-(2k+1)+1=2k\) and \(v_B=8k-(6k+1)+1=2k\); hence
		\(v=4k\), an even number.
		Similarly, if \(r=-1\) then \(v_A=(4k-1)-(2k)+1=2k\)
		and \(v_B=(8k-1)-6k+1=2k\); again \(v=4k\).

		One case down. The next one: \(r=\pm11\).
		If \(r=11\) then \(v_A=4k+2-(2k+2)+1=2k+1\)
		and \(v_B=8k+4-(6k+4)+1=2k+1\). Therefore
		\(v=4k+2\), which is even. If \(r=-11\) instead
		then \(v_A=v_B=2k-1\), so \(v=4k-2\).
		This shows that \(\ileg{5}{p}=1\) when 
		\(p\equiv\pm1,\pm11\pmod{20}.\)

		A similar procedure shows that \(v=4k+1\) is odd
		when \(r=3\), and similarly for \(r=-3\) and \(r=\pm7\),
		thus proving that \(\ileg{5}{p}=-1\) when \(p\equiv\pm3,\pm7\pmod{20}.\)

		Clearly this method is tedious and also not particularly
		illuminating. To illustrate the power of the theorem
		that hides in the table investigated in the first question,
		suppose that the pattern we observed about the row
		\(p=5\) being the same as the column \(q=5\) really
		does continue forever. That is, assume
		\[\leg{5}{p}=\leg{p}{5}.\]
		Now there is a much easier way of finding the primes
		\(p\) for which \(\ileg{5}{p}=1\) and those for
		which \(\ileg{5}{p}=-1\), because we just have
		to evaluate \(\ileg{p}{5}\) for the possible
		values of \(p\) modulo \(5\). For example,
		if \(p\equiv1\pmod{5}\) then 
		\[\leg{5}{p}=\leg{p}{5}=\leg{1}{5}=1.\]
		If \(p\equiv2\pmod{5}\) then
		\[\leg{5}{p}=\leg{p}{5}=\leg{2}{5}=-1\]
		from our earlier investigation into the quadratic
		character of \(2\) modulo odd primes.
		We need not bother with the case \(p\equiv4\)
		because clearly \(\ileg{4}{5}=1\). For
		the case \(p\equiv3\) we have
		\[\leg{5}{p}=\leg{3}{5}=\leg{2}{3}=-1.\]
		Therefore in about half the time we have found
		the primes for which \(5\) is a quadratic residue:
		\[\leg{5}{p}=\begin{cases}
			1&\text{ if }p\equiv\pm1\pmod{5};\\
			-1&\text{ if }p\equiv\pm2\pmod{5}.
		\end{cases}
		\]
		We can verify that this condition on \(p\)
		is the same as the condition on \(p\) modulo \(20\).

		So now what happens if we are asked to evaluate
		\[\leg{5}{3593}?\]
		Without knowing anything about the theory it seems
		insurmountable; how are to verify whether \(5\) is a square
		modulo \(3593\)?
		But, knowing what we now know, in about a second we realise that \(3593\equiv3\pmod{5}\)
		and hence \(\ileg{5}{3593}=\ileg{3}{5}=-1.\)
	\item Using a computer program we find that the first few primes
		for which \(\ileg{7}{p}=1\) are
		\[p=19,29,31,37,47,53,59,83,103\]
		and those for which \(\ileg{7}{p}=-1\) are
		\[11,13,17,23,41,43,61,67,71,73,79,89,97.\]
	\item Let \(P=(p-1)/2\) and
		\(Q=(q-1)/2\).
		On a Cartesian plane, consider the rectangle enclosed by 
		the the horizontal and vertical axes, the line \(y=qx/p\),
		the vertical line \(x=P\), and the horizontal line \(y=Q\).
		How many lattice points are in this rectangle, not counting
		those on the horizontal or vertical axes but counting
		those on the lines \(x=P\) and \(y=Q\)? (Lattice
		points are points in which both the \(x\) and \(y\)
		coordinates are integers.)
		The width of the rectangle is \(P\) and the height
		is \(Q\); therefore the number of lattice points
		is \(PQ=(p-1)(q-1)/4.\) But the number of lattice
		points is also the number of lattice points in the triangle
		below the line \(y=qx/p\) plus the number of lattice
		points in the triangle above the line \(y=qx/p\). This
		is true because there cannot be any lattice points
		on the line itself, since \(q\) and \(p\), being odd primes,
		are coprime.
		First consider the triangle below the line.

		How many lattice points on the vertical line \(x=s\)
		are in the triangle for some \(1\le s\le P\)? 
		(The line passes through the origin so we don't have
		to consider \(s=0\).)
		The \(y\) coordinate on the diagonal line at \(x=t\)
		is \(y=sq/p\); therefore the number of lattice
		points on that vertical line is \(\floor{sq/p}.\)
		If we run \(s\) through the values \(1\) to \(P\) and sum
		them, we will get the number of lattice points
		in the triangle below the line. Hence the number
		of lattice points below the line is 
		\[\sum_{s=1}^{P}\bfloor{\frac{sq}{p}}=S(q,p).\]
		Exactly the same argument shows that the number of lattice
		points in the upper triangle is \(S(p,q)\), because
		all we do is just swap \(p\) and \(q\) and the argument
		still holds. Therefore \(S(q,p)+S(p,q)\) is the number
		of lattice points in the rectangle, meaning
		\[S(q,p)+S(p,q)=\frac{(p-1)(q-1)}{4}.\]
\end{enumerate}
\end{document}

