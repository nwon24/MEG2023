\documentclass[a4paper]{article}

\newcommand{\theterm}{3}
\newcommand{\theweek}{9}
\newcommand{\thepdftitle}{MEG 2023 Term \theterm\ Week \theweek\ Handout}
\newcommand{\thedisplaytitle}{Term \theterm\ Week \theweek\ Handout}

\title{{\thepdftitle}}
\author{Nathan Wong\and Tom Yan}
\date{2023}

\newcommand{\leg}[2]{\left(\frac{#1}{#2}\right)}
\newcommand{\ileg}[2]{(#1|#2)}

%\newcommand{\marginfn}[1]{\marginpar{\footnotemark}\footnotetext{#1}}
\newcommand{\marginnote}[1]{\marginpar{\footnotesize{#1}}}
\newcommand{\marginfnote}[1]{\footnotemark\marginpar{\footnotemark[\value{footnote}]\footnotesize{#1}}}
\usepackage{geometry}
%\geometry{a4paper,left=24.8mm,top=27.4mm,headsep=2\baselineskip,textwidth=107mm,marginparsep=8.2mm,marginparwidth=49.4mm,textheight=49\baselineskip,headheight=\baselineskip}
\geometry{a4paper,left=1in,top=1in,bottom=1in,headsep=2\baselineskip,textwidth=107mm,marginparsep=8.2mm,marginparwidth=49.4mm,textheight=49\baselineskip,headheight=\baselineskip}
\usepackage[bf,tiny]{titlesec}
%\usepackage{fancyhdr}
\usepackage{epigraph}
%\usepackage[indent=0pt,skip=10pt]{parskip}

\usepackage{diagbox}

\usepackage{amsmath}
\usepackage{amsthm}
\newtheorem{theorem}{Theorem}
\usepackage{amssymb}
\let\mathbbalt\mathbb

\usepackage{fontspec}
\usepackage{unicode-math}
\let\mathbb\mathbbalt

\newcommand{\naturals}{\mathbb{N}}
\newcommand{\reals}{\mathbb{R}}
\newcommand{\rationals}{\mathbb{Q}}
\newcommand{\integers}{\mathbb{Z}}

\usepackage[pdfusetitle]{hyperref}

\newcommand{\myquote}[2]{%
  \begin{quote}
    \emph{#1}
    \begin{flushright}---{#2}
    \end{flushright}
  \end{quote}}
\pagestyle{empty}

\begin{document}
\noindent Melbourne High School\\\
\noindent Maths Extension Group 2023\\\
\noindent \textbf{\thedisplaytitle}\\\
\myquote{Stupors \marginnote{Keep this in mind if you ever fall into
a stupor because you're stuck on a problem (or have lost all of your sheep).}
however do not last for ever,
and Farmer Oak recovered from his.}{T.~Hardy, \emph{Far From the Madding Crowd}
(1874)}
\section*{AIMO Problems}
\begin{enumerate}
	\item Gaston \marginnote{AIMO 2019/1} and Jordan are two budding chefs who unfortunately always misread the cooking time required for a recipe. For example, if the required cooking time is written 1:32 meaning 1 hour and 32 minutes, Jordan reads it as 132 minutes while Gaston reads it as 1:32 hours. For one particular recipe, the difference between Jordan's and Gaston's misread time is exactly 90 minutes. What is the actually cooking time in minutes? 
\item Let \marginnote{AIMO 2014/3} $x$ and $y$ be positive integers that simultaneously satisfy the equations $xy=2048$ and $\frac{x}{y}-\frac{y}{x}=7.875$. Find $x$. 
\item There \marginnote{AIMO 2022/5} are $5$ lily pads on a pond, arranged in a circle. A frog can only jump from each lily pad to an adjacent lily pad on either side. How many ways are there for the frog to start on one of these lily pads, make 11 jumps, and end up where it started.
\item A \marginnote{AIMO 2019/7} triangle $PQR$ is to be constructed so that the perpendicular of $PQ$ cuts the side $QR$ at $N$ and the line $PN$ splits the angle $QPR$ into two angles, not necessarily of integer degrees, in the ratio 1:22. If $\angle QPR = p$ degrees, where $p$ is an integer, find the maximum value of $p$. 
\item Prove \marginnote{AIMO 2018/9} that 38 is the largest even integer that is $not$ the sum of two positive odd composite numbers.
\end{enumerate}
\pagebreak
\myquote{If we compared the Bernoullis to the Bach family, then Leonhard Euler is unquestionably the Mozart of mathematics\ldots}{E.~Maor, \emph{e: The Story of a Number} (1994)}
\section*{A Golden Conjecture}
Here's the table of values of \(p\), \(q\), and \(\ileg{p}{q}\)
from last time:
\begin{center}
	\begin{tabular}{|c||c|c|c|c|c|c|c|c|c|c|c|c|}
		\hline
		\diagbox{\(p\)}{\(q\)}& \(3\)&\(5\)&\(7\)&\(11\)&\(13\)&\(17\)&\(19\)&\(23\)&\(29\)&\(31\)&\(37\)\\
		\hline\hline
		\(3\)& \(0\)&	\(-1\)&	\(1\)&	\(-1\)&	\(1\)&	\(-1\)&	\(1\)&	\(-1\)&	\(-1\)&	\(1\)&	\(1\)\\ \hline
		\(5\)&\(-1\)&	\(0\)&	\(-1\)&	\(1\)&	\(-1\)&	\(-1\)&	\(1\)&	\(-1\)&	\(1\)&	\(1\)&	\(-1\)\\ \hline
		\(7\)&\(-1\)&	\(-1\)&	\(0\)&	\(1\)&	\(-1\)&	\(-1\)&	\(-1\)&	\(1\)&	\(1\)&	\(-1\)&	\(1\)\\ \hline
		\(11\)&\(1\)&	\(1\)&	\(-1\)&	\(0\)&	\(-1\)&	\(-1\)&	\(-1\)&	\(1\)&	\(-1\)&	\(1\)&	\(1\)\\ \hline
		\(13\)&\(1\)&	\(-1\)&	\(-1\)&	\(-1\)&	\(0\)&	\(1\)&	\(-1\)&	\(1\)&	\(1\)&	\(-1\)&	\(-1\)\\ \hline
		\(17\)&\(-1\)&	\(-1\)&	\(-1\)&	\(-1\)&	\(1\)&	\(0\)&	\(1\)&	\(-1\)&	\(-1\)&	\(-1\)&	\(-1\)\\ \hline
		\(19\)&\(-1\)&	\(1\)&	\(1\)&	\(1\)&	\(-1\)&	\(1\)&	\(0\)&	\(1\)&	\(-1\)&	\(-1\)&	\(-1\)\\ \hline
		\(23\)&\(1\)&	\(-1\)&	\(-1\)&	\(-1\)&	\(1\)&	\(-1\)&	\(-1\)&	\(0\)&	\(1\)&	\(1\)&	\(-1\)\\ \hline
		\(29\)&\(-1\)&	\(1\)&	\(1\)&	\(-1\)&	\(1\)&	\(-1\)&	\(-1\)&	\(1\)&	\(0\)&	\(-1\)&	\(-1\)\\ \hline
		\(31\)&\(-1\)&	\(1\)&	\(1\)&	\(-1\)&	\(-1\)&	\(-1\)&	\(1\)&	\(-1\)&	\(-1\)&	\(0\)&	\(-1\)\\ \hline
		\(37\)&\(1\)&	\(-1\)&	\(1\)&	\(1\)&	\(-1\)&	\(-1\)&	\(-1\)&	\(-1\)&	\(-1\)&	\(-1\)&	\(0\)\\ \hline
	\end{tabular}
\end{center}
It contains a pattern central to the problem
of evaluating \(\ileg{q}{p}\) for any odd prime \(q\).
For now, however, let's leave it, and instead
try to gain more numerical evidence.\marginnote{Have faith
that this detour is not an unwise one.}

So far, we have proved that \(\ileg{-1}{p}=1\)
for primes of the form \(4k+1\) and \(\ileg{-1}{p}=-1\)
for primes of the form \(4k-1\).
Using Gauss's Lemma we have also shown that
\(\ileg{2}{p}=1\) for primes of the form \(8k\pm1\)
and \(\ileg{2}{p}=-1\) for primes of the form \(8k\pm3\).
Recall that since 
\[
	\leg{a}{p}=\leg{q_1}{p}\leg{q_2}{p}\cdots\leg{q_r}{p}
\]
the remaining step is to evaluate \(\ileg{q}{p}\) for an odd
prime \(q\).\marginnote{Here `evaluate' means to find 
the conditions on \(p\) and \(q\) that determine whether
\(\ileg{q}{p}=1\) or \(\ileg{q}{p}=-1\).}

The natural way forward, given our route thus far, is to
evaluate \(\ileg{3}{p}\), \(\ileg{5}{p}\), and so on.
Perhaps if we solve enough special cases a pattern will
appear. Let's tackle \(\ileg{3}{p}\).

As before, the weapon to use is Gauss's Lemma.
The question: how many numbers in the set
\[
	\left\{3,6,9,\ldots,\frac{3(p-1)}{2}\right\}
\]
are negative when reduced into the range \((-p/2,p/2)\)?

Divide the numbers in the above set into the three intervals
\[
	\left(0,\frac{p}{2}\right), \left(\frac{p}{2},p\right),
	\left(p, \frac{3p}{2}\right).
\]
The numbers in the first interval are positive and less than
\(p/2\); hence they remain unchanged when reduced
to be in the interval \((-p/2,p/2)\). All the numbers
in the second interval will become negative because they
are greater than \(p/2\) but less than \(p\).
The numbers in the third interval, being greater than \(p\)
but less than \(3p/2\), will be reduced to be between \(0\)
and \(p/2\) and therefore remain positive. 
Hence the number of negative signs is the number of numbers
in the second interval, \((p/2, p)\).

So there is some condition on \(p\), most likely to do
with its residue class modulo some number, that determines
whether the number of numbers in the interval \((p/2,p)\)
is even or odd, and that is the same condition that determines
whether \(\ileg{3}{p}=1\) or \(\ileg{3}{p}=-1.\)

Now is the time for the numerical evidence. By simply
evaluating \(\ileg{3}{p}\) for many values of \(p\), we 
find that \(\ileg{3}{p}=1\) for 
\[
	p=11,13,23,37,47,59,61,71,73,83,97,107
\]
and \(\ileg{3}{p}=-1\) for 
\[
	p=5,7,17,19,29,31,41,43,53,67,79,89,101.
\]
If we continue trying out values of \(p\), we notice
that the numbers in the first list are of the form 
\(12k\pm1\) while the numbers in the second are of
the form \(12k\pm5\).
So we conjecture that \(\ileg{3}{p}=1\) for \(p=12k\pm1\)
and \(\ileg{3}{p}=-1\) for \(p=12k\pm5\).

To prove the conjecture, let \(p=12k+r\), where
\(r=\pm1,\pm5\).\marginnote{Note that these values
of \(r\) are the only values for which \(p\) could be prime.
We leave out \(p=2\) because \(\ileg{3}{2}=\ileg{1}{2}\),
which is trivial.}
We wish to know how many numbers \(a\) satisfy
\[ \frac{p}{2}<3a<p,\]
or
\[ \frac{12k+r}{2}<3a<12k+r.\]
Dividing through by \(3\) and splitting the fraction
yield
\[2k+\frac{r}{6}<a<4k+\frac{r}{3}.\]
When \(r=1\), the inequality becomes
\[2k+\frac{1}{6}<a<4k+\frac{1}{3}.\]
If \(a>2k+1/6\), then \(a\ge 2k+1\) because
\(a\) is an integer.
Similarly if \(a<4k+1/3\), then \(a\le4k\).
Therefore the inequality is equivalent to
\[2k+1\le a\le4k.\]
The number of numbers in this interval
is \(4k-(2k+1)+1=2k\), which is even, and hence
\(\ileg{3}{p}=1\).\marginnote{How many numbers
are between \(5\) and \(9\) inclusive of both?
Answer: \(5,6,7,8,9\), so \(5\), or \(9-5+1\).}
Similarly, when \(r=-1\)
the interval becomes 
\[2k-\frac{1}{6}<a<4k-\frac{1}{3}.\]
Using the same reasoning as before,
the number of numbers in this interval
is \(4k-1-2k+1=2k\); therefore \(\ileg{3}{p}=1\)
when \(r=-1\) too.

The second case is much the same as the first.
When \(r=5\) the inequality becomes 
\[2k+\frac{5}{6}<a<4k+\frac{5}{3}\]
and when \(r=-5\) the inequality becomes
\[2k-\frac{5}{6}<a<4k-\frac{5}{3}.\]
In the former the number of values of \(a\)
that satisfy the inequality is \(4k+1-(2k+1)+1=2k+1\);
in the latter case it is \(4k-2-(2k)+1=2k-1\).
In both cases the number is odd and hence
\(\ileg{3}{p}=-1\).

We have successfully solved another special
case of the Legendre symbol: the case \(a=3\).
Is there a pattern we can spot? 
With only two cases completed, \(a=2\) and \(a=3\),
\marginnote{The case \(a=-1\) is special because
\(-1\) is not prime; we need it so evaluate \(\ileg{a}{p}\)
when \(a<0\).}
it might seem hard to find a pattern.
However two things are plain.
Firstly, in the case \(a=2\) the condition on \(p\)
is its residue class modulo \(8\); for \(a=3\) it
is its residue class modulo \(12\).
Secondly, \(\ileg{2}{p}\) is the same for \(p=8k+r\)
and \(8k-r\) for \(r=1,3\), just as \(\ileg{3}{p}\)
is the same for \(p=12k+r\) and \(12k-r\) for \(r=1,5\).
From these two observations, we may guess, perhaps
tentatively, that \(\ileg{a}{p}\) is dependent
on the modulo class of \(p\) modulo \(4a\), and
that the quadratic character of \(a\) modulo \(p=4a-r\)
is the same as that of \(p=4a+r\).

Let's try another case: \(a=5\).
Using a computer, the first few values of \(p\)
for which \(\ileg{5}{p}=1\) are
\[ p=19,29,31,41,59,61,71,79,89,101,109,131,139,149,151\]
and those for which \(\ileg{5}{p}=-1\) are
\[ p=7,13,17,23,37,43,47,53,67,73,83,97,103,107,113,127,137.\]
From experience, it is likely that the pattern has
something to do with the residue class of \(p\)
modulo a multiple of \(4\); it was \(8\) for \(a=2\)
and \(12\) for \(a=3\).
Naturally we try \(20\) for \(a=5\). And, indeed, if
we examine the values of \(p\) closely, it seems
that \(\ileg{5}{p}=1\) for \(p=20k\pm1\) or \(p=20k\pm11\)
and \(\ileg{5}{p}=-1\) for \(p=20k\pm3\) or \(p=20k\pm13\).
The proof of this is far less exciting than the observation
because of the number of different cases, but it is left
as an exercise for the devoted.

Note also how this preceding observation is consistent
with our earlier conjecture about the quadratic character
of \(a\) being the same for \(p=4ak+r\) as it is for
\(p=4ak-r\).
When new numerical data agrees with previous conjectures,
it is more evidence that those conjectures have a grain
of truth to them.

From our guesswork for the case \(a=5\) something else
jumps out as being a possible pattern: if \(p=4ak+r\)
the same values of \(r\) seem to yield the same
values for the Legendre symbol.
That is, all of \(\ileg{2}{p}\), \(\ileg{3}{p}\), and \(\ileg{5}{p}\)
equal \(1\) when \(r=\pm1\), and \(\ileg{2}{p}\) and \(\ileg{5}{p}\)
both equal \(-1\) when \(r=\pm3\).
Of course there is not enough data here, but, perhaps,
there is an inkling of gold just beneath the surface.
\section*{Problems}
\begin{enumerate}
	\item If no patterns are apparent in the table on page 2, try again.
	\item Prove our conjecture about \(\ileg{5}{p}\).
	\item Tackle the case \(a=7\). Do our two hypotheses
		hold?
	\item Let \marginnote{This is a hard problem.}
		\[S(q,p)=\sum_{s=1}^{(p-1)/2}\left\lfloor\frac{sq}{p}\right\rfloor.\]
		Prove that
		\[ S(q,p)+S(p,q)=\frac{(p-1)(q-1)}{4}.\]
\end{enumerate}
\end{document}
