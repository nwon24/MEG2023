\documentclass[a4paper,12pt,addpoints]{exam}

\usepackage{amsmath}
\usepackage[a4paper,pass]{geometry}
\begin{document}
\pagestyle{empty}
\section*{Year 9}
\begin{questions}


\question[2] Three consecutive numbers sum to \(141\). What are the numbers?

\droptotalpoints

\question[3]
There are twelve lockers, numbered from \(110\) to \(121\). The keys
to these twelve lockers are numbered \(1\) to \(12\). 
Each locker number is divisible by the number on its key.

Determine the key number for each locker.

\droptotalpoints


\question  In an AFL game, a goal is worth \(6\) points and a behind is worth \(1\) point.
\begin{parts}
    \part[1] How many points does a team have if they score \(4\) goals and \(8\) behinds?
    \begin{solution}
        \(32\)
    \end{solution}
    \part[2] How many different ways are there to score some number of goals and some number
    of behinds such that the score is the product of the number of goals and the number
    of behinds?
    \begin{solution}
        The problem boils down to finding solutions to the equation \(6x+y=xy\). There are \(5\) solutions for
        nonnegative integers \(x\) and \(y\) but accept if trivial solution \((0,0)\) is not given.
    \end{solution}
    \part[2] By considering a graph, show that you have found all the ways that this could occur.
    \begin{solution}
        Rearrange the equation to get \[y=\frac{6x}{x-1}.\] Via s simple algebraic manipulation
        we get \[y=6+\frac{6}{x-1}.\] The \(4\) solutions obtained through enumeration are
        \((2,12)\), \((3,9)\), \((4,8)\), \((7,7)\). For \(x>7\) the denominator of the
        fraction becomes greater than \(6\) and therefore \(y\) cannot be an integer.
        Hence there are no solutions for \(x>7\) and thus we have found all the solutions.
    \end{solution}
    
  \end{parts}
  \droptotalpoints


\question 
\begin{parts}
\part[2] What is the area of the largest square that can fit inside a unit circle?
\begin{solution}
    The diagonal of the square is the diameter of the circle, which is \(2\). Therefore
    the side length of the square is \(\sqrt2\) units, meaning the area is \(2\) square units.
\end{solution}
\part[2] What is the area of the largest circle that can fit inside a unit square?
\begin{solution}
    The radius of the circle is half the side length of the square, so the radius is \(1/2\).
    The area of the circle is therefore \(\pi(1/2)^2=\pi/4\).
\end{solution}
\part[3] A triangle is placed inside a unit square so that of its vertices either lie on the square's
edges or inside the square. What areas are possible for this triangle?
\begin{solution}
It is possible for the triangle to have an area of \(1/2\), as that occurs when all three
vertices are three vertices of the square. Therefore it can plainly have any area between
\(0\) and \(1/2\). Can it have an area that is greater than \(1/2\)? The answer is no. 
\end{solution}
\part[2] What is the side length of the largest square that can fit inside an equilateral triangle that has side length 1 unit?
\begin{solution}
    Let \(x\) be the sidelength of the square. By similar triangles we can deduce that \[\frac{2x}{1-x}=\sqrt3.\]
    Solving this equation gives \(x=2\sqrt3-3.\)
\end{solution}

\end{parts}
\droptotalpoints


\question[3] What is the volume of a regular tetrahedron with side length \(1\) unit?
\begin{solution}
    The volume of a pyramid is \(1/3\) multiplied by the area of its base times its height. Using
    Pythagoras' Theorem we can deduce that the height of a unit tetrahedron is \(\sqrt{2/3}\). 
    The area of the base of the tetrahedron is the area of an equilateral triangle with
    sidelength \(1\), which is \((1/2)(\sqrt3/2)=\sqrt3/4\). Therefore the area of the
    unit regular tetrahedron is \((1/3)\times(\sqrt3/4)\times\sqrt{2/3}=\sqrt2/12.\)
\end{solution}
\droptotalpoints


\question
\begin{parts}
    \part[1] How many positive integer solutions are there to the equation \(2x+3y=25\)?
    \begin{solution}
        Enumerating the cases and counting gives \(4\) solutions.
    \end{solution}
    \part[2] How many integer solutions are there to the equation \(2x+3y=25\)?
    \begin{solution}
        There are infinitely many integer solutions.
    \end{solution}
    \part[2] How many integer solutions are there to the equation \(51x+24y=17\)?
    \begin{solution}
        There are no integer solutions because every number of the form \(51x+24y\) is a multiple
        of the greatest common divisor of \(51\) and \(24\), which is \(3\). 
    \end{solution}
    \part[1] Consider the expression \(9x+15y\). Put in as many integer values of \(x\) and \(y\) as you can.
    Which number are all the resulting numbers multiples of?
    \begin{solution}
        The resulting numbers are multiples of \(3\).
    \end{solution}
    \part[2] When does the equation \(ax+by=c\) always have integer solutions?
    \begin{solution}
        The equation has integer solutions only when \(c\) is a multiple of the greatest common
        divisor of \(a\) and \(b\).
    \end{solution}
  \end{parts}
\droptotalpoints  

\question 
\begin{parts}
\part[2] When is the sum of 2 consecutive numbers divisible by 2?
\begin{solution}
    Never, because \(n+(n+1)=2n+1\) is odd.
\end{solution}
\part[2] When is the sum of 3 consecutive numbers divisible by 3?
\begin{solution}
    Always, as \(n+(n+1)+(n+2)=3n+3.\)
\end{solution}
\part[2] When is the sum of 4 consecutive numbers divisible by 4?
\begin{solution}
    Never, as \(n+(n+1)+(n+2)+(n+3)=4n+10\equiv2\pmod{4}\).
\end{solution}
\part[2] When is the sum of 5 consecutive numbers divisible by 5?
\begin{solution}
    Always, as \(n+(n+1)+(n+2)+(n+3)+(n+4)=5n+10\).
\end{solution}
\part[2] When is the sum of \(n\) consecutive numbers divisible by \(n\)?
\begin{solution} 
    The sum of \(n\) consecutive numbers starting from \(k\) is \(kn\) plus the \(n-1\)th triangular number, so it
    is \[n\left(k+\frac{n-1}{2}\right).\] The resulting number is divisible by \(n\) if \((n-1)/2\) is an integer,
    and that is when \(n\) is odd.
\end{solution}
\end{parts}
\droptotalpoints


\question
\begin{parts}
    \part[1] Two lines can divide the plane into at most how many regions?
    \begin{solution}
        \(4\)
    \end{solution}
    \part[1] Three lines can divide the plane into at most how many regions?
    \begin{solution}
        \(7\) (If you try it on paper, you will see that the third line can only intersect
        the existing \(2\) lines in at most \(2\) places, resulting in \(3\) new regions.
        Therefore the number of regions is \(4+3=7\).)
    \end{solution}
    \part[1] If we add a fourth line, what is the maximum number of intersection
    points that line can make with the existing lines?
    \begin{solution}
        \(3\)
    \end{solution}
    \part[1] Four lines can divide the plane into at most how many regions?
    \begin{solution}
        \(11\)
    \end{solution}
    \part[3] What is the maximum number of regions that \(n\) lines can divide the plane into?
    \begin{solution}
        \(1+n(n+1)/2\)
    \end{solution}
  \end{parts}
\droptotalpoints  


\question[3] Prove it is possible to pair up the numbers $0, 1, 2, 3, \ldots , 61$ in such a way that when we sum each pair, the product of the 31 numbers we get is a perfect fifth power.
\begin{solution}
    Pair 0 with 1, and $k$ with $63-k$ for all $2 \leq k \leq 31 $. This would result in a product equal to $1 \times 63^{30} = (63^6)^5$ which is a perfect $5^{\text{th}}$ power.
\end{solution}
\droptotalpoints

\end{questions}

Total: \numpoints
\end{document}
