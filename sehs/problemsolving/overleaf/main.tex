\documentclass[a4paper,12pt,addpoints]{exam}
% Exam class so that we can put points next to the questions if we want them to be weighted differently.
\usepackage{graphicx} % Required for inserting images
\usepackage{amsmath}
% Uncomment to print document with solutions to competition quesitons.
%\printanswers

\begin{document}
\section*{Mixed School Problem Solving Questions}
\begin{questions}
  \question Define the “reverse” of a number to be the same digits written backwards. For example, the reverse of 135 is 531, and the reverse of 4630 is 0364. 

Prove that the sum of any four-digit number and its reverse is divisible by 11. 

Hint: Represent the 4-digit number as $ABCD$. 
    \question Suppose you have an analogue clock with an hour hand and a minute hand. If it is currently midnight, how long will it take for the two hands to form a $90^{\circ}$ angle? 

    \question How many triangles are there in the image below?
    \[  \includegraphics[scale=0.3]{how-many-triangles-do-you-see-1.png}\]

      \question 
      \begin{parts}
         \part Find the smallest integer \(x\) such that \(1800x\) is a perfect square. 
         \part What is the general strategy to find the smallest integer $x$ such that $N \times x$ is a perfect square for positive integer $N$?
      \end{parts}

      \question Why does the set of rational numbers on the interval \((0,1)\) not have a least element?

      \question
      \begin{parts}
          \part Must a quadratic pass through the $x$ axis? Why or why not?
          \part Must a cubic pass through the $x$ axis? Why or why not?
      \end{parts}

      \question  The scales at Jasper’s house only work if two people weigh themselves at
    the same time. He has four people in his house who weigh themselves in
    pairs obtaining 89kg, 105kg, 112kg, 113kg, 120kg and 136kg. What is the
    difference between the weights of the heaviest and lightest people in the
    house?

      
\end{questions}
  

    

\pagebreak

1. Express the four digit number as $ABCD$, then $$1000A + 100B + 10C + D + 1000D + 100C + 10B + A = 11(91A + 10B + 10C + 91D).$$ 
2. Solve $6x-0.5x=90$ \\
3. Counting gives 27. \\
4. a) $x=2$ \\
5. Let $\frac{p}{q}$ be the smallest element, then $\frac{p}{q+1} < \frac{p}{q}$ and we have a contradiction. \\
6. a) no b) yes\\
7. Let the weights be $a \le b \le c \le d$. Then $a + b = 89$ since it
is the smallest sum, $b + d = 120$ since it is the second largest sum. Hence
$d - a = b + d - (a + b) = 120 - 89 = 31kg$.
(The possible weights are 41, 48, 64, 72 or 40.5, 48.5, 64.5, 71.5.)


%\pagebreak
%\section*{Competitive Problem Solving Questions}
%\begin{questions}
%    \question Prove it possible to pair up the numbers $0, 1, 2, 3, \ldots , 61$ in such a way that when we sum each pair, the product of the 31 numbers we get is a perfect fifth power.

%    \question What is the area of the largest square that can fit inside a unit circle?
%    \question What is the area of the largest circle that can fit inside a unit square? 
 %   \question Find the smallest integer \(x\) such that \(1800x\) is a perfect square.

 %   \question We know that \(\sqrt2\) is irrational but it is unclear whether \(\sqrt2^{\sqrt2}\) is
%    rational or irrational. Show that there exist irrational numbers \(a\) and \(b\) such that
 %   \(a^b\) is rational.
  %  \question Prove that the sum of three consecutive integers is always divisible by \(3\).
    %\question Is the sum of four consecutive integers ever divisible by \(4\)?
% Some of these problems are to be developed further into multi parts, and require more explanation
% and discussion than getting a solution. Might be harder to mark but will encourage more thinking.
%\question Why does the set of rational numbers on the interval \((0,1)\) not have a least element?
%\question Is it possible to have a biased six-sided die such that when two of them are rolled,
%all possible sums from \(2\) to \(12\) are equally likely?
%\question What is the area of the largest triangle that can fit inside a unit square?


%\end{questions}

%\pagebreak
%1. Pair 0 with 1, and $k$ with $63-k$ for all $2 \leq k \leq 31 $. This would result in a product equal to $1 \times 63^{30} = (63^6)^5$ which is a perfect $5^{\text{th}}$ power.  

% Ask questions more in the style of "is it possible to...?" 
% Will reuse some of the problems between the year levels to make it a bit easier. 
% They're going in Year 9 for now, but they'll be moved. 

\pagebreak
\section*{Year 9}
\begin{questions}


\question  In an AFL game, a goal is worth \(6\) points and a behind is worth \(1\) point.
\begin{parts}
    \part[1] How many points does a team have if they score \(4\) goals and \(8\) behinds?
    \begin{solution}
        \(32\)
    \end{solution}
    \part[2] How many different ways are there to score some number of goals and some number
    of behinds such that the score is the product of the number of goals and the number
    of behinds?
    \begin{solution}
        The problem boils down to finding solutions to the equation \(6x+y=xy\). There are \(5\) solutions for
        nonnegative integers \(x\) and \(y\) but accept if trivial solution \((0,0)\) is not given.
    \end{solution}
    \part[2] By considering a graph, show that you have found all the ways that this could occur.
    \begin{solution}
        Rearrange the equation to get \[y=\frac{6x}{x-1}.\] Via s simple algebraic manipulation
        we get \[y=6+\frac{6}{x-1}.\] The \(4\) solutions obtained through enumeration are
        \((2,12)\), \((3,9)\), \((4,8)\), \((7,7)\). For \(x>7\) the denominator of the
        fraction becomes greater than \(6\) and therfore \(y\) cannot be an integer.
        Hence there are no solutions for \(x>7\) and thus we have found all the solutions.
    \end{solution}
    
\end{parts}


\question 
\begin{parts}
\part[2] What is the area of the largest square that can fit inside a unit circle?
\begin{solution}
    The diagonal of the square is the diameter of the circle, which is \(2\). Therefore
    the side length of the square is \(\sqrt2\) units, meaning the area is \(2\) square units.
\end{solution}
\part[2] What is the area of the largest circle that can fit inside a unit square?
\begin{solution}
    The radius of the circle is half the side length of the square, so the radius is \(1/2\).
    The area of the circle is therefore \(\pi(1/2)^2=\pi/4\).
\end{solution}
\part[3] A triangle is placed inside a unit square so that of its vertices either lie on the square's
edges or inside the square. What areas are possible for this triangle?
\begin{solution}
It is possible for the triangle to have an area of \(1/2\), as that occurs when all three
vertices are three vertices of the square. Therefore it can plainly have any area between
\(0\) and \(1/2\). Can it have an area that is greater than \(1/2\)? The answer is no. 
\end{solution}
\part[2] What is the side length of the largest square that can fit inside an equilateral triangle that has side length 1 unit?
\begin{solution}
    Let \(x\) be the sidelength of the square. By similar triangles we can deduce that \[\frac{2x}{1-x}=\sqrt3.\]
    Solving this equation gives \(x=2\sqrt3-3.\)
\end{solution}

\end{parts}


\question[3] What is the volume of a regular tetrahedron with side length \(1\) unit?
\begin{solution}
    The volume of a pyramid is \(1/3\) multiplied by the area of its base times its height. Using
    Pythagoras' Theorem we can deduce that the height of a unit tetrahedron is \(\sqrt{2/3}\). 
    The area of the base of the tetrahedron is the area of an equilateral triangle with
    sidelength \(1\), which is \((1/2)(\sqrt3/2)=\sqrt3/4\). Therefore the area of the
    unit regular tetrahedron is \((1/3)\times(\sqrt3/4)\times\sqrt{2/3}=\sqrt2/12.\)
\end{solution}


\question
\begin{parts}
    \part[1] How many positive integer solutions are there to the equation \(2x+3y=25\)?
    \begin{solution}
        Enumerating the cases and counting gives \(4\) solutions.
    \end{solution}
    \part[2] How many integer solutions are there to the equation \(2x+3y=25\)?
    \begin{solution}
        There are infinitely many integer solutions.
    \end{solution}
    \part[2] How many integer solutions are there to the equation \(51x+24y=17\)?
    \begin{solution}
        There are no integer solutions because every number of the form \(51x+24y\) is a multiple
        of the greatest common divisor of \(51\) and \(24\), which is \(3\). 
    \end{solution}
    \part[1] Consider the expression \(9x+15y\). Put in as many integer values of \(x\) and \(y\) as you can.
    Which number are all the resulting numbers multiples of?
    \begin{solution}
        The resulting numbers are multiples of \(3\).
    \end{solution}
    \part[2] When does the equation \(ax+by=c\) always have integer solutions?
    \begin{solution}
        The equation has integer solutions only when \(c\) is a multiple of the greatest common
        divisor of \(a\) and \(b\).
    \end{solution}
\end{parts}

\question 
\begin{parts}
\part[2] When is the sum of 2 consecutive numbers divisible by 2?
\begin{solution}
    Never, because \(n+(n+1)=2n+1\) is odd.
\end{solution}
\part[2] When is the sum of 3 consecutive numbers divisible by 3?
\begin{solution}
    Always, as \(n+(n+1)+(n+2)=3n+3.\)
\end{solution}
\part[2] When is the sum of 4 consecutive numbers divisible by 4?
\begin{solution}
    Never, as \(n+(n+1)+(n+2)+(n+3)=4n+10\equiv2\pmod{4}\).
\end{solution}
\part[2] When is the sum of 5 consecutive numbers divisible by 5?
\begin{solution}
    Always, as \(n+(n+1)+(n+2)+(n+3)+(n+4)=5n+10\).
\end{solution}
\part[2] When is the sum of \(n\) consecutive numbers divisible by \(n\)?
\begin{solution} 
    The sum of \(n\) consecutive numbers starting from \(k\) is \(kn\) plus the \(n-1\)th triangular number, so it
    is \[n\left(k+\frac{n-1}{2}\right).\] The resulting number is divisible by \(n\) if \((n-1)/2\) is an integer,
    and that is when \(n\) is odd.
\end{solution}
\end{parts}


\question
\begin{parts}
    \part[1] Two lines can divide the plane into at most how many regions?
    \begin{solution}
        \(4\)
    \end{solution}
    \part[1] Three lines can divide the plane into at most how many regions?
    \begin{solution}
        \(7\) (If you try it on paper, you will see that the third line can only intersect
        the existing \(2\) lines in at most \(2\) places, resulting in \(3\) new regions.
        Therefore the number of regions is \(4+3=7\).)
    \end{solution}
    \part[1] If we add a fourth line, what is the maximum number of intersection
    points that line can make with the existing lines?
    \begin{solution}
        \(3\)
    \end{solution}
    \part[1] Four lines can divide the plane into at most how many regions?
    \begin{solution}
        \(11\)
    \end{solution}
    \part[3] What is the maximum number of regions that \(n\) lines can divide the plane into?
    \begin{solution}
        \(1+n(n+1)/2\)
    \end{solution}
\end{parts}


\question[3] Prove it possible to pair up the numbers $0, 1, 2, 3, \ldots , 61$ in such a way that when we sum each pair, the product of the 31 numbers we get is a perfect fifth power.
\begin{solution}
    Pair 0 with 1, and $k$ with $63-k$ for all $2 \leq k \leq 31 $. This would result in a product equal to $1 \times 63^{30} = (63^6)^5$ which is a perfect $5^{\text{th}}$ power.
\end{solution}


\end{questions}

\pagebreak
\section*{Year 10}

\begin{questions}

\question 
\begin{parts}
\part[2] When is the sum of 2 consecutive numbers divisible by 2?
\begin{solution}
    Never, because \(n+(n+1)=2n+1\) is odd.
\end{solution}
\part[2] When is the sum of 3 consecutive numbers divisible by 3?
\begin{solution}
    Always, as \(n+(n+1)+(n+2)=3n+3.\)
\end{solution}
\part[2] When is the sum of 4 consecutive numbers divisible by 4?
\begin{solution}
    Never, as \(n+(n+1)+(n+2)+(n+3)=4n+10\equiv2\pmod{4}\).
\end{solution}
\part[2] When is the sum of 5 consecutive numbers divisible by 5?
\begin{solution}
    Always, as \(n+(n+1)+(n+2)+(n+3)+(n+4)=5n+10\).
\end{solution}
\part[2] When is the sum of \(n\) consecutive numbers divisible by \(n\)?
\begin{solution} 
    The sum of \(n\) consecutive numbers starting from \(k\) is \(kn\) plus the \(n-1\)th triangular number, so it
    is \[n\left(k+\frac{n-1}{2}\right).\] The resulting number is divisible by \(n\) if \((n-1)/2\) is an integer,
    and that is when \(n\) is odd.
\end{solution}
\end{parts}

\question[3] Prove it possible to pair up the numbers $0, 1, 2, 3, \ldots , 61$ in such a way that when we sum each pair, the product of the 31 numbers we get is a perfect fifth power.
\begin{solution}
    Pair 0 with 1, and $k$ with $63-k$ for all $2 \leq k \leq 31 $. This would result in a product equal to $1 \times 63^{30} = (63^6)^5$ which is a perfect $5^{\text{th}}$ power.
\end{solution}

\question
\begin{parts}
    \part[1] Two lines can divide the plane into at most how many regions?
    \begin{solution}
        \(4\)
    \end{solution}
    \part[1] Three lines can divide the plane into at most how many regions?
    \begin{solution}
        \(7\) (If you try it on paper, you will see that the third line can only intersect
        the existing \(2\) lines in at most \(2\) places, resulting in \(3\) new regions.
        Therefore the number of regions is \(4+3=7\).)
    \end{solution}
    \part[1] If we add a fourth line, what is the maximum number of intersection
    points that line can make with the existing lines?
    \begin{solution}
        \(3\)
    \end{solution}
    \part[1] Four lines can divide the plane into at most how many regions?
    \begin{solution}
        \(11\)
    \end{solution}
    \part[3] What is the maximum number of regions that \(n\) lines can divide the plane into?
    \begin{solution}
        \(1+n(n+1)/2\)
    \end{solution}
\end{parts}

       \question
\begin{parts}
    \part[2] Is the sum or difference of two rational numbers always rational? Why or why not?
    \begin{solution}
        Yes, because \(a/b+c/d=(ad+bc)/bd\) and both \(ad+bc\) and \(bd\) are integers.
    \end{solution}
    \part[2] Is the product or quotient of two rational numbers always rational? Why or why not?
    \begin{solution}
        Yes, because \((a/b)(c/d)=ac/bd\) and both \(ac\) and \(bd\) are integers.
    \end{solution}
    \part[2] When one rational number is raised to another is the result guaranteed to be rational?
    \begin{solution}
        No; an example is \(2^{1/2}\). 
    \end{solution}
    \part[3] It is known that \(\sqrt2\) is irrational, but it is not known whether \(\sqrt2^{\sqrt2}\) is rational or irrational. 
    Do there exist two irrational numbers such that when one is raised to the other the result is rational? Prove your conjecture.
    \begin{solution}
        The answer is yes.
        Let \(A=\sqrt2^{\sqrt2}\). We don't know if \(A\) is rational or irrational, but it doesn't
        matter. If \(A\) is rational, then it is an example of a rational number that is an
        irrational number raised to another irrational number. If \(A\) is irrational,
        then \(A^{\sqrt2}=\sqrt2^{2}=2\) and this number is the desired example.
    \end{solution}
\end{parts}


   
    \question 
     \begin{parts}
     \part[2] Prove that $x^2+y^2 \ge 2xy$ for real numbers $x,y$. 
     \begin{solution}
     Moving $2xy$ to the left we get $(x-y)^2 \ge 0$. Which is true. 
\end{solution}
     \part[2] Prove that $2a^2+b^2+c^2 \ge 2(ab+ac)$ for real numbers $a, b, c$. 
      \begin{solution}
     Using the result from a), we add the two following inequalities $$a^2+b^2 \ge 2ab$$ $$a^2+c^2 \ge 2ac$$ to get $$2a^2+b^2+c^2 \ge 2(ab+ac).$$ 
\end{solution}
     \part[2] Prove that $3(a^2+b^2+c^2+d^2) \ge 2(ab+ac+ad+bc+bd+cd)$ for real numbers $a, b, c, d$. 
      \begin{solution}
    Same deal as part b), just with more inequalities. We add up $$a^2+b^2 \ge 2ab$$ $$a^2+c^2 \ge 2ac$$ $$a^2+d^2 \ge 2ad$$ $$b^2+c^2 \ge 2bc$$ $$b^2+d^2\ge 2bd$$ $$c^2+d^2 \ge 2cd$$ to get $$3(a^2+b^2+c^2+d^2) \ge 2(ab+ac+ad+bc+bd+cd).$$
\end{solution}
     \part[3] Real numbers $a, b, c, d, e$ are linked by the two equations: $$e = 40 - a - b - c -d$$ $$e^2 = 400 - a^2 - b^2 - c^2 - d^2$$ Determine the largest value for $e$. 
      \begin{solution}
    From the first equation we have $$(40-e)^2=(a+b+c+d)^2=a^2+b^2+c^2+d^2+2(ab+ac+ad+bc+bd+cd)$$
Using the result from part c), we get that $$(40-e)^2 \le 4(a^2+b^2+c^2+d^2)$$ Using the second equation we get $$(40-e)^2 \le 4(400-e^2)$$ After expanding the brackets and simplifying, it follows that $$e(80-5e) \ge 0$$ Implying that $0 \le e \le 16$. Thus the largest value for $e$ that satisfies the given equations is 16. 
\end{solution}
     
     \end{parts}

     
\question
\begin{parts}
    \part[1] How many ways are there to create a two-element subset from the set \(\{1,2,3,4\}\)?
    \begin{solution}
        There are \(\binom{4}{2}=6\) ways.
    \end{solution}
    \part[2] How many ways are there to split the set \(\{1,2,3,4\}\) into two disjoint nonempty sets? (Disjoint means the two sets share no elements.) For example, if we had the set \(\{1,2,3\}\) a valid
    splitting would be \(\{1,2\}\) and \(\{3\}\).
    \begin{solution}
        The valid ways are: \(\{1\}\) and \([\{2,3,4\}\), \(\{2\}\) and \(\{1,3,4\}\),  \(\{3\}\) and
        \(\{1,2,4\}\), \(\{4\}\) and \(\{1,2,3\}\), \(\{1,2\}\) and \(\{3,4\}\), \(\{1,3\}\) and \(\{2,4\}\), and \(\{1,4\}\) and \(\{2,3\}\). So \(7\) ways.
    \end{solution}
    \part[2] How many ways are there to split the set \(\{1,2,3,4,5\}\) into two disjoint nonempty sets?
    \begin{solution}
        By listing all the ways again, we find that there are \(15\) ways.
    \end{solution}
    \part[2] Denote by \(S(n,k)\) the number of ways to split a set of \(n\) elements into \(k\) disjoint
    nonempty sets. Suppose we know what \(S(n-1,2)\)  is. What happens when we add another element
    to the set? What happens to each of the \(S(n-1,2)\) ways to do the partitioning?
    \begin{solution}
        For each way to do the partitioning, we may add the new number \(n\) to either
        set to obtain a new partitioning. There is also one new partition that is not
        obtained by adding \(n\) to existing one, and that is the splitting into \(\{n\}\)
        and \(\{1,2,\ldots,n-1\}\). Therefore we have \[S(n,2)=2S(n-1,2)+1.\]
    \end{solution}
    \part[2] What is \(S(n,2)\)?
    \begin{solution}
        We have \(S(1,2)=0\), \(S(2,2)=1\), \(S(3,2)=3\), \(S(4,2)=7\), and \(S(5,2)=15\).
        From this pattern and the recursive formula found above it is easy to prove by
        induction that \(S(n,2)=2^{n-1}-1.\)
    \end{solution}
    \bonuspart[3] If there are \(S(n-1,k-1)\) ways to partition a set of \(n-1\) elements into \(k-1\) nonempty
    disjoint sets, what is \(S(n,k)\) in terms of \(S(n-1,k)\) and \(S(n-1,k-1)\)?
    \begin{solution}
        We need to think about what happens when we add the \(n\)th element. Clearly we may
        put this object into a separate set by itself and partition the other \(n-1\) objects
        into \(k-1\) nonempty disjoint subsets, because all up there would be \(k\) sets. 
        There are \(S(n-1,k-1)\) ways to do this. We may also take the \(n-1\) objects
        and partition them into \(k\) sets and add the \(n\)th element to any one of them
        to get another valid partitioning. How many ways are there to do this? In each
        of the \(S(n-1,k)\) ways to partition \(n-1\) objects into \(k\) nonempty sets, there
        are \(k\) different subsets we can put the \(n\)th object into. Therefore
        in this case there are \(kS(n-1,k)\) ways. In total then we have \[S(n,k)=S(n-1,k-1)+kS(n-1,k).\]
    \end{solution}
\end{parts}
 



     



\end{questions}





\pagebreak
\section*{Year 11/12}
\begin{questions}
\question
\begin{parts}
    \part[1] Two lines can divide the plane into at most how many regions?
    \begin{solution}
        \(4\)
    \end{solution}
    \part[1] Three lines can divide the plane into at most how many regions?
    \begin{solution}
        \(7\) (If you try it on paper, you will see that the third line can only intersect
        the existing \(2\) lines in at most \(2\) places, resulting in \(3\) new regions.
        Therefore the number of regions is \(4+3=7\).)
    \end{solution}
    \part[1] If we add a fourth line, what is the maximum number of intersection
    points that line can make with the existing lines?
    \begin{solution}
        \(3\)
    \end{solution}
    \part[1] Four lines can divide the plane into at most how many regions?
    \begin{solution}
        \(11\)
    \end{solution}
    \part[3] What is the maximum number of regions that \(n\) lines can divide the plane into?
    \begin{solution}
        \(1+n(n+1)/2\)
    \end{solution}
\end{parts}

\question
\begin{parts}
    \part[1] How many ways are there to create a two-element subset from the set \(\{1,2,3,4\}\)?
    \begin{solution}
        There are \(\binom{4}{2}=6\) ways.
    \end{solution}
    \part[2] How many ways are there to split the set \(\{1,2,3,4\}\) into two disjoint nonempty sets? (Disjoint means the two sets share no elements.) For example, if we had the set \(\{1,2,3\}\) a valid
    splitting would be \(\{1,2\}\) and \(\{3\}\).
    \begin{solution}
        The valid ways are: \(\{1\}\) and \([\{2,3,4\}\), \(\{2\}\) and \(\{1,3,4\}\),  \(\{3\}\) and
        \(\{1,2,4\}\), \(\{4\}\) and \(\{1,2,3\}\), \(\{1,2\}\) and \(\{3,4\}\), \(\{1,3\}\) and \(\{2,4\}\), and \(\{1,4\}\) and \(\{2,3\}\). So \(7\) ways.
    \end{solution}
    \part[2] How many ways are there to split the set \(\{1,2,3,4,5\}\) into two disjoint nonempty sets?
    \begin{solution}
        By listing all the ways again, we find that there are \(15\) ways.
    \end{solution}
    \part[2] Denote by \(S(n,k)\) the number of ways to split a set of \(n\) elements into \(k\) disjoint
    nonempty sets. Suppose we know what \(S(n-1,2)\)  is. What happens when we add another element
    to the set? What happens to each of the \(S(n-1,2)\) ways to do the partitioning?
    \begin{solution}
        For each way to do the partitioning, we may add the new number \(n\) to either
        set to obtain a new partitioning. There is also one new partition that is not
        obtained by adding \(n\) to existing one, and that is the splitting into \(\{n\}\)
        and \(\{1,2,\ldots,n-1\}\). Therefore we have \[S(n,2)=2S(n-1,2)+1.\]
    \end{solution}
    \part[2] What is \(S(n,2)\)?
    \begin{solution}
        We have \(S(1,2)=0\), \(S(2,2)=1\), \(S(3,2)=3\), \(S(4,2)=7\), and \(S(5,2)=15\).
        From this pattern and the recursive formula found above it is easy to prove by
        induction that \(S(n,2)=2^{n-1}-1.\)
    \end{solution}
    \bonuspart[3] If there are \(S(n-1,k-1)\) ways to partition a set of \(n-1\) elements into \(k-1\) nonempty
    disjoint sets, what is \(S(n,k)\) in terms of \(S(n-1,k)\) and \(S(n-1,k-1)\)?
    \begin{solution}
        We need to think about what happens when we add the \(n\)th element. Clearly we may
        put this object into a separate set by itself and partition the other \(n-1\) objects
        into \(k-1\) nonempty disjoint subsets, because all up there would be \(k\) sets. 
        There are \(S(n-1,k-1)\) ways to do this. We may also take the \(n-1\) objects
        and partition them into \(k\) sets and add the \(n\)th element to any one of them
        to get another valid partitioning. How many ways are there to do this? In each
        of the \(S(n-1,k)\) ways to partition \(n-1\) objects into \(k\) nonempty sets, there
        are \(k\) different subsets we can put the \(n\)th object into. Therefore
        in this case there are \(kS(n-1,k)\) ways. In total then we have \[S(n,k)=S(n-1,k-1)+kS(n-1,k).\]
    \end{solution}
\end{parts}

       \question
\begin{parts}
    \part[2] Is the sum or difference of two rational numbers always rational? Why or why not?
    \begin{solution}
        Yes, because \(a/b+c/d=(ad+bc)/bd\) and both \(ad+bc\) and \(bd\) are integers.
    \end{solution}
    \part[2] Is the product or quotient of two rational numbers always rational? Why or why not?
    \begin{solution}
        Yes, because \((a/b)(c/d)=ac/bd\) and both \(ac\) and \(bd\) are integers.
    \end{solution}
    \part[2] When one rational number is raised to another is the result guaranteed to be rational?
    \begin{solution}
        No; an example is \(2^{1/2}\). 
    \end{solution}
    \part[3] It is known that \(\sqrt2\) is irrational, but it is not known whether \(\sqrt2^{\sqrt2}\) is rational or irrational. 
    Do there exist two irrational numbers such that when one is raised to the other the result is rational? Prove your conjecture.
    \begin{solution}
        The answer is yes.
        Let \(A=\sqrt2^{\sqrt2}\). We don't know if \(A\) is rational or irrational, but it doesn't
        matter. If \(A\) is rational, then it is an example of a rational number that is an
        irrational number raised to another irrational number. If \(A\) is irrational,
        then \(A^{\sqrt2}=\sqrt2^{2}=2\) and this number is the desired example.
    \end{solution}
\end{parts}

    \question % OK, this is a monster problem... this'll be the last one on the senior questions.
\begin{parts}
\part[1] If two normal six-sided die are rolled, what is the probability that the sum of the two numbers is \(2\)?
\begin{solution}
    Out of the \(36\) possible pairs the only one that has a sum of \(2\) is \((1,1)\), so 
    the probability is \(1/36\).
\end{solution}
\part[1] If two normal six-sided die are rolled, what is the probability that the sum of the two numbers is \(7\)?
\begin{solution}
    There are \(6\) ways to get a \(7\): \((1,6)\), \((2,5\), \((3,4)\), \((4,3)\), \((5,2)\), and
    \((6,1)\). Therefore the probability is \(1/6\).
\end{solution}
\part[2] Consider a biased six-sided die in which the probability of rolling \(n\) is \(p_n\). This die is rigged
so that when two of them are rolled, every possible sum of the two numbers is equally likely. What is \(p_2\) in
terms of \(p_1\)?
\begin{solution}
    The probability of getting a \(3\) is \(p_2p_1+p_1p_2\), which by hypothesis is equal
    to the probability of getting a \(2\), or \(p_1^2\). Therefore \(2p_2=p_1\), which implies
    that \(p_2=p_1/2\).
\end{solution}
\part[1] What is \(p_3\) in terms of \(p_1\)?
\begin{solution}
    Following the previous procedure we find that \(p_3=3p_1/8\).
\end{solution}
\part[2] What is \(p_4\) in terms of \(p_1\)? Do you notice a pattern?
\begin{solution}
    \(p_4=5p_1/16\). The pattern is \(1/2\), \((1/2)(3/4)\), \((1/2)(3/4)(5/6)\), and so on.
\end{solution}
\part[2] Let \(p_{i}=a_{i-1}p_1\). So \(a_0=1\), \(p_2=a_1p_1\), \(p_3=a_2p_1\), \(p_4=a_3p_1\) and so on. What are
\(a_0a_1+a_1a_0\), \(a_0a_2+a_1a_1+a_2a_0\), \(a_0a_3+a_1a_2+a_2a_1+a_3a_0\), etc. all equal to?
\begin{solution}
    If \(p_1p_2+p_2p_1=p_1^2\) then \[a_0p_1a_1p_1+a_0p_1a_1p_1=p_1^2\] which implies 
    \[(a_0a_1+a_1a_0)p_1^2=p_1^2.\] Therefore \(a_0a_1+a_1a_0=1.\) Similarly the other 
    combinations of \(a\)'s can be found to be \(1\) too.
\end{solution}
\part[1] Since we have a six-sided die, \[p_1+p_2+\cdots+p_6=1.\] What is \(p_1\) in terms of \(a_0,a_1,a_2,\ldots,a_5\)?
\begin{solution}
    Since the die is six-sided, we have \(p_1+p_2+\cdots+p_6=1\). We can express this 
    in terms of \(p_1\): \[a_0p_1+a_1p_1+a_2p_1+\cdots+a_5p_1=1\] and hence
    \[ p_1=\frac{1}{a_0+a_1+\cdots+a_5}.\]
\end{solution}
\part[2] Let \(s_i=a_0+a_1+\cdots+a_i.\) What are \(s_0,s_1,s_2\)? What is \(s_5\)?
\begin{solution}
    \(s_0=a_0=1\), \(s_1=a_0+a_1=3/2\), \(s_2=1+1/2+3/8=15/8\). Note that 
    \(s_2=(3/2)(5/4)\), \(s_3=(3/2)(5/4)(7/6)\), \(s_4=(3/2)(5/4)(7/6)(9/8)\) and so on.
    Therefore \[s_5=\frac{3\cdot5\cdot7\cdot9\cdot11}{2\cdot4\cdot6\cdot8\cdot10}=\frac{693}{256}.\]
\end{solution}
\part[3] Is such a six-sided die possible?
\begin{solution}
    From the previous part we find that \(p_1=256/693.\) This means that the probability of
    rolling two of these dice and getting a sum of \(2,3,\ldots,12\) must all be
    \(p_1^2=65536/480249.\) But this is impossible because there are \(11\) possible
    sums and if they are equally likely each must have a probability of \(1/11\). Therefore
    such a die is not possible.
\end{solution}
\end{parts}
\question
\begin{parts}
    \part[1] What is \(1+2+3+\cdots+200\)?
    \begin{solution}
        \(10100\)
    \end{solution}
    \part[2] Find a formula for \(1+2+\cdots+n\) and prove it.
    \begin{solution}
        The formula is \[1+2+\cdots+n=\frac{n(n+1)}{2}.\] It is easily proved by induction.
        When \(n=1\) we have \(1=1(1+1)/2\) so the base case is done. Now assume that
        \[1+2+\cdots+k=\frac{k(k+1)}{2}\] for some natural number \(k\). Add \(k+1\) to
        both sides: \[1+2+\cdots+k+(k+1)=\frac{k(k+1)}{2}+\frac{2(k+1)}{2}=\frac{(k+1)(k+2)}{2}\] 
        and the induction is complete.
    \end{solution}
    \part[1] Define \[F_k(n)=1^k+2^k+\cdots+n^k.\] (So in the previous part you found a formula for
    \(F_1(n)\).) More compactly it may be written as
    \[F_k(n)=\sum_{i=1}^{n}i^k.\]
    What is \[1^3+(2^3-1^3)+(3^3-2^3)+\cdots+(100^3-99^3)?\]
    \begin{solution}
        This sum is an example of a telescoping sum. All the terms vanish except for
        the \(100^3\) term in the last set of parentheses, so the answer is \(100^3\).
    \end{solution}
    \part[2] By considering \[1^3+\sum_{i=1}^{n}((i+1)^3-i^3)\] find a formula for \(F_2(n)\).
    \begin{solution}
    We know that \[1^3+\sum_{i=1}^{n}((i+1)^3-i^3)=(n+1)^3.\] Expanding both sides
    using the Binomial Theorem we get \[1^3+\sum_{i=1}^{n}(3i^2+3i+1)=n^3+3n^2+3n+1.\]
    After removing the \(1\) from both sides, on the LHS we can break up the sum and pull out the constant factors, like so:
    \[3\sum_{i=1}^ni^2+3\sum_{i=1}^ni+\sum_{i=1}^n1=n^3+3n^2+3n.\]
    But note that \(\sum_{i=1}^ni^2=F_2(n)\) and \(\sum_{i=1}^ni=F_1(n)\), so we have
    \[3F_2(n)+3F_1(n)+n=n^3+3n^2+3n.\] Then we just need to rearrange for \(F_2(n)\) after
    substituting in \(F_1(n)=n(n+1)/2\) to
    find that \[F_2(n)=\frac{2n^3+3n^2+n}{6}.\]
    \end{solution}
    \part[2] Find a formula for \(F_3(n)\).
    \begin{solution}
        Using the same method as above, we find that \[F_3(n)=\frac{n^4+2n^3+n^2}{4}.\]
    \end{solution}
    \part[1] The Binomial Theorem states that \[(x+y)^n=\sum_{k=0}^{n}\binom{n}{k}x^ky^{n-k}.\] If 
    \[(i+1)^k-i^k=\sum_{j=0}^{x}\binom{k}{j}i^j\] what is \(x\) in terms of \(k\)?
    \begin{solution}
        The leading term vanishes, so instead of going from \(0\) to \(k\) the loop only
        goes up to \(k-1\). Therefore \(x=k-1.\)
    \end{solution}
    \part[3] By considering the general telescoping sum 
    \[1^k+\sum_{i=1}^n((i+1)^k-i^k)=(n+1)^k\] show that 
    \[F_{k-1}(n)=\frac{(n+1)^k-1}{k}+\frac{1}{k}\sum_{i=0}^{k-2}\binom{k}{i}F_i(n).\]
    \begin{solution}
        Expand the \((n+1)^k\)  term using the Binomial Theorem, applying the previous part:
        \[(n+1)^i=1^k+\sum_{i=1}^{k}\sum_{j=0}^{k-1}\binom{k}{j}i^j.\]
        Now expand the inner sigma notation:
        \[(n+1)^k=1^k+\sum_{i=1}^{k}\left(\binom{k}{0}i^0+\binom{k}{1}i^1+\binom{k}{2}i^2+\cdots+\binom{k}{k-1}i^{k-1}\right).\]
        We may take out the binomial coefficients, as they are independent of the sum variable
        \(i\), so we have 
        \[(n+1)^k=1^k+\binom{i}{0}\sum_{i=1}^ki^0+\binom{k}{1}\sum_{i=1}^ki^1+\binom{k}{2}\sum_{i=1}^ki^2+\cdots+\binom{k}{k-1}\sum_{i=1}^{k}i^{k-1}.\]
        We're interested in \(F_{k-1}(n)\), so we notice that the binomial coefficient
        \(\binom{k}{k-1}=k\) and isolate that term. The other sums are \(F_0(n)\), \(F_1(n)\),
        \(F_2(n)\) and so on. Therefore we get
        \[(n+1)^k=1^k+\binom{k}{0}F_0(n)+\binom{k}{1}F_1(n)+\binom{k}{2}F_2(n)+\cdots+\binom{k}{k-2}F_{k-2}(n)+kF_{k-1}(n)\]
        or more compactly,
        \[(n+1)^k=1+\sum_{i=0}^{k-2}\binom{k}{i}F_i(n)+kF_{k-1}(n).\]
        Rearranging for \(F_{k-1}(n)\) we get the desired result.
    \end{solution}
\end{parts}

\end{questions}

\gradetable

\end{document}
