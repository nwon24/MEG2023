\documentclass[a4paper,12pt,addpoints]{exam}

\usepackage{amsmath}
\usepackage[a4paper,pass]{geometry}
\pagestyle{empty}
\begin{document}
\section*{Year 10}
\begin{questions}
\question[3]
There are twelve lockers, numbered from \(110\) to \(121\). The keys
to these twelve lockers are numbered \(1\) to \(12\). 
Each locker number is divisible by the number on its key.

Determine the key number for each locker.
\droptotalpoints

\begin{solution}
\begin{center}
\begin{tabular}{ |c|c|c|c|c|c|c|c|c|c|c|c| }
\hline 110&111&112&113&114&115&116&117&118&119&120&121\\
\hline 10&3&8&1&6&5&4&9&2&7&12&11\\
\hline
\end{tabular}
\end{center}
\end{solution}
\question[2] Three consecutive numbers sum to \(141\). What are the numbers?

\droptotalpoints

\question 
\begin{parts}
\part[2] When is the sum of 2 consecutive numbers divisible by 2?
\begin{solution}
    Never, because \(n+(n+1)=2n+1\) is odd.
\end{solution}
\part[2] When is the sum of 3 consecutive numbers divisible by 3?
\begin{solution}
    Always, as \(n+(n+1)+(n+2)=3n+3.\)
\end{solution}
\part[2] When is the sum of 4 consecutive numbers divisible by 4?
\begin{solution}
    Never, as \(n+(n+1)+(n+2)+(n+3)=4n+10\equiv2\pmod{4}\).
\end{solution}
\part[2] When is the sum of 5 consecutive numbers divisible by 5?
\begin{solution}
    Always, as \(n+(n+1)+(n+2)+(n+3)+(n+4)=5n+10\).
\end{solution}
\part[2] When is the sum of \(n\) consecutive numbers divisible by \(n\)?
\begin{solution} 
    The sum of \(n\) consecutive numbers starting from \(k\) is \(kn\) plus the \(n-1\)th triangular number, so it
    is \[n\left(k+\frac{n-1}{2}\right).\] The resulting number is divisible by \(n\) if \((n-1)/2\) is an integer,
    and that is when \(n\) is odd.
\end{solution}

\end{parts}
\droptotalpoints

\question[3] Prove it is possible to pair up the numbers $0, 1, 2, 3, \ldots , 61$ in such a way that when we sum each pair, the product of the 31 numbers we get is a perfect fifth power.
\droptotalpoints
\begin{solution}
    Pair 0 with 1, and $k$ with $63-k$ for all $2 \leq k \leq 31 $. This would result in a product equal to $1 \times 63^{30} = (63^6)^5$ which is a perfect $5^{\text{th}}$ power.
\end{solution}

\question
\begin{parts}
    \part[1] Two lines can divide the plane into at most how many regions?
    \begin{solution}
        \(4\)
    \end{solution}
    \part[1] Three lines can divide the plane into at most how many regions?
    \begin{solution}
        \(7\) (If you try it on paper, you will see that the third line can only intersect
        the existing \(2\) lines in at most \(2\) places, resulting in \(3\) new regions.
        Therefore the number of regions is \(4+3=7\).)
    \end{solution}
    \part[1] If we add a fourth line, what is the maximum number of intersection
    points that line can make with the existing lines?
    \begin{solution}
        \(3\)
    \end{solution}
    \part[1] Four lines can divide the plane into at most how many regions?
    \begin{solution}
        \(11\)
    \end{solution}
    \part[3] What is the maximum number of regions that \(n\) lines can divide the plane into?
    \begin{solution}
        \(1+n(n+1)/2\)
    \end{solution}

    \end{parts}
\droptotalpoints

       \question
\begin{parts}
    \part[2] Is the sum or difference of two rational numbers always rational? Why or why not?
    \begin{solution}
        Yes, because \(a/b+c/d=(ad+bc)/bd\) and both \(ad+bc\) and \(bd\) are integers.
    \end{solution}
    \part[2] Is the product or quotient of two rational numbers always rational? Why or why not?
    \begin{solution}
        Yes, because \((a/b)(c/d)=ac/bd\) and both \(ac\) and \(bd\) are integers.
    \end{solution}
    \part[2] When one rational number is raised to another is the result guaranteed to be rational?
    \begin{solution}
        No; an example is \(2^{1/2}\). 
    \end{solution}
    \part[3] It is known that \(\sqrt2\) is irrational, but it is not known whether \(\sqrt2^{\sqrt2}\) is rational or irrational. 
    Do there exist two irrational numbers such that when one is raised to the other the result is rational? Prove your conjecture.
    \begin{solution}
        The answer is yes.
        Let \(A=\sqrt2^{\sqrt2}\). We don't know if \(A\) is rational or irrational, but it doesn't
        matter. If \(A\) is rational, then it is an example of a rational number that is an
        irrational number raised to another irrational number. If \(A\) is irrational,
        then \(A^{\sqrt2}=\sqrt2^{2}=2\) and this number is the desired example.
    \end{solution}

    \end{parts}
\droptotalpoints


   
    \question 
     \begin{parts}
     \part[2] Prove that $x^2+y^2 \ge 2xy$ for real numbers $x,y$. 
     \begin{solution}
     Moving $2xy$ to the left we get $(x-y)^2 \ge 0$. Which is true. 
\end{solution}
     \part[2] Prove that $2a^2+b^2+c^2 \ge 2(ab+ac)$ for real numbers $a, b, c$. 
      \begin{solution}
     Using the result from a), we add the two following inequalities $$a^2+b^2 \ge 2ab$$ $$a^2+c^2 \ge 2ac$$ to get $$2a^2+b^2+c^2 \ge 2(ab+ac).$$ 
\end{solution}
     \part[2] Prove that $3(a^2+b^2+c^2+d^2) \ge 2(ab+ac+ad+bc+bd+cd)$ for real numbers $a, b, c, d$. 
      \begin{solution}
    Same deal as part b), just with more inequalities. We add up $$a^2+b^2 \ge 2ab$$ $$a^2+c^2 \ge 2ac$$ $$a^2+d^2 \ge 2ad$$ $$b^2+c^2 \ge 2bc$$ $$b^2+d^2\ge 2bd$$ $$c^2+d^2 \ge 2cd$$ to get $$3(a^2+b^2+c^2+d^2) \ge 2(ab+ac+ad+bc+bd+cd).$$
\end{solution}
     \part[3] Real numbers $a, b, c, d, e$ are linked by the two equations: $$e = 40 - a - b - c -d$$ $$e^2 = 400 - a^2 - b^2 - c^2 - d^2$$ Determine the largest value for $e$. 
      \begin{solution}
    From the first equation we have $$(40-e)^2=(a+b+c+d)^2=a^2+b^2+c^2+d^2+2(ab+ac+ad+bc+bd+cd)$$
Using the result from part c), we get that $$(40-e)^2 \le 4(a^2+b^2+c^2+d^2)$$ Using the second equation we get $$(40-e)^2 \le 4(400-e^2)$$ After expanding the brackets and simplifying, it follows that $$e(80-5e) \ge 0$$ Implying that $0 \le e \le 16$. Thus the largest value for $e$ that satisfies the given equations is 16. 
\end{solution}
     

\end{parts}
\droptotalpoints

     
\question
\begin{parts}
    \part[1] How many ways are there to create a two-element subset from the set \(\{1,2,3,4\}\)?
    \begin{solution}
        There are \(\binom{4}{2}=6\) ways.
    \end{solution}
    \part[2] How many ways are there to split the set \(\{1,2,3,4\}\) into two disjoint nonempty sets? (Disjoint means the two sets share no elements.) For example, if we had the set \(\{1,2,3\}\) a valid
    splitting would be \(\{1,2\}\) and \(\{3\}\).
    \begin{solution}
        The valid ways are: \(\{1\}\) and \([\{2,3,4\}\), \(\{2\}\) and \(\{1,3,4\}\),  \(\{3\}\) and
        \(\{1,2,4\}\), \(\{4\}\) and \(\{1,2,3\}\), \(\{1,2\}\) and \(\{3,4\}\), \(\{1,3\}\) and \(\{2,4\}\), and \(\{1,4\}\) and \(\{2,3\}\). So \(7\) ways.
    \end{solution}
    \part[2] How many ways are there to split the set \(\{1,2,3,4,5\}\) into two disjoint nonempty sets?
    \begin{solution}
        By listing all the ways again, we find that there are \(15\) ways.
    \end{solution}
    \part[2] Denote by \(S(n,k)\) the number of ways to split a set of \(n\) elements into \(k\) disjoint
    nonempty sets. Suppose we know what \(S(n-1,2)\)  is. What happens when we add another element
    to the set? What happens to each of the \(S(n-1,2)\) ways to do the partitioning?
    \begin{solution}
        For each way to do the partitioning, we may add the new number \(n\) to either
        set to obtain a new partitioning. There is also one new partition that is not
        obtained by adding \(n\) to existing one, and that is the splitting into \(\{n\}\)
        and \(\{1,2,\ldots,n-1\}\). Therefore we have \[S(n,2)=2S(n-1,2)+1.\]
    \end{solution}
    \part[2] What is \(S(n,2)\)?
    \begin{solution}
        We have \(S(1,2)=0\), \(S(2,2)=1\), \(S(3,2)=3\), \(S(4,2)=7\), and \(S(5,2)=15\).
        From this pattern and the recursive formula found above it is easy to prove by
        induction that \(S(n,2)=2^{n-1}-1.\)
    \end{solution}
    \bonuspart[3] If there are \(S(n-1,k-1)\) ways to partition a set of \(n-1\) elements into \(k-1\) nonempty
    disjoint sets, what is \(S(n,k)\) in terms of \(S(n-1,k)\) and \(S(n-1,k-1)\)?
    \begin{solution}
        We need to think about what happens when we add the \(n\)th element. Clearly we may
        put this object into a separate set by itself and partition the other \(n-1\) objects
        into \(k-1\) nonempty disjoint subsets, because all up there would be \(k\) sets. 
        There are \(S(n-1,k-1)\) ways to do this. We may also take the \(n-1\) objects
        and partition them into \(k\) sets and add the \(n\)th element to any one of them
        to get another valid partitioning. How many ways are there to do this? In each
        of the \(S(n-1,k)\) ways to partition \(n-1\) objects into \(k\) nonempty sets, there
        are \(k\) different subsets we can put the \(n\)th object into. Therefore
        in this case there are \(kS(n-1,k)\) ways. In total then we have \[S(n,k)=S(n-1,k-1)+kS(n-1,k).\]
    \end{solution}

    \end{parts}
\droptotalpoints
 



     



\end{questions}
Total: \numpoints
\end{document}
