\documentclass[a4paper,12pt]{exam}
\usepackage{geometry}

\usepackage{fontspec}
\usepackage{unicode-math}
\setmainfont{TeX Gyre Termes}
\setmathfont{TeX Gyre Termes Math}

\begin{document}
\begin{questions}
  \question The Fibonacci sequence is the sequence \[1,1,2,3,5,8\ldots.\] Find the sum of the first \(10\) terms.
  \question The sum of the first \(n\) odd numbers is \(841\). Find \(n\).
  \question Today (Tuesday) I start taking lectures on potions. If these lectures
  happen every second day, which lecture will be the first to fall on a Sunday?
  \question If the sum of two positive real numbers is \(4\) times their product, what is the sum of the reciprocals of the
  two numbers?
  \question What is the area of the equilaterial triangle circumscribed by a circle of radius \(1\)?
  \question My car averages \(40\) kilometres per litre of petrol, while my friend's car averages
  \(10\) kilometres per litre of petrol. If we both drive the same distance, what is the combined rate
  of kilometres per litre of petrol?
  \question If the side lengths of a rectangle are in the ratio \(4:3\) and \(d\) is the length of the diagonal, it can be shown that the area
  of the rectangle is given by \(kd^2\). Find \(k\).
  \question The vertices of an isoceles triangle lie on the graph of \(y=x^2\). If the area of the triangle
  is \(64\), what is the length of the shortest side?
  \question A palindromic number is one that is the same read forwards and backwards. For example, \(292\)
  is palindromic. How many three-digit palindromic numbers are there less than \(1000\)?
  \question Three of a rectangular prism's faces have areas of \(12\), \(28\), and \(21\). What is the volume of the rectangular prism?
  \question What is the last digit of \(9^{2023}\)?
  \question Two hoses can be used to fill a swimming pool. Hose B takes twice as long as Hose A to fill the pool.
  If Hose A takes \(6\) minutes to fill the pool, how long will it take for both hoses together to fill the pool?
  \question A normal coin is flipped \(3\) times. Given that at least one coin landed tails, what is the probability
  that there are two consecutive heads?
  \question There is a tennis tournament with \(1025\) players. Each round, every player is paired with another, and if
  there are an odd number of players one player sits out. A loss immediately knocks out a player from the tournament. The tournament
  continues until only one player remains. How many matches are played in total?
  \question For how many integers \(n\), where \(1\le n\le 100\), is \(n^n\) a square number?
  \question What is the smallest positive integer \(n\) such that \[(2^2-1)(3^2-1)(4^2-1)\cdots(n^2-1)\] is a square number?
  \question In the nine digit number \[347*47*64\] two digits are missing, as indicated by the asterisks. If two digits are
  randomly chosen for the two missing spots, what is the probability that the number is divisible by \(36\)?
  \question Given an arithmetic sequence \(a_1,a_2,\ldots,a_n\) where \(a_{k+1}-a_{k}=d\) for \(k=0,1,\ldots,n-1\),
  the sum is given by \[a_1+a_2+\cdots+a_n=\frac{n}{2}\left(2a+(n-1)d\right).\]
  Find \[1-4+9-16+\cdots+99^2-100^2.\]
  \question Positive numbers are written in a \(3\times3\) grid such that the product of the numbers in every row
  and column is \(1\) and the product of the numbers in every possible \(2\times2\)  grid is \(2\). Which number is in
  the centre of the grid?
  \question In \(\triangle ABC\), \(\angle C\) is a right angle and \(AB=12\). Squares \(ACWZ\) and \(ABXY\) are constructed
  so that points \(X\), \(Y\), \(W\), and \(Z\) lie outside the triangle. If those four points also lie on a circle, what is the perimeter
  of the triangle?
\end{questions}
\pagebreak
\begin{enumerate}
\item 143
\item \(29\)
\item \(7\)th lecture
\item \(4\)
\item \(3\sqrt3/4\)
\item \(16\) kilometres per litre
\item \(12/25\)
\item \(8\)
\item \(90\)
\item \(84\)
\item \(9\)
\item \(4\) minutes
\item \(2/7\)
\item \(1024\)
\item \(55\)
\item \(8\)
\item \(11/100\)
\item \(-5050\)
\item \(16\)
\item \(12+12\sqrt2\)
\end{enumerate}
\end{document}