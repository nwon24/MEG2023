\documentclass[a4paper]{article}

\newcommand{\theterm}{2}
\newcommand{\theweek}{6}
\newcommand{\thepdftitle}{MEG 2023 Term \theterm\ Week \theweek\ Handout}
\newcommand{\thedisplaytitle}{Term \theterm\ Week \theweek\ Handout}

\title{{\thepdftitle}}
\author{Nathan Wong\and Tom Yan}
\date{2023}

% \newcommand{\marginfn}[1]{\marginpar{\footnotemark}\footnotetext{#1}}
\newcommand{\marginnote}[1]{\marginpar{\footnotesize{#1}}}
\newcommand{\marginfnote}[1]{\footnotemark\marginpar{\footnotemark[\value{footnote}]\footnotesize{#1}}}
\usepackage{geometry}
% \geometry{a4paper,left=24.8mm,top=27.4mm,headsep=2\baselineskip,textwidth=107mm,marginparsep=8.2mm,marginparwidth=49.4mm,textheight=49\baselineskip,headheight=\baselineskip}
\geometry{a4paper,left=1in,top=1in,bottom=1in,headsep=2\baselineskip,textwidth=107mm,marginparsep=8.2mm,marginparwidth=49.4mm,textheight=49\baselineskip,headheight=\baselineskip}
\usepackage[bf,tiny]{titlesec}
% \usepackage{fancyhdr}
\usepackage{epigraph}
% \usepackage[indent=0pt,skip=10pt]{parskip}

\usepackage{amsmath}
\usepackage{amsthm}
\newtheorem{theorem}{Theorem}
\usepackage{amssymb}
\let\mathbbalt\mathbb

\usepackage{fontspec}
\usepackage{unicode-math}
\let\mathbb\mathbbalt

\newcommand{\naturals}{\mathbb{N}}
\newcommand{\reals}{\mathbb{R}}
\newcommand{\rationals}{\mathbb{Q}}
\newcommand{\integers}{\mathbb{Z}}

\usepackage[pdfusetitle]{hyperref}

\newcommand{\myquote}[2]{%
  \begin{quote}
    \emph{#1}
    \begin{flushright}---{#2}
    \end{flushright}
  \end{quote}}
\pagestyle{empty}
\begin{document}
\noindent Melbourne High School\\
\noindent Maths Extension Group 2023\\
\noindent \textbf{\thedisplaytitle}\\
\myquote{I love geometry.}{Anonymous, unpublished correspondence (2023)}
\section*{More interesting geometry problems}
\addcontentsline{toc}{section}{More interesting geometry problems}
\begin{enumerate}
\item What is the measure of an angle, in degrees, if its supplement is six times its complement? 
\item The point $O$ is the center of the circle circumscribed about $\triangle ABC$, with $\angle BOC = 120^{\circ}$ and $\angle AOB = 140^{\circ}$. What is the degree measure of $\angle ABC$?
\item  Prove \marginnote{Inscribed angle theorem} that if $\angle ACB$ is inscribed in a circle, then it subtends an arc with measure 2$\angle ACB$. 
\item  Points \marginnote{Russia 1996} $E$ and $F$ are on side $\overline{BC}$ of convex quadrilateral $ABCD$ (with E closer to B). It is known that $\angle BAE = \angle CDF $ and $\angle EAF = \angle FDE $. Prove that $\angle FAC = \angle EDB$ 
\item  Points \marginnote{USAJMO 2011/5} $A$, $B$, $C$, $D$, $E$ lie on a circle $\omega$ and point $P$ lies outside the circle. The given points are such that (i) lines $PB$ and $PD$ are tangent to $\omega$, (ii) $P$, $A$, $C$ are collinear, and (iii) $\overline{DE} \parallel \overline{AC}$. Prove that $\overline{BE}$ bisects $\overline{AC}$.
  \end{enumerate}
  \pagebreak
  
\myquote{Et cette proposition est généralement vraie en toutes progressions et en tous nombres premiers; de quoi je vous envoierois la démonstration, si je n'appréhendois d'être trop long.}{P.~de Fermat, letter to F.~de Bessy (October 18, 1640)}
\section*{Fermat's Little Theorem}
\addcontentsline{toc}{section}{Fermat's Little Theorem}
\marginnote{Do not confuse Fermat's Little Theorem with his \emph{last theorem}. While Fermat's Little Theorem
  is fairly easy to prove and was first proved hundreds of years ago, Fermat's Last Theorem was an unsolved mystery
for centuries.} Modular arithmetic is an incredibly powerful tool in part because it allows us to
reduce an infinite set of numbers (the integers) into a finite set with comparatively
few elements, and yet (most of) the usual laws of arithmetic hold. It is unlikely
we would be able to calculate the value of \(2^{100}\) in a normal equation, but
what about that expression in a congruence modulo \(101\)? The answer is that we
can easily evaluate it; in fact, \(2^{100}\) is congruent to a very nice number modulo \(101\).
For this marvel we have Fermat's Little Theorem to thank.

We being by examining only prime moduli, as is customary in number theory.
Take any prime \(p\) and any integer \(a\) not divisible by \(p\) and consider the powers of \(a\)
\[a^0,a^1,a^2,a^3,\ldots\] modulo \(p\). Our first observation is that eventually
the numbers must repeat because there are only \(p\) possible residues. Since the sequence
starts with \(a^0\equiv 1\), we know for sure that \(1\) will appear again;
that is, there exists some \(l>0\) such that \[a^l\equiv 1\pmod{p}.\] For example, consider
the powers of \(3\) modulo \(7\) shown in \autoref{fig:p3mod7}.
\begin{figure}[h]
  \centering
  \begin{tabular}{|c|c|c|c|c|c|c|}
    \hline
    \(n\)&\(1\)&\(2\)&\(3\)&\(4\)&\(5\)&\(6\)\\
    \hline
    \(3^n\)&\(3\)&\(9\)&\(27\)&\(81\)&\(243\)&\(729\)\\
    \hline
    \(3^n\pmod{7}\)&\(3\)&\(2\)&\(6\)&\(4\)&\(5\)&\(1\)\\
    \hline
  \end{tabular}
  \caption{The powers of \(3\) modulo \(7\)}
  \label{fig:p3mod7}
\end{figure}
It is plain that the first power of \(3\) that is congruent to \(1\)
modulo \(7\) is \(3^6\). Having reached \(1\), the powers of \(3\)
will then cycle, since \[3^7=3^6\times3\equiv 1\times3\equiv 3\pmod{7}.\]
So if we continue finding the powers of \(3\) modulo \(7\) we get \(3^7\equiv 3\), \(3^8\equiv 2\), \(3^9\equiv 6\), and so on.
More generally, if \(a^l\equiv 1\pmod{p}\), then \(a^{l+k}\equiv a^k\) for any
positive integer \(k\).

We define the smallest integer \(l>0\) such that \(a^l\equiv 1\pmod{p}\) to
be the \emph{order of \(a\) to the modulus \(p\)}.
In the above example,
the order of \(3\) to the modulus \(7\) is \(6\), because \(6\) is the
smallest positive integer satisfying \(3^n\equiv 1\pmod{7}\). If we replace
the modulus by \(11\), we find that the order of \(3\) to this new
modulus is \(5\).\marginnote{If you have time, list powers of numbers
  to various prime moduli and try to spot some patterns. }

Combining our observation that the powers of a number modulo a prime number
are periodic and the definition of the order of a number, we find that
if \(a^n\equiv 1\pmod{p}\) then \(n\) must be a multiple of the order of \(a\). 
Conversely, if \(n\) is a multiple of the order of \(a\) then \(a^n\equiv 1\pmod{p}\).
The former proposition follows immediately from the cyclic nature of the powers of \(a\).
If we let \(l\) be the order of \(a\), then the powers of \(a\) modulo \(p\) repeat every \(l\) terms,
so \(a^n\equiv 1\) only if \(n\) can be obtained from \(l\) by adding a multiple of \(l\). Consequently
\(n\) must be a multiple of \(l\).

This latter statement is also easy to prove. If again \(l\) is the order of \(a\) and \(n=lk\)
for some integer \(k\), then \(a^{lk}=(a^l)^k\). By definition \(a^l\equiv 1\pmod{p}\) so
\(a^{lk}\) is congruent to \(1^k\), which is just \(1\).

Fermat's Little Theorem (yes, we are finally getting to it) states that for any prime
\(p\) and an integer \(a\) coprime with \(p\), the order of \(a\) is a divisor of \(p-1\).
Equivalently, we may state the theorem as the congruence
\begin{equation}\label{eq:fermatlt}
  a^{p-1}\equiv 1\pmod{p}.
\end{equation}

Sometimes the theorem is expressed as
\begin{equation}\label{eq:fermatlt2}
  a^p\equiv a\pmod{p}.
  \end{equation}
This equation holds for any \(a\),
not just those coprime with \(p\). We can obtain Equation \eqref{eq:fermatlt} just by dividing both
sides of Equation \eqref{eq:fermatlt2} by \(a\)---but recall that this is only allowed if \(a\) is coprime with \(p\), and hence we
derive the condition on \(a\) for the first formulation of the theorem.

There are several proofs of Fermat's Little Theorem. The proof we present here was discovered
by British Mathematician James Ivory in 1806. Interestingly, there are a range of combinatorial proofs,
\marginnote{If you're interested, look up `Fermat's Little Theorem necklace proof.'} which suggest
there is a deeper connection with combinatorics that might be apparent at first glance.

We shall prove the formulation of the theorem as described in Equation \eqref{eq:fermatlt}. Remember
  that in this case \(a\) is coprime with \(p\).

  To begin the proof, we consider the numbers
  \[1a,2a,3a,\ldots,(p-1)a.\]
  None of these numbers are divisible by \(p\), so each is congruent to one of the numbers
  \[1,2,3,\ldots,p-1\]
  modulo \(p\). 
  In fact, each of the numbers in the first list is congruent to a different number in the second
  list. Why is this the case? Take two numbers in the first list, say \(ja\) and \(ka\) for
  \(1\le j,k\le p-1.\) If \(ja\equiv ka\) then by the cancellation law \(j\equiv k\), and since
  both \(j\) and \(k\) are less than \(p\), the two must be equal. Note that this is where we
  use the assumption that \(a\) is coprime with \(p\), for otherwise the cancellation of \(a\)
  from both sides is invalid.

  Hence we have shown that each of the numbers in the first list is congruent to exactly
  one number in the second, because if any two numbers in the first list are congruent then
  they are actually the same number. As a result the product of the two sets must be congruent to each other.
  That is, we have \[a\cdot2a\cdot3a\cdot\cdots\cdot (p-1)a\equiv 1\cdot2\cdot3\cdot\cdots\cdot (p-1)\pmod{p}.\]
  On both sides we discover a \((p-1)!\) term and on the LHS we have \(p-1\) copies of \(a\), so the congruence becomes
  \[a^{p-1}(p-1)!\equiv(p-1)!\pmod{p}.\]
  We observe that \((p-1)!\) consists only of factors less than \(p\) and is therefore not divisible by \(p\),
  by the Fundamental Theorem of Arithmetic. Therefore, on cancelling \((p-1)!\) from both sides, we obtain
  \[a^{p-1}\equiv 1\pmod{p}.\]

  Fermat's Little Theorem holds an interesting place in the theory of primality testing, the area of mathematics
  dedicated to finding ways of telling whether a number is prime or not.\marginnote{It's very
    easy to multiply large prime numbers together, but given a large number it is \emph{very difficult} to determine
    if it is prime. This simple idea is the basis of many ideas in cryptography.} Rather frustratingly, the converse
  of Fermat's Little Theorem is not true. If \(a^{p-1}\equiv 1\pmod{p}\), then \(p\) need not be prime. In fact, there
  are numbers that look very much like they are primes based on Fermat's Little Theorem, but are in fact composite; these
  are called \emph{Carmichael numbers}. A Carmichael
  number is a composite number \(n\) that satisfies \[a^n\equiv a\pmod{n}\]
  for all integers \(a\). Thus from the perspective of Fermat's Little Theorem Carmichael numbers look very much like
  they are prime, but they are actually composite. The smallest Carmichael number is \(561\).  \marginnote{Carmichael numbers are named after Robert Carmichael, an American mathematician who was an early pioneer
    in this area of primality testing.}


  \section*{Problems}\addcontentsline{toc}{subsection}{Problems}
  \begin{enumerate}
  \item Prove that Equation \eqref{eq:fermatlt2} holds for all positive integers \(a\) by induction.
  \item Let \(\phi(m)\) denote the number of numbers less than \(m\) that are coprime with \(m\),
    \marginnote{This function \(\phi(m)\) is called \emph{Euler's Totient Function}.}
      and let the numbers \[k_1,k_2,\ldots,k_{\phi(m)}\] be those numbers.
      Using a method similar to Ivory's proof of Fermat's Little Theorem, show that if \(a\) is coprime with
      \(m\), then \marginnote{Can you see how this is a generalisation of Fermat's Little Theorem?}
      \[a^{\phi(m)}\equiv 1\pmod{m}.\]
  \end{enumerate}
\end{document}
