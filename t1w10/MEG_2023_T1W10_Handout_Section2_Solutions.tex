%\pagewidth 210mm
%\pageheight 297mm
\documentclass[a4paper,10pt]{article}

\usepackage[pdfusetitle]{hyperref}

\usepackage[a4paper]{geometry}
\usepackage[bf,tiny]{titlesec}
\usepackage{fancyhdr}

%\usepackage[indent=0pt,skip=10pt]{parskip}

\usepackage{amsmath}
\usepackage{amsthm}
\newtheorem{theorem}{Theorem}
\usepackage{amssymb}
\let\mathbbalt\mathbb

\usepackage{fontspec}
\usepackage{unicode-math}
\let\mathbb\mathbbalt

\newcommand{\naturals}{\mathbb{N}}
\newcommand{\reals}{\mathbb{R}}
\newcommand{\rationals}{\mathbb{Q}}
\newcommand{\integers}{\mathbb{Z}}

\newcounter{problemno}
\setcounter{problemno}{1}
\newcommand{\problem}{\section*{\textbf{Problem \theproblemno}}\refstepcounter{problemno}}
\newcounter{interludeno}
\setcounter{interludeno}{1}
\newcommand{\interlude}[1]{\par\noindent\\\fbox{\begin{minipage}{\linewidth}\textbf{Interlude \theinterludeno.\enspace}#1\end{minipage}}\\\refstepcounter{interludeno}}

\newcounter{questionno}
\setcounter{questionno}{1}
\newcommand{\question}[1]{\par\noindent{\thequestionno.\enspace}#1\refstepcounter{questionno}\\}

\newcommand{\thetitle}{Term 1 Week 10 Handout Solutions}

\renewcommand\thesection{\S{\arabic{section}}.}
\renewcommand\thesubsection{\S{\arabic{section}.\arabic{subsection}}}

\title{MEG 2023 Term 1 Week 10 Handout Solutions}
\author{Nathan Wong\and Tom Yan}
\date{2023}
\begin{document}
\noindent Melbourne High School\\
Maths Extension Group 2023\\
\textbf{\thetitle}\\
\section{Problems relating to combinatorial games}
\section{Modular arithmetic and congruences}
\subsection{Theory and notation}
\subsection{Exercises}
\begin{enumerate}
\item If \(a\equiv b\pmod{m}\) then \(a-b=km\) for some integer \(k\).
  Then \(b-a=-km\), and hence \(b\equiv a\).

  If \(b\equiv c\) then let \(b-c=mt\) for some integer \(t\).
  Then \(a-c=a-(b-mt)=km+mt=m(k+t)\). Hence \(a\equiv c\).
\item A number \(n\) in decimal with digits \(a_0,a_1,\ldots,a_n\)
  can be written as \[n=a_0+a_110+\cdots+a_n10^n.\]
  Since \(10\equiv 1\) modulo \(3\) and \(9\),
  \[n\equiv a_0+a_1+\cdots+a_n\] and hence \(n\equiv 0\) if and only
  if its digit sum is also congruent to \(0\) modulo \(3\) or \(9\).
\item This proof is much the same as the preceding one except that we
  have
  \[ 10^n\equiv 
    \begin{cases}
      1& \text{if } n \text{ is even}\\
      -1& \text{if } n \text{ is odd}\\
    \end{cases}
    \pmod{11}
  \]
  Therefore \[n\equiv a_0-a_1+a_2-a_3+\cdots+(-1)^na_n.\]
  That is, a number is divisible by \(11\) if the difference between
  the sum of all the digits in the even places and the digits in the
  odd places is divisible by \(11\). (This is the same thing as saying
  that divisibility depends on the alternating sum of the digits.)
\item Let \(a-b=2tm\) for some integer \(t\). Then \[a^2=b^2+4btm+4t^2m^2\]
  which implies \[a^2-b^2=4m(bt+t^2m).\]
  Therefore \(a^2\equiv b^2\pmod{4m}\).

  The more general result requires the binomial theorem. Again let \(a=tkm+b\) for some
  integer \(t\). Then by the binomial theorem
  \[
    \begin{split}
    a^k&=\sum_{i=0}^{k}b^{k-i}(tkm)^i.\\
       &=b^k+\binom{k}{1}b^{k-1}(tkm)+\binom{k}{2}b^{k-2}(tkm)^2+\cdots+(tkm)^k.\\
      &=b^k+b^{k-1}(tk^2m)+\binom{k}{2}b^{k-2}(tkm)^2+\cdots+(tkm)^k.
    \end{split}
  \]
  Hence \[a^k-b^k=b^{k-1}(tk^2m)+\binom{k}{2}b^{k-2}(tkm)^2+\cdots+(tkm)^k.\]
  Each term on the RHS is clearly divisible by \(k^2m\). This completes the proof.

\end{enumerate}
\subsection{Linear congruences}
\subsection{Problems}
\begin{enumerate}
\item If the cupcake vendor only sells in batches of \(97\), I need to order
  \(97x\) cupcakes, for some positive integer \(x\). There are only \(105\)
  people at my party, and the remainder after everyone eats the same amount needs
  to be \(13\). Therefore the linear congruence to solve is \[97x\equiv13\pmod{105}.\]
  We find that the solution is \(x\equiv 64\), which is greater than \(50\). Therefore
  I cannot satisfy the needs of my party.
\item Finding the multiplicative inverse \(a^{-1}\) is really the same as solving
  the congruence \[ax\equiv 1\pmod{m},\] which has \(1\) solution when \(a\) and \(m\)
  are coprime.
\item From the preceding problem, every number in the list \[1,2,3,...,(p-1)\] has exactly
  one multiplicative inverse in the same list. It is clear that apart from \(1\) and \((p-1)\),
  each number will not be paired with itself. Hence taking the product of \(2,3,\ldots,(p-2)\) is
  the same as taking the product of pairs that all are congruent to \(1\) modulo \(p\); hence
  \[2\cdot3\cdot4\cdots(p-2)\equiv 1\pmod{p}.\]
  Multiplying both sides by \(p-1\) yields the desired result.
\item Both sets have \(m\) numbers, so we only need to show that none of the numbers in the first
  set are congruent. If \(ia\equiv ja\pmod{m}\) for \(0\le i,j<m\), by the cancellation law
  we have \(i\equiv j\pmod{m}\). But since \(0\le i,j<m\), we must have \(i=j\), and therefore
  no two numbers in the first set are congruent. This completes the proof.

  Now for the linear congruence. We may assume that \(b\) is already a least residue; that is, \(0\le b<m-1\).
  We have already shown that the numbers \[0,a,2a,\ldots,(m-1)a\] run through a complete set of
  least residues, which must include \(b\). Therefore one of those numbers is congruent to \(b\), say \(ka\equiv b\pmod{m}\),
  and this number \(k\) is the solution we are looking for.
\end{enumerate}
\end{document}