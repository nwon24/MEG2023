\documentclass{article}
\usepackage{graphicx} % Required for inserting images

\title{Maths Games Team selection test}
\author{Tom Yan}
\date{June 2023}

\begin{document}

\maketitle
TOTAL MARKS: $2 + 2 + 2 + 3 + 3 + 4 + 5 = 21$ \\\\
1: A/N  AMC Senior 2020/14\\\\ 2: N AMC Senior 2019/11\\\\3. G AIMO 2006/1 \\\\ 4: G AMC 2020/24  \\\\5. A AMC Upper Primary 2018/29 \\\\ 6. A/N Crux Mathematicorum \\\\7. C/G AMO 2005/2
\newpage
\section{Introduction}
1. Given that $x$ and $y$ are both integers and $2^{x+1}+2^x=3^{y+2}-3^y$, what is the value of $x+y$? (2 marks) \\\\ 2. The 5-digit number $P679Q$ is divisible by 72. What is the digit $P$ equal to? (2 marks) \\\\ 3. Consider a cube of edge $9cm$. In the centre of three different and not opposite faces a square hole is made which goes through to the opposite face. Each side of each hole has width $3cm$. What is the surface area, in $cm^2$, of the remaining solid? (2 marks) \\\\
4. Five squares of unit area are circumscribed by a circle as shown. What is the radius of the circle? (3 marks) \\ 
\centerline{\includegraphics[width=2in, height=2in]{circ.png}}
5. Jan and Jill are both on a circular track. \\ Jill runs at a steady pace, completing each circuit 72 seconds. \\ Jan walks at a steady pace in the opposite direction and meets Jill every 56 seconds. How long does it take Jan to walk each circuit? (3 marks)\\\\
6. Show that $n^4-20n^2+4$ is composite for all integers $n$, where $n>4$. (4 marks)  \\\\
7. Consider a polyhedron whose faces are convex polygons. Show that it has at least two faces with the same number of edges. (5 marks)\\\\

\newpage
\section{Solutions:}
1. Solution: Solving, $$2^{x+1}+2^x=3^{y+2}-3^y$$ $$2^x(2+1)=3^y(3^2-1)$$ $$2^x \times 3 = 8 \times 3^y$$ Clearly $x=3$ and $y=1$ is a possible solution. Due to uniqueness of prime factorisation, this is the only solution where $x$ and $y$ are integers. Consequently $x+y = 3+1=4$. \\\\ 
2. Solution: let $N = P679Q$, the $N$ is divisible by 9 and divisible by $8$. Since any number of thousands is divisible by 8, and 800 is divisible by 8, the 5-digit number $P6800$ is a known multiple of 8 near $N$. From this $N = P6792$ and $Q=2$. \\ For $N$ to be divisible by $9$, its digit sum $P+6+7+9+2=P+24$ must be divisible by 9. Then $P+24=27$ and $P=3$. \\\\
3. Solution: Each of the six external faces of the solid is a square $9cm \times 9cm$ with a $3cm \times 3cm$ square removed. Its area is $81-9=72cm^2$. \\ The tunnel from the centre of each face towards the centre of the solid has four walls, each $3cm \times 3cm$, so the area of each tunnel is $4 \times 3 \times 3 = 36 cm^2$. Hence the total area is $6 \times (72+36) cm^2 = 648 cm^2$. \\\\
4. Solution: In the upper circle, label the lengths as shown and then equate the two radii using Pythagoras' theorem. \\
\centerline{\includegraphics[width=2in, height=1.2in]{sol.png}} \\ We have $r^2=x^2+1$ and $r^2=(3-x)^2+(1/2)^2$ $$=x^2-6x+\frac{37}{4}$$ So $$x^2=1=x^2-6x+\frac{37}{4}$$ $$6x=\frac{33}{4} \rightarrow x=\frac{11}{8} $$  Then $r=\sqrt{x^2+1}=\sqrt{\frac{185}{64}}=\frac{\sqrt{185}}{8}$ \\\\
5. Solution: In 56 seconds, Jill runs $\frac{56}{72}=\frac{7}{9}$ of the track, and so in the same 56 seconds, Jan has walked along the other $\frac{2}{9}$ of the track. In half this time which is $28$ seconds, Jan will have walked $\frac{1}{9}$ of the track. Then Jan will take $9 \times 28 = 252$ seconds to walk the complete track. \\\\
6. $n^4-20n^2+4=n^4-4n^2+4-16n^2=(n^2-2)^2-(4n)^2$, Thus we have difference of perfect squares, which factors as $$(n^2-2)^2-(4n)^2=(n^2-4n-2)(n^4+4n-2)$$ To show that this expression is always composite, it suffices to show that both factors are always at least 2. For $n\ge 5$, we have $n^2-4n-2=(n-2)^2-6 \ge 3^2-6=3$, and $n^2+4n-2=(n+2)^2-6 \ge 7^2-6=43$, so both factors are greater than 2. Therefore, $n^4-20n^2+4$ is composite for all integers $n > 4$. \\\\
7. Suppose for the sake of contradiction that there exists a polyhedron whose faces are convex but no two of its faces have the same number of edges. Let its face with with the largest number of edges be called $X$ and let $X$ have $n$ edges. Since there cannot be another face with $n$ edges, all other faces must have either $n-1, n-2, \ldots, 5, 4$ or 3 edges. This makes a total of $n-3$ possibilities. However since $X$ has $n$ edges there must be at least $n$ other faces in the polyhedron, one for each such edge. But we have already shown that there are only $n-3$ possibilities for the number of edges of these $n$ other faces. This means that two of these other faces must have the same number of edges - a contradiction. \\ Thus the original assumption is wrong and hence there are two faces with the same number of edges. 

\end{document} 
