\documentclass[a4paper,addpoints,11pt]{exam}
\usepackage{fontspec}
\usepackage{unicode-math}
\usepackage{graphicx}
\begin{document}
\noindent Melbourne High School\\
\noindent \textbf{MAV Maths Games Day and SEHS Maths Games Day Selection Test 2023}\\

\begin{center}
  {\fbox{\parbox{5.5in}{\centering Questions 1--6 require only
        a numerical answer. Write it in the box provided at the bottom of the page. Questions
        7--8 require a written response.}}}
  \end{center}
\vspace{5mm}
\makebox[0.75\textwidth]{Name and form:\enspace\hrulefill}

\begin{questions}
\question[2] Given that $x$ and $y$ are both integers and $2^{x+1}+2^x=3^{y+2}-3^y$, what is the value of $x+y$?   
\question[2] The 5-digit number $P679Q$ is divisible by 72. What is the digit $P$ equal to?   
\question[2] Consider a cube of edge $9$ cm. In the centre of three different and not opposite faces a square hole is made which goes through to the opposite face. Each side of each hole has width $3$ cm. What is the surface area, in cm$^2$, of the remaining solid?   
\question[3] Five squares of unit area are circumscribed by a circle as shown. What is the radius of the circle?   \[{\includegraphics[width=2in, height=2in]{circ.png}}\]
\question[3] Jan and Jill are both on a circular track.  Jill runs at a steady pace, completing each circuit 72 seconds.  Jan walks at a steady pace in the opposite direction and meets Jill every 56 seconds. How long does it take Jan to walk each circuit?
\question[3] Quadratic polynomials \(P(x)\) and \(Q(x)\) have leading
coefficients \(2\) and \(-2\), respectively. The graphs of both polynomials
pass through the points \((16,54)\) and \((20,53)\). Find \(P(0)+Q(0)\).
\question[4] Show that $n^4-20n^2+4$ is composite for all integers $n$, where $n>4$.    
\question[5] Consider a polyhedron whose faces are convex polygons. Show that it has at least two faces with the same number of edges.
\end{questions}
\medskip
\bgroup
\def\arraystretch{1.5}
\setlength\tabcolsep{7mm}
  \begin{center}
    \begin{tabular}{|c|c|c|c|c|c|c|}
      \hline Question&1\quad&2\quad&3\quad&4\quad&5\quad&6\quad\\
      \hline Answer&\quad &\quad &\quad &\quad &\quad &\quad \\
      \hline
    \end{tabular}
  \end{center}
  \egroup
  \medskip
Total marks: \qquad/\numpoints
\end{document}